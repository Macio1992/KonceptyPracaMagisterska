\documentclass[11pt, a4paper]{article}
\usepackage{polski}
\usepackage[utf8]{inputenc}
\usepackage{listings}

\begin{document}
\lstset{language=C++}

\subsection{Parametryzacja szablonów}

Parametry szablonu są określane na dwa sposoby:

\begin{enumerate}

\item \emph{parametry szablonu} – wyraźnie wspomniane jako parametry w deklaracji szablonu

\item \emph{nazwy zależne} - wywnioskowane z użycia parametrów w definicji szablonu

\end{enumerate}

W \emph{C++} nazwa nie może być użyta bez wcześniejszej deklaracji. To wymaga od użytkownika ostrożnego traktowania definicji szablonów. Np. w definicji funkcji \verb#kwadrat# nie ma widocznej deklaracji symbolu *. Jednak, podczas inicjalizacji szablonu \verb#kwadrat<int># kompilator może sprowadzić symbol * do (wbudowanego) operatora mnożenia dla wartości \verb#int#. Dla wywołania \verb#kwadrat(zespolona(2.0))#, operator * zostałby rozwiązany do (zdefiniowanego przez użytkownika) operatora mnożenia dla wartości \verb#zespolona#. Symbol * jest więc \emph{nazwą zależną} w definicji funkcji \verb#kwadrat#. Oznacza to, że jest to ukryty parametr definicji szablonu. Możemy uczynić z operacji mnożenia formalny parametr:

\begin{lstlisting}[frame=single]
template<typename Multiply, typename T>
T square(T x) {
   return Multiply() (x,x);
}
\end{lstlisting}

Pod-wyrażenie \verb#Multiply()# tworzy obiekt funkcji, który wprowadza operację mnożenia wartości typu \verb#T#. Pojęcie \emph{nazw zależnych} pomaga utrzymać liczbę jawnych argumentów.

\end{document}