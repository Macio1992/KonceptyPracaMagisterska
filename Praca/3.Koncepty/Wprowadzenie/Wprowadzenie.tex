\documentclass[11pt, a4paper]{article}
\usepackage{polski}
\usepackage[utf8]{inputenc}
\usepackage{listings}

\begin{document}
\lstset{language=C++}

%\subsection*{Wprowadzenie}
W 1987 próbowano projektować szablony z odpowiednimi interfejsami. Chciano by szablony:

\begin{itemize}

\item były w pełni ogólne i wyraziste
\item by nie wykorzystywały większych zasobów w porównaniu do kodowania ręcznego
\item by miały dobrze określone interfejsy

\end{itemize}

\noindent Długo nie dało się osiągnąć tych trzech rzeczy, ale za to osiągnięto:

\begin{itemize}

\item \emph{kompletność Turinga}\footnote{(ang. Turing Completness) umiejętność do rozwiązania każdego zadania, czyli udzielenie odpowiedzi na każde zadanie. Program, który jest kompletny według Turinga może być wykorzystany do symulacji jakiejkolwiek 1-taśmowej maszyny Turinga}
\item lepszą wydajność (w porównaniu do kodu pisanego ręcznie)
\item kiepskie interfejsy (praktycznie \emph{typowanie kaczkowe czasu kompilacji})\footnote{(ang. duck typing) rozpoznanie typu obiektu, nie na podstawie deklaracji, ale przez badanie metod udostępnionych przez obiekt}

\end{itemize}

Brak dobrze określonych interfejsów prowadzi do szczególnie złych wiadomości błędów. Dwie pozostałe właściwości uczyniły z szablonów sukces.

Rozwiązanie problemu specyfikacji interfejsu zostało, przez Alexa Stepanova nazwane \textbf{konceptami}. \textbf{Koncept} to zbiór wymagań argumentów szablonu. Można też go nazwać systemem typów dla szablonów, który obiecuje znacząco ulepszyć diagnostyki błędów i zwiększyć siłę ekspresji, taką jak przeciążanie konceptowe oraz częściową specjalizację szablonu funkcji.

Koncepty (\emph{The Concepts TS} zostały opublikowane i zaimplementowane w wersji 6.1 kompilatora GCC w kwietniu 2016 roku. Fundamentalnie to predykaty czasu kompilacji typów i wartości. Mogą być łączone zwykłymi operatorami logicznymi (\verb#&&#, \verb#||#, \verb#!#)

%\addcontentsline{toc}{subsection}{Wprowadzenie}

\end{document}