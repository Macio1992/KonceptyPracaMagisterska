\documentclass[11pt, a4paper]{article}
\usepackage{polski}
\usepackage[utf8]{inputenc}
\usepackage{listings}

\begin{document}
\lstset{language=C++}

\subsection{Definicja konceptu}

Rozróżniamy dwa rodzaje konceptów:\newline\newline
\noindent\textbf{Zmienna konceptowa} - jest typem czasu kompilacji i nie niesie za sobą żadnych kosztów czasu wykonania.\newline

\noindent Najprostsza forma zmiennej konceptowej:
\begin{lstlisting}[frame=single]
template<template T>
concept bool zmienna_konceptowa = true;
\end{lstlisting}

Taka zmienna nie może być zadeklarowana z jakimkolwiek innym typem niż \verb#bool# oraz bez inicjalizatora. Błąd pojawi się też, gdy inicjalizatorem nie będzie ograniczone wyrażenie.

\noindent Przykład użycia:
\begin{lstlisting}[frame=single]
template<template T>
requires zmienna_konceptowa<T>
void f(T t){
   std::cout << t << "\n";
}
\end{lstlisting}

\noindent\textbf{Funkcja konceptowa} - wygląda i zachowuje się jak zwykła funkcja.\newline
\begin{lstlisting}[frame=single]
template<template T>
concept bool funkcja_konceptowa(){
   return true;
}
\end{lstlisting}

Funkcja konceptowa nie może:
\begin{itemize}
\item być zadeklarowana z żadnym specyfikatorem funkcji w deklaracji
\item zwracać żadnego innego typu niż \verb#bool#
\item mieć żadnych elementów w liście parametrów
\item mieć innego ciała niż \verb#{ return E; }#, gdzie \verb#E# to wyrażenie ograniczone
\end{itemize}

\newpage
\noindent Przykład użycia:
\begin{lstlisting}[frame=single]
template<template T>
requires funkcja_konceptowa<T>()
void f(T t){
   std::cout << t << "\n";
}
\end{lstlisting}

\end{document}