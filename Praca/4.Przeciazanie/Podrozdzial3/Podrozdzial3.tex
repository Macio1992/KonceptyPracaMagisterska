\documentclass[11pt, a4paper]{article}
\usepackage{polski}
\usepackage[utf8]{inputenc}
\usepackage{listings}

\linespread{1.3}

\begin{document}
\lstset{language=C++}

\subsection{Semantyczne udoskonalanie}

W niektórych przypadkach udoskonalenia są czysto semantyczne. Nie dostarczają operacji, które kompilator może wykorzystać do odróżnienia przeciążeń. W rzeczywistości ten problem pojawia się w standardowej hierarchii iteratorów: \emph{iteratory input} i \emph{iteratory forward} dzielą dokładnie te same zestawy operacji.

Pojęciowo \emph{iterator input} jest iteratorem reprezentującym pozycję w strumieniu wejściowym. Ponieważ jest zwiększany, poprzednie elementy są konsumowane. Oznacza to, że wcześniej dostępne elementy nie są już dostępne przez iterator lub dowolną jego kopię. W przeciwieństwie do tego, \emph{iterator forward} nie konsumuje elementów przy zwiększaniu. Wcześniej dostępne elementy mogą być uzyskane dzięki kopiom. Jest to zazwyczaj określane mianem właściwości \emph{multipass}. Jest to czysto semantyczna własność.

\begin{lstlisting}[frame=single]
template<typename I>
concept bool Input_iterator() {
  return Regular<I>() && requires (I i) {
    typename value_type_t<I>;
    { *i } -> value_type_t<I> const&;
    { ++i } -> I&;
  };
}

template<typename I>
concept bool Forward_iterator() {
  return Input_iterator<I>();
}
\end{lstlisting}

Wszystkie wymagania składniowe są zdefiniowane w koncepcie \verb#Input_i-# \newline \verb#terator#. Koncept \verb#Forward_iterator# zawiera tylko \verb#Input_iterators#. Innymi słowy, zestaw wymagań \verb#Forward_iterator# jest dokładnie taki sam, jak \verb#Input_iterator#. Jeśli próbujemy zdefiniować przeciążenia wymagające tych konceptów, wynik byłby zawsze dwuznaczny (ani lepszy od drugiego). Zróżnicowanie pomiędzy tymi konceptami jest tak naprawdę przydatny. Na przykład jeden z konstruktorów \verb#vector# ma bardziej wydajną implementację \emph{iteratorów forward} niż dla \emph{iteratorów input}.

\begin{lstlisting}[frame=single]
template<Object_type T, Allocator_of<T> A>
class vector {
  template<Input_iterator I>
    requires Same_as<T, value_type_t<I>>()
  vector(I first, I limit) {
    for ( ; first != limit; ++first)
      push_back(*first);
  }

  template<Forward_iterator I>
    requires Same_as<T, value_type_t<I>>()
  vector::vector(I first, I limit) {
    reserve(distance(first, limit)); 
      // 1 allocation
    insert(begin(), first, limit);
  }
  // ...
\end{lstlisting}

To nie zadziała, jeśli kompilator nie może odróżnić \verb#Forward_iterator# z \verb#Input_iterator#. 

Można to naprawić dodając nowe wymagania składniowe do \verb#Forward_iterator#, które odnoszą się do jego rangi w hierarchii iteratorów. To tradycyjnie zostało zrobione przy użyciu \emph{tag dispatch}. Łączenie \emph{etykiety klasy}\footnote{(ang. tag class) Pusta klasa w hierarchii dziedziczenia} z typem iteratora w celu wybrania odpowiedniego przeciążenia. Ten skojarzony typ to \verb#iterator_category#. Zmieniony \verb#Forward_iterator# może wyglądać tak:

\begin{lstlisting}[frame=single]
template<typename I>
  concept bool Forward_iterator() {
    return Input_iterator<I>() && requires {
      typename iterator_category_t<I>;
      requires Derived_from<I,
        forward_iterator_tag>();
    };
  }
\end{lstlisting}

Dzięki tej definicji wymagania \verb#Forward_iterator# zaliczają wymagania \verb#Input_iterator#, a kompilator może rozróżnić powyższe przeciążenia. Jako dodatkowa zaleta, używanie \emph{iteratorów random access} będzie jeszcze bardziej wydajne bo \verb#distance()# wymaga tylko jednej operacji całkowitej.

Jako inny przykład, \emph{C++17} dodaje nową kategorię iteratorów: \emph{iteratory contiguous}. \emph{Iterator contiguous} jest \emph{iteratorem random access}, którego obiekty odwoławcze są przydzielane w sąsiednich obszarach pamięci, których adresy rosną wraz z każdym przyrostem iteratora. Powoduje to otwarcie drzwi na wiele optymalizacji pamięci na niższym poziomie. Jest to oczywiście zupełnie czysta semantyka. Jeśli chcemy zdefiniować nowy koncept, musimy ją odróżnić od \verb#Random_access_iterator#. Na szczęście właśnie zdefiniowaliśmy maszynę, aby to zrobić.

\begin{lstlisting}[frame=single]
template<typename I>
concept bool Contiguous_iterator() {
  return Random_access_iterator<I>() && requires {
    requires Derived_from<I,
    contiguous_iterator_tag>();
  };
}
\end{lstlisting}

%Etykiety klasy nie są jedynym sposobem na rozwiązanie tego problemu. Używana jest istniejąca standardowa infrastruktura biblioteki. W rzeczywistości jedynymi konceptami wymagającymi tych klas znaczników są \verb#Forward_iterator# i \verb#Contiguous_iterator#. Nie potrzebujemy żadnej innej klasy tagów. Możemy po prostu użyć \emph{associated type trait}, zmiennej szablonu lub nawet dodatkowej operacji. Innymi słowy, możemy zrobić coś podobnego do poniższego kodu dla \emph{iteratorów forward}.

%begin{lstlisting}[frame=single]
%template<typename T>
%constexpr bool is_forward_iterator_v = false;

%template<typename T>
%constexpr bool is_forward_iterator_v<T*> = true;

%template<typename I>
%concept bool Forward_iterator() {
 % return Input_iterator<I>() &&
  %  is_forward_iterator_v<T*>;
%}
%end{lstlisting}

%Wszystkie te podejścia dałyby ten sam wynik; zdolność kompilatora do odróżniania przeładowań wymagających tych konceptów. Zgodnie z ogólną zasadą, ta technika powinna być używana tylko do różnicowania pojęć, które różnią się tylko w ich semantyce. Preferuj definiowanie pojęć tak, aby ich interfejsy odzwierciedlały ich różne semantyki.

\end{document}