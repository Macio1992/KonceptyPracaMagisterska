\documentclass[11pt, a4paper]{article}
\usepackage{polski}
\usepackage[utf8]{inputenc}

\author{Maciej Zbierowski}
\title{Koncepty jako sposób ograniczania argumentów szablonu}

\linespread{1.3}
\usepackage{hyperref}
\usepackage{listings}
\usepackage{amsthm}
\usepackage[]{algorithm2e}
\usepackage{standalone}

\hypersetup{
	colorlinks=true,
	linkcolor=black,
	filecolor=magenta,
	urlcolor=cyan,
	pdftitle=Praca Magisterska,
	bookmarks=true,
	pdfpagemode=FullScreen
}

\begin{document}
	\pagenumbering{gobble}
	\maketitle
	\newpage
	\tableofcontents
	\pagenumbering{arabic}
	\newpage

	\input{1.Wstep/Wstep}
	
	\documentclass[11pt, a4paper]{article}
\usepackage{polski}
\usepackage[utf8]{inputenc}
\usepackage{listings}
\usepackage{standalone}

\begin{document}
\lstset{language=C++}

\include{2.Szablony/Wstep/Wstep}

\documentclass[11pt, a4paper]{article}
\usepackage{polski}
\usepackage[utf8]{inputenc}
\usepackage{listings}

\linespread{1.3}

\begin{document}
\lstset{language=C++}

\subsection{Rozszerzanie algorytmów}

Rozwiązaniem jest dodanie kolejnego przeciążenia, które jako parametr przyjmuje kontener asocjacyjny\footnote{(ang. associative container) grupa szablonów klas w standardowej bibliotece \emph{C++}, która implementuje uporządkowane tablice asocjacyjne. Kontenery zdefiniowane w obecnej wersji standardu: set, map, multiset, multimap, unordered set, unordered multiset, unordered map, unordered multimap.}.\newline
\begin{lstlisting}[frame=single]
template<Associative_container A,
    Same_as<key_type_t<T>> T>
bool czyIstnieje(const A &assoc, const T &value) {
   return assoc.find(value) != assoc.end();
}
\end{lstlisting}

Ta wersja funkcji \verb#czyIstnieje()# ma tylko dwa ograniczenia: \verb#A# musi być typu \verb#Associative_container#, a typ \verb#T# musi być taki sam jak typ klucza \verb#A# (\verb#key_type_t<A>#). W przypadku kontenerów asocjacyjnych po prostu wyszukujemy wartość przy użyciu \verb#find()#, a następnie sprawdzamy, czy znaleźliśmy ją przez porównanie z \verb#end()#. To prawdopodobnie szybsze rozwiązanie niż wyszukiwanie sekwencyjne. W przeciwieństwie do wersji \verb#Sequence#, \verb#T# nie musi być typu \verb#Equality_comparable#. Wynika to z faktu, że dokładne wymagania \verb#T# są określone przez kontener asocjacyjny, a wymogi te są zwykle określane przez osobny komparator lub funkcję haszującą. \newline

\noindent Koncept \verb#Associative_container#:

\begin{lstlisting}[frame=single]
template<typename S>
concept bool Associative_container() {
  return Regular<S> && Range<S>() && 
    requires {
      typename key_type_t<S>;
      requires Object_type<key_type_t<S>>;
    } &&
    requires (S s, key_type_t<S> k) {
      { s.empty() } -> bool;
      { s.size() } -> int;
      { s.find(k) } -> iterator_t<S>;
      { s.count(k) } -> int;
    };
}
\end{lstlisting}

\noindent Kontener asocjacyjny jest typu \verb#Regular#, definiuje \verb#Range# elementów, ma \verb#key_type# (który może różnić się od wartości \verb#value_type#), a także zestaw operacji, w tym \verb#find()#, itd.

Podobnie jak poprzednio w przypadku \verb#Sequence#, nie jest to wyczerpująca lista wymagań dla kontenera asocjacyjnego. Nie dotyczy wstawiania i usuwania, a także wyklucza szczególne wymagania dotyczące iteratorów \verb#const#. Ponadto nie opisaliśmy dokładnie tego, jak oczekujemy, że zachowają się funkcje \verb#size()#, \verb#empty()#, \verb#find()# i \verb#count()#.

Ten koncept dotyczy wszystkich kontenerów asocjacyjnych z biblioteki standardowej \emph{C++} (\verb#set#, \verb#map#, \verb#unordered_multiset#, itp.). Obejmuje również te niestandardowe, zakładając, że narażają interfejs. Na przykład przeciążenie to będzie działało dla wszystkich kontenerów asocjacyjnych typu \verb#Q#(\verb#QSet<T>#, \verb#QHash<T>#).

Aby używać konceptów do rozwijania algorytmów, należy zrozumieć, jak kompilator wybiera pomiędzy wersją \verb#Sequence# a \verb#Associative_container#. Innymi słowy, co się dzieje gdy wywoływana jest funkcja \verb#czyIstnieje()#

\begin{lstlisting}[frame=single]
std::vector<int> v { ... };
std::set<int> s { ... };

if (czyIstnieje(v, 42)) // (1)
   //...
if (czyIstnieje(s, 42)) // (2)
   //...
\end{lstlisting}

\noindent(1) - wywołuje przeciążenie \verb#Sequence#\newline
(2) - wywołuje przeciążenie \verb#Associative_container#\newline

\noindent Dla każdego wywołania \verb#czyIstnieje# kompilator określa, która funkcja jest wywoływana na podstawie podanych argumentów. Nazywa się to \emph{rozwiązaniem przeciążenia}\footnote{(ang. overload resolution)}. Jest to algorytm, który próbuje znaleźć jedną najlepszą funkcję (wśród jednego lub więcej kandydatów), aby ją wywołać na podstawie podanych argumentów. Oba wywołania funkcji odnoszą się do szablonów, więc kompilator wykonuje dedukcję argumentów szablonu, a potem formuje specjalizację deklaracji w oparciu o wyniki. W obydwu przypadkach dedukcja i zastąpienie powiodą się w zwykły i przewidywalny sposób, dlatego w każdym punkcie wywołania musimy wybrać jedną z dwóch specjalizacji. W tym miejscu ograniczenia wchodzą w grę. Tylko funkcje których ograniczenia są spełnione mogą być wybrane przez rozwiązanie przeciążenia. Aby określić, czy ograniczenia funkcji są spełnione, zastępujemy dedukowane argumenty szablonu powiązanymi ograniczeniami deklaracji szablonu funkcji, a następnie oceniamy wynikowe wyrażenie. Ograniczenia są spełnione, gdy substytucja się powiedzie, a wyrażenie okaże się prawdziwe.

W pierwszym wywołaniu, dedukowane argumenty szablonu to \verb#vector<int># i \verb#int#. Argumenty te spełniają ograniczenia \verb#Sequence#, ale nie tych \newline \verb#Asociative_container#, ponieważ \verb#vector# nie ma \verb#find()# lub \verb#count()#. Dlatego kandydat \verb#Asociative_container# zostaje odrzucony, pozostawiając tylko kandydata \verb#Sequence#. W drugim wywołaniu, dedukowane argumenty to \verb#set<int># i \verb#int#. Rozwiązanie jest odwrotne do poprzedniego: \verb#set# nigdy nie jest \verb#Sequence#, ponieważ brakuje mu operacji \verb#front()# i \verb#back()#, tak więc kandydat jest odrzucany, a rozwiązanie przeciążenia wybiera kandydata \verb#Asociative_container#. To działa, ponieważ ograniczenia obu przeciążeń są wystarczająco wyczerpujące, aby zapewnić, że kontener spełnia ograniczenia jednego szablonu lub drugiego, ale nie obu. Sytuacja jest nieco bardziej interesująca, jeśli chcemy dodać więcej przeciążeń tego algorytmu. Możemy rozszerzyć algorytm dla konkretnych typów lub szablonów, tak jak mogliśmy to zrobić bez konceptów. Zasadniczo możemy określić prawidłowe definicje funkcji dla tych typów. Jeśli będziemy mieli szczęście, wiele z tych nowych przeładowań będzie miało identyczne definicje.

Ogólnie rzecz biorąc, możemy kontynuować rozszerzanie definicji algorytmu generycznego przez dodanie przeciążeń, które różnią się tylko ich ograniczeniami. Są trzy przypadki, które trzeba wziąć pod uwagę podczas przeładowywania z konceptami:

\begin{enumerate}
\item Rozszerzać definicję poprzez dostarczenie przeciążenia, które działa dla zupełnie innego zestawu typów. Ograniczenia tych nowych przeładowań byłyby wzajemnie wykluczające lub miałyby minimalną ilość nakładania się na istniejące ograniczenia.

\item Dostarczać zoptymalizowaną wersję istniejącego przeciążenia, specjalizując ją w podzbiorze swoich argumentów. Wymaga to utworzenia nowego przeciążenia, które ma silniejsze ograniczenia niż jego bardziej ogólna forma.

\item Dostarczać uogólnioną wersję, która jest zdefiniowana w kategoriach ograniczeń współużytkowanych przez jedno lub więcej istniejących przeładowań.

\end{enumerate}

Jeśli ograniczenia nie są rozłączne z wieloma kandydatami, mogą być opłacalne. Kompilator musi określić najlepszego kandydata na wywołanie. Jeśli jednak kompilator nie może określić najlepszego kandydata, rozwiązanie jest niejednoznaczne. Gdy w pierwszym algorytmie \verb#czyIstnieje()# zmieni się wymaganie \verb#Sequence# zamiast tylko \verb#Range#. To zminimalizuje ilość nakładania się, a zatem i prawdopodobieństwo dwuznaczności.

Ograniczenia rozłączne nie gwarantują, że połączenie będzie niedwuznaczne. Możemy na przykład spróbować zdefiniować kontener, który spełnia wymagania zarówno \verb#Sequence# i \verb#Associative_container#. W tym przypadku oba przeciążenia byłyby opłacalne, ale przeciążenie nie jest z natury lepsze od innych. Chyba że dodamy nowe przeciążenia, aby dostosować się do tego rodzaju struktury danych, wynik byłby niejednoznacznym rozwiązaniem.

\verb#Sequence# i \verb#Associative_container# tak naprawdę mają pokrywające się ograniczenia. Oba wymagają konceptu \verb#Range#. Możemy rozważyć te przeciążenia jako przykład trzeciego przypadku. To wskazuje, że może istnieć algorytm, który można zdefiniować w odniesieniu do wymagań przecinających. Ale to nie jest takie proste.

Drugi przypadek jest ważną cechą programowania generycznego w języku \emph{C++} i jest podstawą optymalizacji typów w bibliotekach generycznych. Ograniczenie subsumpcji pozwala na optymalizację generycznych algorytmów opartych na interfejsach dostarczonych przez ich argumenty.

\end{document}

\documentclass[11pt, a4paper]{article}
\usepackage{polski}
\usepackage[utf8]{inputenc}
\usepackage{listings}

\linespread{1.3}

\begin{document}
\lstset{language=C++}

\subsection{Specjalizacja algorytmów}

W niektórych przypadkach możemy definiować struktury danych z rozszerzonym zestawem właściwości lub operacji, które mogą być wykorzystane do definiowania bardziej dopuszczalnych lub bardziej wydajnych wersji algorytmu. Ten pomysł jest realizowany przez hierarchię iteratorów biblioteki standardowej.

\emph{Iteratory forward} mogą być użyte do przechodzenia przez sekwencję w jednym kierunku (do przodu) poprzez przesuwanie się po jednym elemencie naraz, używając operatora \verb#++#.\newline

\noindent Prosty koncept iteratora forward:\newline

\begin{lstlisting}[frame=single]
template<typename I>
concept bool Forward_iterator() {
  return Regular<I>() && requires (I i) {
    typename value_type_t<I>;
    { *i } -> const value_type_t<I>&;
    { ++i } -> I&;
  };
}
\end{lstlisting}

Opierając się na tym koncepcie, możemy zdefiniować dwa użyteczne algorytmy. Jeden, który przechodzi przez iterator wielokrotnymi krokami używając pętli i drugi, który oblicza liczbę kroków między dwoma iteratorami.

\begin{lstlisting}[frame=single]
template<Forward_iterator I>
void advance(I& iter, int n) {
  //(1)
  while (n != 0) { ++iter; --n; }
}
template<Forward_iterator I>
int distance(I first, I limit) {
  (2)
  for (int n = 0; first != limit; ++first, ++n);
  return n;
}
\end{lstlisting}

(1) - warunek wstępny: \verb#n >= 0# (2) - warunek wstępny: \verb#limit# jest osiągalny z \verb#first#

Parametr \verb#n# funkcji \verb#advance# musi być nieujemny bo \emph{iteratory forward} nie mogą iść do tyłu. Ale \emph{iterator bidirectional} może być użyty do wędrowania po sekwencji w oba kierunki (do przodu i do tyłu) poprzez przechodzenie po elementach naraz używając operatorów \verb#++# lub \verb#--#.

\begin{lstlisting}[frame=single]
template<typename I>
  concept bool Bidirectional_iterator() {
    return Forward_iterator<I>() && requires (I i)
    {
      { --i } -> I&;
    };
  }
\end{lstlisting}

Koncept \verb#Bidirectional_iterator# jest zbudowany na podstawie \newline \verb#Forward_iterator#. Czyli \emph{iterator bidirectional} jest \emph{iteratorem forward}, który również może poruszać się do tyłu. Zestaw wymagań konceptu \verb#Bidirecti-# \newline \verb#onal_iterator# całkowicie zalicza się do zestawu konceptu \verb#Forward_iterator#. W wyniku czego, za każdym razem gdy \verb#Bidirectional_iterator<X># jest prawdziwe (dla wszystkich \verb#X#), \verb#Forward_iterator<X># musi tez być prawdziwy. W tym przypadku mówimy, że koncept \verb#Bidirectional_iterator# \emph{udoskonala}\footnote{(ang. refine)} koncept \verb#Forward_iterator#.

To \emph{udoskonalenie} pozwala nam zdefiniować nową wersję \verb#advance()#, która może poruszać się w oba kierunki.

\begin{lstlisting}[frame=single]
template<Bidirectional_iterator I>
  void advance(I& iter, int n) {
    if (n > 0)
      while (n != 0) { ++iter; --n; }
    else if (n < 0)
      while (n != 0) { --iter; ++n; }
  }
\end{lstlisting}

Koncept \verb#Bidirectional_iterator# pozwala nam uspokoić warunek wstępny funkcji \verb#advance()#, dzięki czemu możemy użyć ujemnych wartości \verb#n#. Z drugiej strony \verb#Bidirectional_iterator# nie zawiera żadnych nowych informacji, które mogłyby pomóc nam ulepszyć \verb#distance()#. Możemy jednak zapewnić optymalizację zarówno \verb#advance()# jak i \verb#distance()# dla \emph{iteratorów random access}. Te iteratory mogą być użyte do przebycia sekwencji w dwóch kierunkach, ale mogą posuwać się do wielu elementów w jednym kroku używając operatorów + = lub - =. Możemy również policzyć odległość między dwoma iteratorami, odejmując je.

\begin{lstlisting}[frame=single]
template<typename I>
  concept bool Random_access_iterator() {
    return Bidirectional_iterator<I>() && 
      requires (I i, int n) {
      { i += n } -> I&;
      { i -= n } -> I&;
      { i - i } -> int;
    };
  }
\end{lstlisting}

Koncept \verb#Random_access_iterator# udoskonala koncept \verb#Bidirectional-# \newline \verb#_iterator#. Dodaje trzy nowe wymagane operacje. Dzięki tym operacjom możemy konstruować zoptymalizowane wersje \verb#advance()# i \verb#distance()#, które nie wymagają pętli.

\begin{lstlisting}[frame=single]
template<Random_access_iterator I>
void advance(I& iter, int n) {
   iter += n;
}
template<Random_access_iterator I>
int distance(I first, I limit) {
   return limit - first;
}
\end{lstlisting}

Te algorytmy można używać do zdefiniowania dużej liczby użytecznych operacji.

\begin{lstlisting}[frame=single]
template<Forward_iterator I, Ordered T>
  requires Same_as<T, value_type_t<I>>()
bool binary_search(I first, I limit, T const& value) {
  if (first == limit)
    return false;
  auto mid = first;
  advance(mid, distance(first, limit) / 2);
  if (value < *mid)
    return search(first, mid, value);
  else if (*mid < value)
    return search(++mid, limit, value);
  else
    return true;
}

\end{lstlisting}

Algorytm jest definiowany dla iteratorów \emph{forward}, ale oczywiście może być używany również do \emph{bidirectional} i \emph{random access}. Wersje \verb#advance()# i \verb#distance()#, które są używane, zależą od typu iteratora przekazanego do algorytmu. W przypadku \emph{iteratorów forward} i \emph{bidirectional}, algorytm jest liniowy w zakresie wielkości wejściowych. W przypadku \emph{iteratorów random access} algorytm jest znacznie szybszy, ponieważ \verb#distance()# i \verb#advance()# nie wymagają dodatkowych przejazdów sekwencji wejściowej.

Zdolność do specjalizacji algorytmów według ograniczeń i typów ma decydujące znaczenie dla wydajności bibliotek generycznych języka \emph{C++}. Koncepty znacznie ułatwiają definiowanie i wykorzystywanie tych specjalizacji. Ale jak kompilator wie, które przeciążenie wybrać?

W poprzednich przykładach wykorzystujących sekwencje i kontenery asocjacyjne, tylko jedno przeciążenie funkcji \verb#czyIstnieje()# było zawsze opłacalne, ponieważ argumenty były jednego lub drugiego, ale nie obu. Jeśli jednak wywołamy \verb#binarny_search()# z \emph{iteratorami random access}, powiedzmy, że są to wskaźniki do tablicy, wszystkie trzy przeciążenia \verb#advance()# i oba przeciążenia \verb#distance()# będą opłacalne. To ma sens. Każda implementacja tych funkcji jest doskonale zdefiniowana dla wskaźników.

W takim przypadku kompilator musi wybrać najlepszego spośród potencjalnych kandydatów. Ogólnie rzecz ujmując, \emph{C++} uważa, że jedna z funkcji jest lepsza od innej za pomocą następujących reguł:

\begin{enumerate}
\item Funkcje wymagające mniejszych lub "tańszych" konwersji argumentów są lepsze niż te wymagające większych lub bardziej kosztownych konwersji.

\item Funkcje nieszablonowe są lepsze niż specjalizacje szablonów funkcji.

\item Jedna specjalizacja szablonu funkcji jest lepsza od innej, jej typy parametrów są bardziej wyspecjalizowane. Na przykład \verb#T*# jest bardziej wyspecjalizowany niż \verb#T#, i tak samo \verb#vector<T>#, ale \verb#T*# nie jest bardziej wyspecjalizowany niż \verb#vector<T>#, ani też nie jest przeciwnie.

\textbf{Specyfikacja techniczna konceptów dodaje jeszcze jedną zasadę:}

\item Jeśli dwie funkcje nie mogą być sortowane, ponieważ mają równoważne konwersje lub są specjalizacjami szablonów funkcji o równoważnych typach parametrów, tym lepsza jest bardziej ograniczona. Są to najmniej ograniczone funkcje nie ograniczone. 

\end{enumerate}

Innymi słowy, ograniczenia działają jako łącznik dla zwykłych reguł przeciążania w \emph{C++}. Kolejność ograniczeń (bardziej ograniczona) zależy zasadniczo od porównania zestawów wymagań dla każdego szablonu w celu określenia, czy jest to ścisły nadzbiór drugiego. W celu porównania ograniczeń, kompilator najpierw analizuje powiązane ograniczenia funkcji w celu zbudowania zestawu tak zwanych ograniczeń atomowych. Są \emph{atomowe}, ponieważ nie mogą być podzielone na mniejsze części. Ograniczenia atomowe zawierają wyrażenia stałe \emph{C++} (np. \emph{type traits}) i wymagania w wyrażeniu \verb#requires#.

Na przykład, w rozwiązaniu \verb#advance()#, gdy jest wywołany z \emph{iteratorem random access}, zestaw ograniczeń dla każdego przeciążenia to:

\noindent \begin{tabular}{|p{5cm}|p{7cm}|p{1cm}|} \hline
  \hline 
  Koncept & Atomowe wymagania \\
  \hline 
  \verb#Forward_iterator# & \verb#value_type_t<I># \verb#{ *i } -> value_type_t<I> const&# \verb#{ ++i } -> I&# \\
  \hline
  \verb#Bidirectional_iterator# & \verb#value_type_t<I># \verb#{ *i } -> value_type_t<I> const&# \verb#{ ++i } -> I&# \newline \verb#{ --i } -> I&# \\
  \hline
  \verb#Random_access_iterator# & \verb#value_type_t<I># \verb#{ *i } -> value_type_t<I> const&# \verb#{ ++i } -> I&# \newline \verb#{ --i } -> I&# \newline \verb#{ i += n } -> I&# \newline \verb#{ i -= n } -> I&# \newline \verb#{ i - j } -> int# \\
  \hline
  
\end{tabular} \newline

Dla zwięzłości wyłączyłem ograniczenie \verb#Regular<I># pojawiające się w \verb#Forward_iterator#, ponieważ on(i jego wymagania) są wspólne dla wszystkich konceptów iterujących. Porównując powyższe stwierdzimy, że \verb#Bidirec-# \newline \verb#tional_iterator# ma ścisły nadzbiór wymagań \verb#Forward_iterator#, a \verb#Rand-# \newline \verb#om_access_iterator# ma ścisły nadzbiór wymagań \verb#Bidirectional_iterator#. Z tego względu \verb#Random_access_iterator# jest najbardziej ograniczony i to przeciążenie zostało wybrane. Nowa reguła przeciążania nie gwarantuje, że rozwiązanie przeciążenia odniesie sukces. W szczególności, jeśli dwóch realnych kandydatów ma nakładające się lub logicznie równoważne ograniczenia, rozwiązanie będzie niejednoznaczna. Jest kilka powodów, dla których to miałoby się zdarzyć.

\end{document}

\documentclass[11pt, a4paper]{article}
\usepackage{polski}
\usepackage[utf8]{inputenc}
\usepackage{listings}

\linespread{1.3}

\begin{document}
\lstset{language=C++}

\subsection{Semantyczne udoskonalanie}

W niektórych przypadkach udoskonalenia są czysto semantyczne. Nie dostarczają operacji, które kompilator może wykorzystać do odróżnienia przeciążeń. W rzeczywistości ten problem pojawia się w standardowej hierarchii iteratorów: \emph{iteratory input} i \emph{iteratory forward} dzielą dokładnie te same zestawy operacji.

Pojęciowo \emph{iterator input} jest iteratorem reprezentującym pozycję w strumieniu wejściowym. Ponieważ jest zwiększany, poprzednie elementy są konsumowane. Oznacza to, że wcześniej dostępne elementy nie są już dostępne przez iterator lub dowolną jego kopię. W przeciwieństwie do tego, \emph{iterator forward} nie konsumuje elementów przy zwiększaniu. Wcześniej dostępne elementy mogą być uzyskane dzięki kopiom. Jest to zazwyczaj określane mianem właściwości \emph{multipass}. Jest to czysto semantyczna własność.

\begin{lstlisting}[frame=single]
template<typename I>
concept bool Input_iterator() {
  return Regular<I>() && requires (I i) {
    typename value_type_t<I>;
    { *i } -> value_type_t<I> const&;
    { ++i } -> I&;
  };
}

template<typename I>
concept bool Forward_iterator() {
  return Input_iterator<I>();
}
\end{lstlisting}

Wszystkie wymagania składniowe są zdefiniowane w koncepcie \verb#Input_i-# \newline \verb#terator#. Koncept \verb#Forward_iterator# zawiera tylko \verb#Input_iterators#. Innymi słowy, zestaw wymagań \verb#Forward_iterator# jest dokładnie taki sam, jak \verb#Input_iterator#. Jeśli próbujemy zdefiniować przeciążenia wymagające tych konceptów, wynik byłby zawsze dwuznaczny (ani lepszy od drugiego). Zróżnicowanie pomiędzy tymi konceptami jest tak naprawdę przydatny. Na przykład jeden z konstruktorów \verb#vector# ma bardziej wydajną implementację \emph{iteratorów forward} niż dla \emph{iteratorów input}.

\begin{lstlisting}[frame=single]
template<Object_type T, Allocator_of<T> A>
class vector {
  template<Input_iterator I>
    requires Same_as<T, value_type_t<I>>()
  vector(I first, I limit) {
    for ( ; first != limit; ++first)
      push_back(*first);
  }

  template<Forward_iterator I>
    requires Same_as<T, value_type_t<I>>()
  vector::vector(I first, I limit) {
    reserve(distance(first, limit)); 
      // 1 allocation
    insert(begin(), first, limit);
  }
  // ...
\end{lstlisting}

To nie zadziała, jeśli kompilator nie może odróżnić \verb#Forward_iterator# z \verb#Input_iterator#. 

Można to naprawić dodając nowe wymagania składniowe do \verb#Forward_iterator#, które odnoszą się do jego rangi w hierarchii iteratorów. To tradycyjnie zostało zrobione przy użyciu \emph{tag dispatch}. Łączenie \emph{etykiety klasy}\footnote{(ang. tag class) Pusta klasa w hierarchii dziedziczenia} z typem iteratora w celu wybrania odpowiedniego przeciążenia. Ten skojarzony typ to \verb#iterator_category#. Zmieniony \verb#Forward_iterator# może wyglądać tak:

\begin{lstlisting}[frame=single]
template<typename I>
  concept bool Forward_iterator() {
    return Input_iterator<I>() && requires {
      typename iterator_category_t<I>;
      requires Derived_from<I,
        forward_iterator_tag>();
    };
  }
\end{lstlisting}

Dzięki tej definicji wymagania \verb#Forward_iterator# zaliczają wymagania \verb#Input_iterator#, a kompilator może rozróżnić powyższe przeciążenia. Jako dodatkowa zaleta, używanie \emph{iteratorów random access} będzie jeszcze bardziej wydajne bo \verb#distance()# wymaga tylko jednej operacji całkowitej.

Jako inny przykład, \emph{C++17} dodaje nową kategorię iteratorów: \emph{iteratory contiguous}. \emph{Iterator contiguous} jest \emph{iteratorem random access}, którego obiekty odwoławcze są przydzielane w sąsiednich obszarach pamięci, których adresy rosną wraz z każdym przyrostem iteratora. Powoduje to otwarcie drzwi na wiele optymalizacji pamięci na niższym poziomie. Jest to oczywiście zupełnie czysta semantyka. Jeśli chcemy zdefiniować nowy koncept, musimy ją odróżnić od \verb#Random_access_iterator#. Na szczęście właśnie zdefiniowaliśmy maszynę, aby to zrobić.

\begin{lstlisting}[frame=single]
template<typename I>
concept bool Contiguous_iterator() {
  return Random_access_iterator<I>() && requires {
    requires Derived_from<I,
    contiguous_iterator_tag>();
  };
}
\end{lstlisting}

%Etykiety klasy nie są jedynym sposobem na rozwiązanie tego problemu. Używana jest istniejąca standardowa infrastruktura biblioteki. W rzeczywistości jedynymi konceptami wymagającymi tych klas znaczników są \verb#Forward_iterator# i \verb#Contiguous_iterator#. Nie potrzebujemy żadnej innej klasy tagów. Możemy po prostu użyć \emph{associated type trait}, zmiennej szablonu lub nawet dodatkowej operacji. Innymi słowy, możemy zrobić coś podobnego do poniższego kodu dla \emph{iteratorów forward}.

%begin{lstlisting}[frame=single]
%template<typename T>
%constexpr bool is_forward_iterator_v = false;

%template<typename T>
%constexpr bool is_forward_iterator_v<T*> = true;

%template<typename I>
%concept bool Forward_iterator() {
 % return Input_iterator<I>() &&
  %  is_forward_iterator_v<T*>;
%}
%end{lstlisting}

%Wszystkie te podejścia dałyby ten sam wynik; zdolność kompilatora do odróżniania przeładowań wymagających tych konceptów. Zgodnie z ogólną zasadą, ta technika powinna być używana tylko do różnicowania pojęć, które różnią się tylko w ich semantyce. Preferuj definiowanie pojęć tak, aby ich interfejsy odzwierciedlały ich różne semantyki.

\end{document}

\end{document}
	
	\newpage
	
	\documentclass[11pt, a4paper]{article}
\usepackage{polski}
\usepackage[utf8]{inputenc}
\usepackage{listings}
\usepackage{standalone}

\begin{document}
\lstset{language=C++}

\section{Koncepty}

\input{3.Koncepty/Wprowadzenie/Wprowadzenie}

\documentclass[11pt, a4paper]{article}
\usepackage{polski}
\usepackage[utf8]{inputenc}
\usepackage{listings}

\begin{document}
\lstset{language=C++}

\subsection{System konceptów}

Reprezentacja definicji szablonu w \emph{C++} to zazwyczaj \emph{drzewo wyprowadzania}\footnote{(ang. Parse Tree) - uporządkowane, zakorzenione drzewo, które reprezentuje strukturę składniową łańcucha znakowego zgodnie z gramatyką bezkontekstową. Zwane również drzewem składniowym}. Używając identycznych technik kompilatora, możemy przekonwertować koncepty do takich drzew. Posiadając to, sprawdzanie konceptów możemy zaimplementować jako \emph{abstrakcyjne drzewo dopasowań}. Wygodnym sposobem implementowania takiego dopasowywania jest generowanie i porównywanie zestawów wymaganych funkcji i typów (zwane \emph{zestawami ograniczeń}) z definicji szablonów i konceptów. Definicja konceptu to zestaw równań \emph{drzewa AST}\footnote{(ang. Abstract Syntax Tree) Drzewo składniowe, drzewo składni abstrakcyjnej - drzewo etykietowane, wynik przeprowadzenia analizy składniowej zzdania (słowa) zgodnie z pewną gramatyką.} z założeniami typu. \newline

\noindent Koncepty dają dwa zamysły:

\begin{enumerate}

\item w \emph{definicjach szablonu}, koncepty działają jak reguły osądzania typowania. Jeśli \emph{drzewo AST} zależy od parametrów szablonu i nie może być rozwiązane przez otaczające środowisko typowania, wtedy musi się pojawić w strzegących ciałach konceptów. Takie zależne \emph{drzewa AST} są domniemanymi parametrami konceptów i zostaną rozwiązane przez sprawdzanie konceptów w momentach użycia.

\item w \emph{użyciach szablonów}, koncepty działają jak zestawy predykatów, które argumenty szablonu muszą spełniać. Sprawdzanie konceptów rozwiązuje domniemane parametry w momentach inicjalizacji.

\end{enumerate}

Jeśli zestaw konceptów definicji szablonu określa zbyt mało operacji, kompilacja szablonu nie powiedzie się z powodu sprawdzania konceptów. Szablon będzie w takim wypadku "prawie ograniczony". Odwrotnie, jeśli zestaw konceptów definicji szablonu określa więcej operacji niż potrzeba, niektóre inne uzasadnione użycia mogą również zawieźć sprawdzanie konceptów. Szablon będzie wtedy "nad ograniczony". Przez "inne uzasadnione" rozumie się, że sprawdzanie typów udałoby się w przypadku braku sprawdzania konceptów.

\end{document}

\documentclass[11pt, a4paper]{article}
\usepackage{polski}
\usepackage[utf8]{inputenc}
\usepackage{listings}

\begin{document}
\lstset{language=C++}

\subsection{System konceptów}

Reprezentacja definicji szablonu w \emph{C++} to zazwyczaj \emph{drzewo wyprowadzania}\footnote{(ang. Parse Tree) - uporządkowane, zakorzenione drzewo, które reprezentuje strukturę składniową łańcucha znakowego zgodnie z gramatyką bezkontekstową. Zwane również drzewem składniowym}. Używając identycznych technik kompilatora, możemy przekonwertować koncepty do takich drzew. Posiadając to, sprawdzanie konceptów możemy zaimplementować jako \emph{abstrakcyjne drzewo dopasowań}. Wygodnym sposobem implementowania takiego dopasowywania jest generowanie i porównywanie zestawów wymaganych funkcji i typów (zwane \emph{zestawami ograniczeń}) z definicji szablonów i konceptów. Definicja konceptu to zestaw równań \emph{drzewa AST}\footnote{(ang. Abstract Syntax Tree) Drzewo składniowe, drzewo składni abstrakcyjnej - drzewo etykietowane, wynik przeprowadzenia analizy składniowej zzdania (słowa) zgodnie z pewną gramatyką.} z założeniami typu. \newline

\noindent Koncepty dają dwa zamysły:

\begin{enumerate}

\item w \emph{definicjach szablonu}, koncepty działają jak reguły osądzania typowania. Jeśli \emph{drzewo AST} zależy od parametrów szablonu i nie może być rozwiązane przez otaczające środowisko typowania, wtedy musi się pojawić w strzegących ciałach konceptów. Takie zależne \emph{drzewa AST} są domniemanymi parametrami konceptów i zostaną rozwiązane przez sprawdzanie konceptów w momentach użycia.

\item w \emph{użyciach szablonów}, koncepty działają jak zestawy predykatów, które argumenty szablonu muszą spełniać. Sprawdzanie konceptów rozwiązuje domniemane parametry w momentach inicjalizacji.

\end{enumerate}

Jeśli zestaw konceptów definicji szablonu określa zbyt mało operacji, kompilacja szablonu nie powiedzie się z powodu sprawdzania konceptów. Szablon będzie w takim wypadku "prawie ograniczony". Odwrotnie, jeśli zestaw konceptów definicji szablonu określa więcej operacji niż potrzeba, niektóre inne uzasadnione użycia mogą również zawieźć sprawdzanie konceptów. Szablon będzie wtedy "nad ograniczony". Przez "inne uzasadnione" rozumie się, że sprawdzanie typów udałoby się w przypadku braku sprawdzania konceptów.

\end{document}

\documentclass[11pt, a4paper]{article}
\usepackage{polski}
\usepackage[utf8]{inputenc}
\usepackage{listings}

\begin{document}
\lstset{language=C++}

\subsection{System konceptów}

Reprezentacja definicji szablonu w \emph{C++} to zazwyczaj \emph{drzewo wyprowadzania}\footnote{(ang. Parse Tree) - uporządkowane, zakorzenione drzewo, które reprezentuje strukturę składniową łańcucha znakowego zgodnie z gramatyką bezkontekstową. Zwane również drzewem składniowym}. Używając identycznych technik kompilatora, możemy przekonwertować koncepty do takich drzew. Posiadając to, sprawdzanie konceptów możemy zaimplementować jako \emph{abstrakcyjne drzewo dopasowań}. Wygodnym sposobem implementowania takiego dopasowywania jest generowanie i porównywanie zestawów wymaganych funkcji i typów (zwane \emph{zestawami ograniczeń}) z definicji szablonów i konceptów. Definicja konceptu to zestaw równań \emph{drzewa AST}\footnote{(ang. Abstract Syntax Tree) Drzewo składniowe, drzewo składni abstrakcyjnej - drzewo etykietowane, wynik przeprowadzenia analizy składniowej zzdania (słowa) zgodnie z pewną gramatyką.} z założeniami typu. \newline

\noindent Koncepty dają dwa zamysły:

\begin{enumerate}

\item w \emph{definicjach szablonu}, koncepty działają jak reguły osądzania typowania. Jeśli \emph{drzewo AST} zależy od parametrów szablonu i nie może być rozwiązane przez otaczające środowisko typowania, wtedy musi się pojawić w strzegących ciałach konceptów. Takie zależne \emph{drzewa AST} są domniemanymi parametrami konceptów i zostaną rozwiązane przez sprawdzanie konceptów w momentach użycia.

\item w \emph{użyciach szablonów}, koncepty działają jak zestawy predykatów, które argumenty szablonu muszą spełniać. Sprawdzanie konceptów rozwiązuje domniemane parametry w momentach inicjalizacji.

\end{enumerate}

Jeśli zestaw konceptów definicji szablonu określa zbyt mało operacji, kompilacja szablonu nie powiedzie się z powodu sprawdzania konceptów. Szablon będzie w takim wypadku "prawie ograniczony". Odwrotnie, jeśli zestaw konceptów definicji szablonu określa więcej operacji niż potrzeba, niektóre inne uzasadnione użycia mogą również zawieźć sprawdzanie konceptów. Szablon będzie wtedy "nad ograniczony". Przez "inne uzasadnione" rozumie się, że sprawdzanie typów udałoby się w przypadku braku sprawdzania konceptów.

\end{document}

%\documentclass[11pt, a4paper]{article}
\usepackage{polski}
\usepackage[utf8]{inputenc}
\usepackage{listings}

\begin{document}
\lstset{language=C++}

\subsection{System konceptów}

Reprezentacja definicji szablonu w \emph{C++} to zazwyczaj \emph{drzewo wyprowadzania}\footnote{(ang. Parse Tree) - uporządkowane, zakorzenione drzewo, które reprezentuje strukturę składniową łańcucha znakowego zgodnie z gramatyką bezkontekstową. Zwane również drzewem składniowym}. Używając identycznych technik kompilatora, możemy przekonwertować koncepty do takich drzew. Posiadając to, sprawdzanie konceptów możemy zaimplementować jako \emph{abstrakcyjne drzewo dopasowań}. Wygodnym sposobem implementowania takiego dopasowywania jest generowanie i porównywanie zestawów wymaganych funkcji i typów (zwane \emph{zestawami ograniczeń}) z definicji szablonów i konceptów. Definicja konceptu to zestaw równań \emph{drzewa AST}\footnote{(ang. Abstract Syntax Tree) Drzewo składniowe, drzewo składni abstrakcyjnej - drzewo etykietowane, wynik przeprowadzenia analizy składniowej zzdania (słowa) zgodnie z pewną gramatyką.} z założeniami typu. \newline

\noindent Koncepty dają dwa zamysły:

\begin{enumerate}

\item w \emph{definicjach szablonu}, koncepty działają jak reguły osądzania typowania. Jeśli \emph{drzewo AST} zależy od parametrów szablonu i nie może być rozwiązane przez otaczające środowisko typowania, wtedy musi się pojawić w strzegących ciałach konceptów. Takie zależne \emph{drzewa AST} są domniemanymi parametrami konceptów i zostaną rozwiązane przez sprawdzanie konceptów w momentach użycia.

\item w \emph{użyciach szablonów}, koncepty działają jak zestawy predykatów, które argumenty szablonu muszą spełniać. Sprawdzanie konceptów rozwiązuje domniemane parametry w momentach inicjalizacji.

\end{enumerate}

Jeśli zestaw konceptów definicji szablonu określa zbyt mało operacji, kompilacja szablonu nie powiedzie się z powodu sprawdzania konceptów. Szablon będzie w takim wypadku "prawie ograniczony". Odwrotnie, jeśli zestaw konceptów definicji szablonu określa więcej operacji niż potrzeba, niektóre inne uzasadnione użycia mogą również zawieźć sprawdzanie konceptów. Szablon będzie wtedy "nad ograniczony". Przez "inne uzasadnione" rozumie się, że sprawdzanie typów udałoby się w przypadku braku sprawdzania konceptów.

\end{document}

\documentclass[11pt, a4paper]{article}
\usepackage{polski}
\usepackage[utf8]{inputenc}
\usepackage{listings}

\begin{document}
\lstset{language=C++}

\subsection{System konceptów}

Reprezentacja definicji szablonu w \emph{C++} to zazwyczaj \emph{drzewo wyprowadzania}\footnote{(ang. Parse Tree) - uporządkowane, zakorzenione drzewo, które reprezentuje strukturę składniową łańcucha znakowego zgodnie z gramatyką bezkontekstową. Zwane również drzewem składniowym}. Używając identycznych technik kompilatora, możemy przekonwertować koncepty do takich drzew. Posiadając to, sprawdzanie konceptów możemy zaimplementować jako \emph{abstrakcyjne drzewo dopasowań}. Wygodnym sposobem implementowania takiego dopasowywania jest generowanie i porównywanie zestawów wymaganych funkcji i typów (zwane \emph{zestawami ograniczeń}) z definicji szablonów i konceptów. Definicja konceptu to zestaw równań \emph{drzewa AST}\footnote{(ang. Abstract Syntax Tree) Drzewo składniowe, drzewo składni abstrakcyjnej - drzewo etykietowane, wynik przeprowadzenia analizy składniowej zzdania (słowa) zgodnie z pewną gramatyką.} z założeniami typu. \newline

\noindent Koncepty dają dwa zamysły:

\begin{enumerate}

\item w \emph{definicjach szablonu}, koncepty działają jak reguły osądzania typowania. Jeśli \emph{drzewo AST} zależy od parametrów szablonu i nie może być rozwiązane przez otaczające środowisko typowania, wtedy musi się pojawić w strzegących ciałach konceptów. Takie zależne \emph{drzewa AST} są domniemanymi parametrami konceptów i zostaną rozwiązane przez sprawdzanie konceptów w momentach użycia.

\item w \emph{użyciach szablonów}, koncepty działają jak zestawy predykatów, które argumenty szablonu muszą spełniać. Sprawdzanie konceptów rozwiązuje domniemane parametry w momentach inicjalizacji.

\end{enumerate}

Jeśli zestaw konceptów definicji szablonu określa zbyt mało operacji, kompilacja szablonu nie powiedzie się z powodu sprawdzania konceptów. Szablon będzie w takim wypadku "prawie ograniczony". Odwrotnie, jeśli zestaw konceptów definicji szablonu określa więcej operacji niż potrzeba, niektóre inne uzasadnione użycia mogą również zawieźć sprawdzanie konceptów. Szablon będzie wtedy "nad ograniczony". Przez "inne uzasadnione" rozumie się, że sprawdzanie typów udałoby się w przypadku braku sprawdzania konceptów.

\end{document}

\documentclass[11pt, a4paper]{article}
\usepackage{polski}
\usepackage[utf8]{inputenc}
\usepackage{listings}

\begin{document}
\lstset{language=C++}

\subsection{System konceptów}

Reprezentacja definicji szablonu w \emph{C++} to zazwyczaj \emph{drzewo wyprowadzania}\footnote{(ang. Parse Tree) - uporządkowane, zakorzenione drzewo, które reprezentuje strukturę składniową łańcucha znakowego zgodnie z gramatyką bezkontekstową. Zwane również drzewem składniowym}. Używając identycznych technik kompilatora, możemy przekonwertować koncepty do takich drzew. Posiadając to, sprawdzanie konceptów możemy zaimplementować jako \emph{abstrakcyjne drzewo dopasowań}. Wygodnym sposobem implementowania takiego dopasowywania jest generowanie i porównywanie zestawów wymaganych funkcji i typów (zwane \emph{zestawami ograniczeń}) z definicji szablonów i konceptów. Definicja konceptu to zestaw równań \emph{drzewa AST}\footnote{(ang. Abstract Syntax Tree) Drzewo składniowe, drzewo składni abstrakcyjnej - drzewo etykietowane, wynik przeprowadzenia analizy składniowej zzdania (słowa) zgodnie z pewną gramatyką.} z założeniami typu. \newline

\noindent Koncepty dają dwa zamysły:

\begin{enumerate}

\item w \emph{definicjach szablonu}, koncepty działają jak reguły osądzania typowania. Jeśli \emph{drzewo AST} zależy od parametrów szablonu i nie może być rozwiązane przez otaczające środowisko typowania, wtedy musi się pojawić w strzegących ciałach konceptów. Takie zależne \emph{drzewa AST} są domniemanymi parametrami konceptów i zostaną rozwiązane przez sprawdzanie konceptów w momentach użycia.

\item w \emph{użyciach szablonów}, koncepty działają jak zestawy predykatów, które argumenty szablonu muszą spełniać. Sprawdzanie konceptów rozwiązuje domniemane parametry w momentach inicjalizacji.

\end{enumerate}

Jeśli zestaw konceptów definicji szablonu określa zbyt mało operacji, kompilacja szablonu nie powiedzie się z powodu sprawdzania konceptów. Szablon będzie w takim wypadku "prawie ograniczony". Odwrotnie, jeśli zestaw konceptów definicji szablonu określa więcej operacji niż potrzeba, niektóre inne uzasadnione użycia mogą również zawieźć sprawdzanie konceptów. Szablon będzie wtedy "nad ograniczony". Przez "inne uzasadnione" rozumie się, że sprawdzanie typów udałoby się w przypadku braku sprawdzania konceptów.

\end{document}

\documentclass[11pt, a4paper]{article}
\usepackage{polski}
\usepackage[utf8]{inputenc}
\usepackage{listings}

\begin{document}
\lstset{language=C++}

\subsection{System konceptów}

Reprezentacja definicji szablonu w \emph{C++} to zazwyczaj \emph{drzewo wyprowadzania}\footnote{(ang. Parse Tree) - uporządkowane, zakorzenione drzewo, które reprezentuje strukturę składniową łańcucha znakowego zgodnie z gramatyką bezkontekstową. Zwane również drzewem składniowym}. Używając identycznych technik kompilatora, możemy przekonwertować koncepty do takich drzew. Posiadając to, sprawdzanie konceptów możemy zaimplementować jako \emph{abstrakcyjne drzewo dopasowań}. Wygodnym sposobem implementowania takiego dopasowywania jest generowanie i porównywanie zestawów wymaganych funkcji i typów (zwane \emph{zestawami ograniczeń}) z definicji szablonów i konceptów. Definicja konceptu to zestaw równań \emph{drzewa AST}\footnote{(ang. Abstract Syntax Tree) Drzewo składniowe, drzewo składni abstrakcyjnej - drzewo etykietowane, wynik przeprowadzenia analizy składniowej zzdania (słowa) zgodnie z pewną gramatyką.} z założeniami typu. \newline

\noindent Koncepty dają dwa zamysły:

\begin{enumerate}

\item w \emph{definicjach szablonu}, koncepty działają jak reguły osądzania typowania. Jeśli \emph{drzewo AST} zależy od parametrów szablonu i nie może być rozwiązane przez otaczające środowisko typowania, wtedy musi się pojawić w strzegących ciałach konceptów. Takie zależne \emph{drzewa AST} są domniemanymi parametrami konceptów i zostaną rozwiązane przez sprawdzanie konceptów w momentach użycia.

\item w \emph{użyciach szablonów}, koncepty działają jak zestawy predykatów, które argumenty szablonu muszą spełniać. Sprawdzanie konceptów rozwiązuje domniemane parametry w momentach inicjalizacji.

\end{enumerate}

Jeśli zestaw konceptów definicji szablonu określa zbyt mało operacji, kompilacja szablonu nie powiedzie się z powodu sprawdzania konceptów. Szablon będzie w takim wypadku "prawie ograniczony". Odwrotnie, jeśli zestaw konceptów definicji szablonu określa więcej operacji niż potrzeba, niektóre inne uzasadnione użycia mogą również zawieźć sprawdzanie konceptów. Szablon będzie wtedy "nad ograniczony". Przez "inne uzasadnione" rozumie się, że sprawdzanie typów udałoby się w przypadku braku sprawdzania konceptów.

\end{document}

%\documentclass[11pt, a4paper]{article}
\usepackage{polski}
\usepackage[utf8]{inputenc}
\usepackage{listings}

\begin{document}
\lstset{language=C++}

\subsection{System konceptów}

Reprezentacja definicji szablonu w \emph{C++} to zazwyczaj \emph{drzewo wyprowadzania}\footnote{(ang. Parse Tree) - uporządkowane, zakorzenione drzewo, które reprezentuje strukturę składniową łańcucha znakowego zgodnie z gramatyką bezkontekstową. Zwane również drzewem składniowym}. Używając identycznych technik kompilatora, możemy przekonwertować koncepty do takich drzew. Posiadając to, sprawdzanie konceptów możemy zaimplementować jako \emph{abstrakcyjne drzewo dopasowań}. Wygodnym sposobem implementowania takiego dopasowywania jest generowanie i porównywanie zestawów wymaganych funkcji i typów (zwane \emph{zestawami ograniczeń}) z definicji szablonów i konceptów. Definicja konceptu to zestaw równań \emph{drzewa AST}\footnote{(ang. Abstract Syntax Tree) Drzewo składniowe, drzewo składni abstrakcyjnej - drzewo etykietowane, wynik przeprowadzenia analizy składniowej zzdania (słowa) zgodnie z pewną gramatyką.} z założeniami typu. \newline

\noindent Koncepty dają dwa zamysły:

\begin{enumerate}

\item w \emph{definicjach szablonu}, koncepty działają jak reguły osądzania typowania. Jeśli \emph{drzewo AST} zależy od parametrów szablonu i nie może być rozwiązane przez otaczające środowisko typowania, wtedy musi się pojawić w strzegących ciałach konceptów. Takie zależne \emph{drzewa AST} są domniemanymi parametrami konceptów i zostaną rozwiązane przez sprawdzanie konceptów w momentach użycia.

\item w \emph{użyciach szablonów}, koncepty działają jak zestawy predykatów, które argumenty szablonu muszą spełniać. Sprawdzanie konceptów rozwiązuje domniemane parametry w momentach inicjalizacji.

\end{enumerate}

Jeśli zestaw konceptów definicji szablonu określa zbyt mało operacji, kompilacja szablonu nie powiedzie się z powodu sprawdzania konceptów. Szablon będzie w takim wypadku "prawie ograniczony". Odwrotnie, jeśli zestaw konceptów definicji szablonu określa więcej operacji niż potrzeba, niektóre inne uzasadnione użycia mogą również zawieźć sprawdzanie konceptów. Szablon będzie wtedy "nad ograniczony". Przez "inne uzasadnione" rozumie się, że sprawdzanie typów udałoby się w przypadku braku sprawdzania konceptów.

\end{document}

\end{document}
	
	\newpage
	
	\documentclass[11pt, a4paper]{article}
\usepackage{polski}
\usepackage[utf8]{inputenc}
\usepackage{listings}
\usepackage{standalone}

\begin{document}
\lstset{language=C++}

\section{Przeciążanie}

\documentclass[11pt, a4paper]{article}
\usepackage{polski}
\usepackage[utf8]{inputenc}
\usepackage{listings}

\linespread{1.3}

\begin{document}
\lstset{language=C++}

%\subsection*{Wprowadzenie}

Koncepty są użyteczne nie tylko w poprawianiu wiadomości błędów i precyzyjnej specyfikacji interfejsów. Zwiększają również ekspresyjność. Używane są do skracania kodu, robieniu go generycznym i zwiększania wydajności. Wyjątkowo potężną cechą jest ich rola w przeciążaniu funkcji. 

W kwietniu 2016 został wydany kompilator \emph{GCC 6.2}. Ta wersja zawierała główne unowocześnienie dwóch komponentów implementacji konceptów. Jeden z nich to generator diagnostyki, który został znacznie odnowiony, aby zapewnić dokładną diagnostykę niepowodzeń konceptu przy sprawdzaniu czy jest spełniony. Drugi to wsparcie dla przeciążania ograniczeń, które zostało całkowicie przepisane, aby zapewnić znaczne zwiększenie wydajności. W \emph{GCC} można teraz używać konceptów do projektów o znacznej wielkości i złożoności.

Niektórzy twierdzą, że wyrażenia takie jak \verb#SFINAE#\footnote{(ang. Substitution failure is not an error) sytuacja w \emph{C++} gdzie nieprawidłowe zastąpienie parametrów szablonu nie jest samo w sobie błędem}, \verb#constexpr if#\footnote{Wyrażenie, którego wartość warunku musi być kontekstowo konwertowanym stałym wyrażeniem typu bool.}, \newline \verb#static_assert#\footnote{Wykonuje sprawdzanie porównania w czasie kompilacji} i mądre techniki metaprogramowania w zupełności wystarczą do przeciążania. To oczywiście poprawne myślenie, lecz jest to obniżanie poziomu abstrakcji, co skutkuje tym, że programuje się w sposób żeby było zrobione a nie jak powinno być. Wynikiem jest więcej pracy dla programisty, zwiększona ilość błędów i mniej szans optymalizacyjnych. \emph{C++} nie jest przeznaczony do metaprogramowania szablonów. Koncepty pomagają nam podnieść poziom programowania i ułatwić kod, bez dodawania kosztów czasu wykonania. \newline

\begin{lstlisting}[frame=single]
template<Sequence S, Equality_comparable T>
  requires Same_as<T, value_type_t<S>>
bool czyIstnieje(const S &seq, const T &value) {
  for (const auto &x : range)
    if (x == value)
      return true;
  return false;
}
\end{lstlisting}

\noindent Funkcja \verb#czyIstnieje# przyjmuje sekwencję typu \verb#Sequence# jako pierwszy argument i wartość \verb#Equality comparable# jako drugi. Algorytm ma trzy ograniczenia:

\begin{itemize}

\item \verb#seq# musi być typu \verb#Sequence#
\item \verb#value# musi być typu \verb#Equality_comparable#
\item typ \verb#value# musi być taki sam jak element typu \verb#seq#

\end{itemize}

\noindent Wyrażenie \verb#value_type_t# to alias typu, który odnosi się do zdeklarowanego lub wydedukowanego typu wartości \verb#R#. Definicje konceptów \verb#Sequence# i \verb#Range# potrzebne do tego algorytmu wyglądają tak: \newline

\begin{lstlisting}[frame=single]
template<typename R>
concept bool Range() {
  return requires (R range) {
    typename value_type_t<R>;
    typename iterator_t<R>;
    { begin(range) } -> iterator_t<R>;
    { end(range) } -> iterator_t<R>;
    requires Input_iterator<iterator_t<R>>();
    requires Same_as<value_type_t<R>,
      value_type_t<iterator_t<R>>>();
  };
}

template<typename S>
concept bool Sequence() {
  return Range<R>() && requires (S seq) {
    { seq.front() } -> const value_type<S>&;
    { seq.back() } -> const value_type<S>&;
  };
}
\end{lstlisting}

Większość sekwencji posiada operacje \verb#front()# i \verb#back()#, które zwracają pierwszy i ostatni element przedziału. To nie jest w pełni rozwinięta specyfikacja sekwencji. Możemy użyć algorytmu do określenia, czy element znajduje się w dowolnej sekwencji. Niestety, algorytm nie działa w przypadku niektórych kolekcji:

\begin{lstlisting}[frame=single]
std::set<int> testSet { ... };
if (czyIstnieje(testSet, 42)) // (1)
  ...
\end{lstlisting}

(1) - błąd: brak operacji \verb#front()# lub \verb#back()#\newline

Potrzebny jest sposób, żeby jasno określić, czy klucz znajduję się w zbiorze.

%\addcontentsline{toc}{subsection}{Wprowadzenie}

\end{document}

\documentclass[11pt, a4paper]{article}
\usepackage{polski}
\usepackage[utf8]{inputenc}
\usepackage{listings}

\linespread{1.3}

\begin{document}
\lstset{language=C++}

\subsection{Rozszerzanie algorytmów}

Rozwiązaniem jest dodanie kolejnego przeciążenia, które jako parametr przyjmuje kontener asocjacyjny\footnote{(ang. associative container) grupa szablonów klas w standardowej bibliotece \emph{C++}, która implementuje uporządkowane tablice asocjacyjne. Kontenery zdefiniowane w obecnej wersji standardu: set, map, multiset, multimap, unordered set, unordered multiset, unordered map, unordered multimap.}.\newline
\begin{lstlisting}[frame=single]
template<Associative_container A,
    Same_as<key_type_t<T>> T>
bool czyIstnieje(const A &assoc, const T &value) {
   return assoc.find(value) != assoc.end();
}
\end{lstlisting}

Ta wersja funkcji \verb#czyIstnieje()# ma tylko dwa ograniczenia: \verb#A# musi być typu \verb#Associative_container#, a typ \verb#T# musi być taki sam jak typ klucza \verb#A# (\verb#key_type_t<A>#). W przypadku kontenerów asocjacyjnych po prostu wyszukujemy wartość przy użyciu \verb#find()#, a następnie sprawdzamy, czy znaleźliśmy ją przez porównanie z \verb#end()#. To prawdopodobnie szybsze rozwiązanie niż wyszukiwanie sekwencyjne. W przeciwieństwie do wersji \verb#Sequence#, \verb#T# nie musi być typu \verb#Equality_comparable#. Wynika to z faktu, że dokładne wymagania \verb#T# są określone przez kontener asocjacyjny, a wymogi te są zwykle określane przez osobny komparator lub funkcję haszującą. \newline

\noindent Koncept \verb#Associative_container#:

\begin{lstlisting}[frame=single]
template<typename S>
concept bool Associative_container() {
  return Regular<S> && Range<S>() && 
    requires {
      typename key_type_t<S>;
      requires Object_type<key_type_t<S>>;
    } &&
    requires (S s, key_type_t<S> k) {
      { s.empty() } -> bool;
      { s.size() } -> int;
      { s.find(k) } -> iterator_t<S>;
      { s.count(k) } -> int;
    };
}
\end{lstlisting}

\noindent Kontener asocjacyjny jest typu \verb#Regular#, definiuje \verb#Range# elementów, ma \verb#key_type# (który może różnić się od wartości \verb#value_type#), a także zestaw operacji, w tym \verb#find()#, itd.

Podobnie jak poprzednio w przypadku \verb#Sequence#, nie jest to wyczerpująca lista wymagań dla kontenera asocjacyjnego. Nie dotyczy wstawiania i usuwania, a także wyklucza szczególne wymagania dotyczące iteratorów \verb#const#. Ponadto nie opisaliśmy dokładnie tego, jak oczekujemy, że zachowają się funkcje \verb#size()#, \verb#empty()#, \verb#find()# i \verb#count()#.

Ten koncept dotyczy wszystkich kontenerów asocjacyjnych z biblioteki standardowej \emph{C++} (\verb#set#, \verb#map#, \verb#unordered_multiset#, itp.). Obejmuje również te niestandardowe, zakładając, że narażają interfejs. Na przykład przeciążenie to będzie działało dla wszystkich kontenerów asocjacyjnych typu \verb#Q#(\verb#QSet<T>#, \verb#QHash<T>#).

Aby używać konceptów do rozwijania algorytmów, należy zrozumieć, jak kompilator wybiera pomiędzy wersją \verb#Sequence# a \verb#Associative_container#. Innymi słowy, co się dzieje gdy wywoływana jest funkcja \verb#czyIstnieje()#

\begin{lstlisting}[frame=single]
std::vector<int> v { ... };
std::set<int> s { ... };

if (czyIstnieje(v, 42)) // (1)
   //...
if (czyIstnieje(s, 42)) // (2)
   //...
\end{lstlisting}

\noindent(1) - wywołuje przeciążenie \verb#Sequence#\newline
(2) - wywołuje przeciążenie \verb#Associative_container#\newline

\noindent Dla każdego wywołania \verb#czyIstnieje# kompilator określa, która funkcja jest wywoływana na podstawie podanych argumentów. Nazywa się to \emph{rozwiązaniem przeciążenia}\footnote{(ang. overload resolution)}. Jest to algorytm, który próbuje znaleźć jedną najlepszą funkcję (wśród jednego lub więcej kandydatów), aby ją wywołać na podstawie podanych argumentów. Oba wywołania funkcji odnoszą się do szablonów, więc kompilator wykonuje dedukcję argumentów szablonu, a potem formuje specjalizację deklaracji w oparciu o wyniki. W obydwu przypadkach dedukcja i zastąpienie powiodą się w zwykły i przewidywalny sposób, dlatego w każdym punkcie wywołania musimy wybrać jedną z dwóch specjalizacji. W tym miejscu ograniczenia wchodzą w grę. Tylko funkcje których ograniczenia są spełnione mogą być wybrane przez rozwiązanie przeciążenia. Aby określić, czy ograniczenia funkcji są spełnione, zastępujemy dedukowane argumenty szablonu powiązanymi ograniczeniami deklaracji szablonu funkcji, a następnie oceniamy wynikowe wyrażenie. Ograniczenia są spełnione, gdy substytucja się powiedzie, a wyrażenie okaże się prawdziwe.

W pierwszym wywołaniu, dedukowane argumenty szablonu to \verb#vector<int># i \verb#int#. Argumenty te spełniają ograniczenia \verb#Sequence#, ale nie tych \newline \verb#Asociative_container#, ponieważ \verb#vector# nie ma \verb#find()# lub \verb#count()#. Dlatego kandydat \verb#Asociative_container# zostaje odrzucony, pozostawiając tylko kandydata \verb#Sequence#. W drugim wywołaniu, dedukowane argumenty to \verb#set<int># i \verb#int#. Rozwiązanie jest odwrotne do poprzedniego: \verb#set# nigdy nie jest \verb#Sequence#, ponieważ brakuje mu operacji \verb#front()# i \verb#back()#, tak więc kandydat jest odrzucany, a rozwiązanie przeciążenia wybiera kandydata \verb#Asociative_container#. To działa, ponieważ ograniczenia obu przeciążeń są wystarczająco wyczerpujące, aby zapewnić, że kontener spełnia ograniczenia jednego szablonu lub drugiego, ale nie obu. Sytuacja jest nieco bardziej interesująca, jeśli chcemy dodać więcej przeciążeń tego algorytmu. Możemy rozszerzyć algorytm dla konkretnych typów lub szablonów, tak jak mogliśmy to zrobić bez konceptów. Zasadniczo możemy określić prawidłowe definicje funkcji dla tych typów. Jeśli będziemy mieli szczęście, wiele z tych nowych przeładowań będzie miało identyczne definicje.

Ogólnie rzecz biorąc, możemy kontynuować rozszerzanie definicji algorytmu generycznego przez dodanie przeciążeń, które różnią się tylko ich ograniczeniami. Są trzy przypadki, które trzeba wziąć pod uwagę podczas przeładowywania z konceptami:

\begin{enumerate}
\item Rozszerzać definicję poprzez dostarczenie przeciążenia, które działa dla zupełnie innego zestawu typów. Ograniczenia tych nowych przeładowań byłyby wzajemnie wykluczające lub miałyby minimalną ilość nakładania się na istniejące ograniczenia.

\item Dostarczać zoptymalizowaną wersję istniejącego przeciążenia, specjalizując ją w podzbiorze swoich argumentów. Wymaga to utworzenia nowego przeciążenia, które ma silniejsze ograniczenia niż jego bardziej ogólna forma.

\item Dostarczać uogólnioną wersję, która jest zdefiniowana w kategoriach ograniczeń współużytkowanych przez jedno lub więcej istniejących przeładowań.

\end{enumerate}

Jeśli ograniczenia nie są rozłączne z wieloma kandydatami, mogą być opłacalne. Kompilator musi określić najlepszego kandydata na wywołanie. Jeśli jednak kompilator nie może określić najlepszego kandydata, rozwiązanie jest niejednoznaczne. Gdy w pierwszym algorytmie \verb#czyIstnieje()# zmieni się wymaganie \verb#Sequence# zamiast tylko \verb#Range#. To zminimalizuje ilość nakładania się, a zatem i prawdopodobieństwo dwuznaczności.

Ograniczenia rozłączne nie gwarantują, że połączenie będzie niedwuznaczne. Możemy na przykład spróbować zdefiniować kontener, który spełnia wymagania zarówno \verb#Sequence# i \verb#Associative_container#. W tym przypadku oba przeciążenia byłyby opłacalne, ale przeciążenie nie jest z natury lepsze od innych. Chyba że dodamy nowe przeciążenia, aby dostosować się do tego rodzaju struktury danych, wynik byłby niejednoznacznym rozwiązaniem.

\verb#Sequence# i \verb#Associative_container# tak naprawdę mają pokrywające się ograniczenia. Oba wymagają konceptu \verb#Range#. Możemy rozważyć te przeciążenia jako przykład trzeciego przypadku. To wskazuje, że może istnieć algorytm, który można zdefiniować w odniesieniu do wymagań przecinających. Ale to nie jest takie proste.

Drugi przypadek jest ważną cechą programowania generycznego w języku \emph{C++} i jest podstawą optymalizacji typów w bibliotekach generycznych. Ograniczenie subsumpcji pozwala na optymalizację generycznych algorytmów opartych na interfejsach dostarczonych przez ich argumenty.

\end{document}

\documentclass[11pt, a4paper]{article}
\usepackage{polski}
\usepackage[utf8]{inputenc}
\usepackage{listings}

\linespread{1.3}

\begin{document}
\lstset{language=C++}

\subsection{Specjalizacja algorytmów}

W niektórych przypadkach możemy definiować struktury danych z rozszerzonym zestawem właściwości lub operacji, które mogą być wykorzystane do definiowania bardziej dopuszczalnych lub bardziej wydajnych wersji algorytmu. Ten pomysł jest realizowany przez hierarchię iteratorów biblioteki standardowej.

\emph{Iteratory forward} mogą być użyte do przechodzenia przez sekwencję w jednym kierunku (do przodu) poprzez przesuwanie się po jednym elemencie naraz, używając operatora \verb#++#.\newline

\noindent Prosty koncept iteratora forward:\newline

\begin{lstlisting}[frame=single]
template<typename I>
concept bool Forward_iterator() {
  return Regular<I>() && requires (I i) {
    typename value_type_t<I>;
    { *i } -> const value_type_t<I>&;
    { ++i } -> I&;
  };
}
\end{lstlisting}

Opierając się na tym koncepcie, możemy zdefiniować dwa użyteczne algorytmy. Jeden, który przechodzi przez iterator wielokrotnymi krokami używając pętli i drugi, który oblicza liczbę kroków między dwoma iteratorami.

\begin{lstlisting}[frame=single]
template<Forward_iterator I>
void advance(I& iter, int n) {
  //(1)
  while (n != 0) { ++iter; --n; }
}
template<Forward_iterator I>
int distance(I first, I limit) {
  (2)
  for (int n = 0; first != limit; ++first, ++n);
  return n;
}
\end{lstlisting}

(1) - warunek wstępny: \verb#n >= 0# (2) - warunek wstępny: \verb#limit# jest osiągalny z \verb#first#

Parametr \verb#n# funkcji \verb#advance# musi być nieujemny bo \emph{iteratory forward} nie mogą iść do tyłu. Ale \emph{iterator bidirectional} może być użyty do wędrowania po sekwencji w oba kierunki (do przodu i do tyłu) poprzez przechodzenie po elementach naraz używając operatorów \verb#++# lub \verb#--#.

\begin{lstlisting}[frame=single]
template<typename I>
  concept bool Bidirectional_iterator() {
    return Forward_iterator<I>() && requires (I i)
    {
      { --i } -> I&;
    };
  }
\end{lstlisting}

Koncept \verb#Bidirectional_iterator# jest zbudowany na podstawie \newline \verb#Forward_iterator#. Czyli \emph{iterator bidirectional} jest \emph{iteratorem forward}, który również może poruszać się do tyłu. Zestaw wymagań konceptu \verb#Bidirecti-# \newline \verb#onal_iterator# całkowicie zalicza się do zestawu konceptu \verb#Forward_iterator#. W wyniku czego, za każdym razem gdy \verb#Bidirectional_iterator<X># jest prawdziwe (dla wszystkich \verb#X#), \verb#Forward_iterator<X># musi tez być prawdziwy. W tym przypadku mówimy, że koncept \verb#Bidirectional_iterator# \emph{udoskonala}\footnote{(ang. refine)} koncept \verb#Forward_iterator#.

To \emph{udoskonalenie} pozwala nam zdefiniować nową wersję \verb#advance()#, która może poruszać się w oba kierunki.

\begin{lstlisting}[frame=single]
template<Bidirectional_iterator I>
  void advance(I& iter, int n) {
    if (n > 0)
      while (n != 0) { ++iter; --n; }
    else if (n < 0)
      while (n != 0) { --iter; ++n; }
  }
\end{lstlisting}

Koncept \verb#Bidirectional_iterator# pozwala nam uspokoić warunek wstępny funkcji \verb#advance()#, dzięki czemu możemy użyć ujemnych wartości \verb#n#. Z drugiej strony \verb#Bidirectional_iterator# nie zawiera żadnych nowych informacji, które mogłyby pomóc nam ulepszyć \verb#distance()#. Możemy jednak zapewnić optymalizację zarówno \verb#advance()# jak i \verb#distance()# dla \emph{iteratorów random access}. Te iteratory mogą być użyte do przebycia sekwencji w dwóch kierunkach, ale mogą posuwać się do wielu elementów w jednym kroku używając operatorów + = lub - =. Możemy również policzyć odległość między dwoma iteratorami, odejmując je.

\begin{lstlisting}[frame=single]
template<typename I>
  concept bool Random_access_iterator() {
    return Bidirectional_iterator<I>() && 
      requires (I i, int n) {
      { i += n } -> I&;
      { i -= n } -> I&;
      { i - i } -> int;
    };
  }
\end{lstlisting}

Koncept \verb#Random_access_iterator# udoskonala koncept \verb#Bidirectional-# \newline \verb#_iterator#. Dodaje trzy nowe wymagane operacje. Dzięki tym operacjom możemy konstruować zoptymalizowane wersje \verb#advance()# i \verb#distance()#, które nie wymagają pętli.

\begin{lstlisting}[frame=single]
template<Random_access_iterator I>
void advance(I& iter, int n) {
   iter += n;
}
template<Random_access_iterator I>
int distance(I first, I limit) {
   return limit - first;
}
\end{lstlisting}

Te algorytmy można używać do zdefiniowania dużej liczby użytecznych operacji.

\begin{lstlisting}[frame=single]
template<Forward_iterator I, Ordered T>
  requires Same_as<T, value_type_t<I>>()
bool binary_search(I first, I limit, T const& value) {
  if (first == limit)
    return false;
  auto mid = first;
  advance(mid, distance(first, limit) / 2);
  if (value < *mid)
    return search(first, mid, value);
  else if (*mid < value)
    return search(++mid, limit, value);
  else
    return true;
}

\end{lstlisting}

Algorytm jest definiowany dla iteratorów \emph{forward}, ale oczywiście może być używany również do \emph{bidirectional} i \emph{random access}. Wersje \verb#advance()# i \verb#distance()#, które są używane, zależą od typu iteratora przekazanego do algorytmu. W przypadku \emph{iteratorów forward} i \emph{bidirectional}, algorytm jest liniowy w zakresie wielkości wejściowych. W przypadku \emph{iteratorów random access} algorytm jest znacznie szybszy, ponieważ \verb#distance()# i \verb#advance()# nie wymagają dodatkowych przejazdów sekwencji wejściowej.

Zdolność do specjalizacji algorytmów według ograniczeń i typów ma decydujące znaczenie dla wydajności bibliotek generycznych języka \emph{C++}. Koncepty znacznie ułatwiają definiowanie i wykorzystywanie tych specjalizacji. Ale jak kompilator wie, które przeciążenie wybrać?

W poprzednich przykładach wykorzystujących sekwencje i kontenery asocjacyjne, tylko jedno przeciążenie funkcji \verb#czyIstnieje()# było zawsze opłacalne, ponieważ argumenty były jednego lub drugiego, ale nie obu. Jeśli jednak wywołamy \verb#binarny_search()# z \emph{iteratorami random access}, powiedzmy, że są to wskaźniki do tablicy, wszystkie trzy przeciążenia \verb#advance()# i oba przeciążenia \verb#distance()# będą opłacalne. To ma sens. Każda implementacja tych funkcji jest doskonale zdefiniowana dla wskaźników.

W takim przypadku kompilator musi wybrać najlepszego spośród potencjalnych kandydatów. Ogólnie rzecz ujmując, \emph{C++} uważa, że jedna z funkcji jest lepsza od innej za pomocą następujących reguł:

\begin{enumerate}
\item Funkcje wymagające mniejszych lub "tańszych" konwersji argumentów są lepsze niż te wymagające większych lub bardziej kosztownych konwersji.

\item Funkcje nieszablonowe są lepsze niż specjalizacje szablonów funkcji.

\item Jedna specjalizacja szablonu funkcji jest lepsza od innej, jej typy parametrów są bardziej wyspecjalizowane. Na przykład \verb#T*# jest bardziej wyspecjalizowany niż \verb#T#, i tak samo \verb#vector<T>#, ale \verb#T*# nie jest bardziej wyspecjalizowany niż \verb#vector<T>#, ani też nie jest przeciwnie.

\textbf{Specyfikacja techniczna konceptów dodaje jeszcze jedną zasadę:}

\item Jeśli dwie funkcje nie mogą być sortowane, ponieważ mają równoważne konwersje lub są specjalizacjami szablonów funkcji o równoważnych typach parametrów, tym lepsza jest bardziej ograniczona. Są to najmniej ograniczone funkcje nie ograniczone. 

\end{enumerate}

Innymi słowy, ograniczenia działają jako łącznik dla zwykłych reguł przeciążania w \emph{C++}. Kolejność ograniczeń (bardziej ograniczona) zależy zasadniczo od porównania zestawów wymagań dla każdego szablonu w celu określenia, czy jest to ścisły nadzbiór drugiego. W celu porównania ograniczeń, kompilator najpierw analizuje powiązane ograniczenia funkcji w celu zbudowania zestawu tak zwanych ograniczeń atomowych. Są \emph{atomowe}, ponieważ nie mogą być podzielone na mniejsze części. Ograniczenia atomowe zawierają wyrażenia stałe \emph{C++} (np. \emph{type traits}) i wymagania w wyrażeniu \verb#requires#.

Na przykład, w rozwiązaniu \verb#advance()#, gdy jest wywołany z \emph{iteratorem random access}, zestaw ograniczeń dla każdego przeciążenia to:

\noindent \begin{tabular}{|p{5cm}|p{7cm}|p{1cm}|} \hline
  \hline 
  Koncept & Atomowe wymagania \\
  \hline 
  \verb#Forward_iterator# & \verb#value_type_t<I># \verb#{ *i } -> value_type_t<I> const&# \verb#{ ++i } -> I&# \\
  \hline
  \verb#Bidirectional_iterator# & \verb#value_type_t<I># \verb#{ *i } -> value_type_t<I> const&# \verb#{ ++i } -> I&# \newline \verb#{ --i } -> I&# \\
  \hline
  \verb#Random_access_iterator# & \verb#value_type_t<I># \verb#{ *i } -> value_type_t<I> const&# \verb#{ ++i } -> I&# \newline \verb#{ --i } -> I&# \newline \verb#{ i += n } -> I&# \newline \verb#{ i -= n } -> I&# \newline \verb#{ i - j } -> int# \\
  \hline
  
\end{tabular} \newline

Dla zwięzłości wyłączyłem ograniczenie \verb#Regular<I># pojawiające się w \verb#Forward_iterator#, ponieważ on(i jego wymagania) są wspólne dla wszystkich konceptów iterujących. Porównując powyższe stwierdzimy, że \verb#Bidirec-# \newline \verb#tional_iterator# ma ścisły nadzbiór wymagań \verb#Forward_iterator#, a \verb#Rand-# \newline \verb#om_access_iterator# ma ścisły nadzbiór wymagań \verb#Bidirectional_iterator#. Z tego względu \verb#Random_access_iterator# jest najbardziej ograniczony i to przeciążenie zostało wybrane. Nowa reguła przeciążania nie gwarantuje, że rozwiązanie przeciążenia odniesie sukces. W szczególności, jeśli dwóch realnych kandydatów ma nakładające się lub logicznie równoważne ograniczenia, rozwiązanie będzie niejednoznaczna. Jest kilka powodów, dla których to miałoby się zdarzyć.

\end{document}

\documentclass[11pt, a4paper]{article}
\usepackage{polski}
\usepackage[utf8]{inputenc}
\usepackage{listings}

\linespread{1.3}

\begin{document}
\lstset{language=C++}

\subsection{Semantyczne udoskonalanie}

W niektórych przypadkach udoskonalenia są czysto semantyczne. Nie dostarczają operacji, które kompilator może wykorzystać do odróżnienia przeciążeń. W rzeczywistości ten problem pojawia się w standardowej hierarchii iteratorów: \emph{iteratory input} i \emph{iteratory forward} dzielą dokładnie te same zestawy operacji.

Pojęciowo \emph{iterator input} jest iteratorem reprezentującym pozycję w strumieniu wejściowym. Ponieważ jest zwiększany, poprzednie elementy są konsumowane. Oznacza to, że wcześniej dostępne elementy nie są już dostępne przez iterator lub dowolną jego kopię. W przeciwieństwie do tego, \emph{iterator forward} nie konsumuje elementów przy zwiększaniu. Wcześniej dostępne elementy mogą być uzyskane dzięki kopiom. Jest to zazwyczaj określane mianem właściwości \emph{multipass}. Jest to czysto semantyczna własność.

\begin{lstlisting}[frame=single]
template<typename I>
concept bool Input_iterator() {
  return Regular<I>() && requires (I i) {
    typename value_type_t<I>;
    { *i } -> value_type_t<I> const&;
    { ++i } -> I&;
  };
}

template<typename I>
concept bool Forward_iterator() {
  return Input_iterator<I>();
}
\end{lstlisting}

Wszystkie wymagania składniowe są zdefiniowane w koncepcie \verb#Input_i-# \newline \verb#terator#. Koncept \verb#Forward_iterator# zawiera tylko \verb#Input_iterators#. Innymi słowy, zestaw wymagań \verb#Forward_iterator# jest dokładnie taki sam, jak \verb#Input_iterator#. Jeśli próbujemy zdefiniować przeciążenia wymagające tych konceptów, wynik byłby zawsze dwuznaczny (ani lepszy od drugiego). Zróżnicowanie pomiędzy tymi konceptami jest tak naprawdę przydatny. Na przykład jeden z konstruktorów \verb#vector# ma bardziej wydajną implementację \emph{iteratorów forward} niż dla \emph{iteratorów input}.

\begin{lstlisting}[frame=single]
template<Object_type T, Allocator_of<T> A>
class vector {
  template<Input_iterator I>
    requires Same_as<T, value_type_t<I>>()
  vector(I first, I limit) {
    for ( ; first != limit; ++first)
      push_back(*first);
  }

  template<Forward_iterator I>
    requires Same_as<T, value_type_t<I>>()
  vector::vector(I first, I limit) {
    reserve(distance(first, limit)); 
      // 1 allocation
    insert(begin(), first, limit);
  }
  // ...
\end{lstlisting}

To nie zadziała, jeśli kompilator nie może odróżnić \verb#Forward_iterator# z \verb#Input_iterator#. 

Można to naprawić dodając nowe wymagania składniowe do \verb#Forward_iterator#, które odnoszą się do jego rangi w hierarchii iteratorów. To tradycyjnie zostało zrobione przy użyciu \emph{tag dispatch}. Łączenie \emph{etykiety klasy}\footnote{(ang. tag class) Pusta klasa w hierarchii dziedziczenia} z typem iteratora w celu wybrania odpowiedniego przeciążenia. Ten skojarzony typ to \verb#iterator_category#. Zmieniony \verb#Forward_iterator# może wyglądać tak:

\begin{lstlisting}[frame=single]
template<typename I>
  concept bool Forward_iterator() {
    return Input_iterator<I>() && requires {
      typename iterator_category_t<I>;
      requires Derived_from<I,
        forward_iterator_tag>();
    };
  }
\end{lstlisting}

Dzięki tej definicji wymagania \verb#Forward_iterator# zaliczają wymagania \verb#Input_iterator#, a kompilator może rozróżnić powyższe przeciążenia. Jako dodatkowa zaleta, używanie \emph{iteratorów random access} będzie jeszcze bardziej wydajne bo \verb#distance()# wymaga tylko jednej operacji całkowitej.

Jako inny przykład, \emph{C++17} dodaje nową kategorię iteratorów: \emph{iteratory contiguous}. \emph{Iterator contiguous} jest \emph{iteratorem random access}, którego obiekty odwoławcze są przydzielane w sąsiednich obszarach pamięci, których adresy rosną wraz z każdym przyrostem iteratora. Powoduje to otwarcie drzwi na wiele optymalizacji pamięci na niższym poziomie. Jest to oczywiście zupełnie czysta semantyka. Jeśli chcemy zdefiniować nowy koncept, musimy ją odróżnić od \verb#Random_access_iterator#. Na szczęście właśnie zdefiniowaliśmy maszynę, aby to zrobić.

\begin{lstlisting}[frame=single]
template<typename I>
concept bool Contiguous_iterator() {
  return Random_access_iterator<I>() && requires {
    requires Derived_from<I,
    contiguous_iterator_tag>();
  };
}
\end{lstlisting}

%Etykiety klasy nie są jedynym sposobem na rozwiązanie tego problemu. Używana jest istniejąca standardowa infrastruktura biblioteki. W rzeczywistości jedynymi konceptami wymagającymi tych klas znaczników są \verb#Forward_iterator# i \verb#Contiguous_iterator#. Nie potrzebujemy żadnej innej klasy tagów. Możemy po prostu użyć \emph{associated type trait}, zmiennej szablonu lub nawet dodatkowej operacji. Innymi słowy, możemy zrobić coś podobnego do poniższego kodu dla \emph{iteratorów forward}.

%begin{lstlisting}[frame=single]
%template<typename T>
%constexpr bool is_forward_iterator_v = false;

%template<typename T>
%constexpr bool is_forward_iterator_v<T*> = true;

%template<typename I>
%concept bool Forward_iterator() {
 % return Input_iterator<I>() &&
  %  is_forward_iterator_v<T*>;
%}
%end{lstlisting}

%Wszystkie te podejścia dałyby ten sam wynik; zdolność kompilatora do odróżniania przeładowań wymagających tych konceptów. Zgodnie z ogólną zasadą, ta technika powinna być używana tylko do różnicowania pojęć, które różnią się tylko w ich semantyce. Preferuj definiowanie pojęć tak, aby ich interfejsy odzwierciedlały ich różne semantyki.

\end{document}

\end{document}
	
	\newpage
	
	\documentclass[11pt, a4paper]{article}
\usepackage{polski}
\usepackage[utf8]{inputenc}
\usepackage{listings}
\usepackage{amsthm}
\usepackage[]{algorithm2e}
\renewcommand{\algorithmcfname}{Algorytm}

\begin{document}
\lstset{language=C++}

\section{Implementacja algorytmu Fleury'ego jako przykład wykorzystujący koncepty}

\subsection{Omówienie problemu}
\emph{Algorytm Fleury'ego} to algorytm pozwalający na znalezienie \emph{cyklu Eulera} w \emph{grafie eulerowskim.}

\newtheorem{mydef}{Definicja}

\begin{mydef}
\textbf{Graf} - struktura służąca do przedstawiania i badania relacji między obiektami. Jest to zbiór wierzchołków, które mogą być połączone krawędziami, gdzie krawędź zaczyna się i kończy w którymś z wierzchołków.
\end{mydef}

\begin{mydef}
\textbf{Multigraf} - graf, w którym mogą występować krawędzie wielokrotne(powtarzające się) oraz pętle (krawędzie, których końcami jest ten sam wierzchołek).
\end{mydef}

\begin{mydef}
\textbf{Graf spójny} - graf spełniający warunek, że dla każdej pary wierzchołków istnieje ścieżka, która je łączy.
\end{mydef}

\begin{mydef}
\textbf{Ścieżka} - ciąg wierzchołków, połączonych krawędziami.
\end{mydef}

\begin{mydef}
\textbf{Droga} - ścieżka, w której wierzchołki są różne.
\end{mydef}

\begin{mydef}
\textbf{Cykl} - droga zamknięta czyli ścieżka, w której pierwszy i ostatni wierzchołek są równe.
\end{mydef}

\begin{mydef}
\textbf{Graf eulerowski} - spójny multigraf posiadający cykl, który zawiera wszystkie krawędzie. 
\end{mydef}

\begin{mydef}
\textbf{Warunek istnienia cyklu Eulera w spójnym multigrafie} - stopień każdego wierzchołka musi być liczbą parzystą.
\end{mydef}

\begin{algorithm}[H]
 \KwData{G = (V,E), G - spójny multigraf, V - zbiór wierzchołków, E - zbiór krawędzi}\label{Wejście}
 \KwResult{zbiór wierzchołków reprezentujących cykl Eulera}
 Zaczynamy od dowolnego wierzchołka ze zbioru V\;
 \While{Dopóki zbiór krawędzi nie jest pusty}{
  \eIf{Jeżeli z bieżącego wierzchołka x odchodzi tylko jedna krawędź}{
   to przechodzimy wzdłuż tej krawędzi do następnego wierzchołka i usuwamy tę krawędź wraz z wierzchołkiem x\;
   }{
  wybieramy tę krawędź, której usunięcie nie rozspójnia grafu i przechodzimy wzdłuż tej krawędzi do następnego wierzchołka, a następnie usuwamy tę krawędź z grafu\;
  }
 }
 \caption{Algorytm Fleury'ego}
\end{algorithm}

\subsection{Działanie programu}
Założeniem programu jest symulacja algorytmu Fleury'ego dla jak największej ilości kontenerów biblioteki STL. Dzięki przeciążaniu funkcji jakie oferują koncepty, w prosty i czytelny sposób, udało się napisać generyczny algorytm.

W programie są dwa kontenery do przechowywania krawędzi i wierzchołków. Krawędź reprezentowana jest przez klasę \verb#Edge#, która przyjmuje dwie wartości typu \verb#int# do konstruktora. Wierzchołek, z kolei reprezentowany jest przez zmienną \verb#int#. 

Dane (pary wierzchołków) są wczytywane z pliku, a potem w zależności od rodzaju kontenera i iteratora, sortowane. Dla kontenerów: 
\begin{itemize}
\item sekwencyjnych z iteratorem \verb#Random access# (\verb#vector#) wywoływana jest funkcja szablonu ograniczonego przez koncepty: \verb#Sequence# i \newline \verb#Random_access_iterator#.

\begin{lstlisting}[frame=single]
template<Sequence S, Random_access_iterator R>
void sortVertices(S &seq){
    sort(seq.begin(), seq.end());
}
\end{lstlisting}

\item sekwencyjnych z iteratorem \verb#Bidirectional# (\verb#list#) wywoływana jest funkcja szablonu ograniczonego przez koncepty: \verb#Sequence# i \newline \verb#Bidirectional_iterator#.

\begin{lstlisting}[frame=single]
template<Sequence S, Bidirectional_iterator R>
void sortVertices(S &seq){
    seq.sort();
}
\end{lstlisting}

\item asocjacyjnych (\verb#set#) wywoływana jest funkcja szablonu ograniczonego przez koncept: \verb#Associative_container #.

\begin{lstlisting}[frame=single]
template<Associative_container A>
void sortVertices(A &seq){}
\end{lstlisting}

\end{itemize}
\newpage
Omawiany algorytm wykonuje funkcja \verb#determineEulerCycle#:
\begin{lstlisting}[frame=single]
template<typename E, typename V>
void determineEulerCycle(E &edges, V &vertices){
   int v = 0;
   bool condition = (checkIfGraphConnected(edges, 
   vertices, 0, v) && checkIfAllEdgesEvenDegree
   (edges, vertices));
    
   if(condition){
      cout << "Euler cycle:" << endl << endl << v;
      while(!edges.empty()){
         switch(getNeighboursCount(edges, v)){
            case 1 : {
               removeEdgeWithOneNeighbour(edges, v);
               break;
            }
            default: {
               removeEdgeWithMoreNeighbour(edges,
               v, vertices);
               break;
            }
         }
         cout <<" -> "<<v;
      }
      cout << endl << endl;
    } else {
       cout<<"Invalid graph."<<endl;
       if(!checkIfGraphConnected(edges, vertices,
       0, v))
          cout <<"Graph is not connected"<<endl;
       else if(!checkIfAllEdgesEvenDegree(edges,
       vertices))
          cout <<"Not all the edges are even"<<endl;
    }
}
\end{lstlisting}

Żeby algorytm się wykonał, muszą zostać spełnione dwa warunki: graf musi być spójny(za to odpowiedzialna jest funkcja \verb#checkIfGraphConnected#) i wszystkie krawędzie muszą być parzystego stopnia (\verb#checkIfAllEdgesEvenDegree#).

\verb#checkIfGraphConnected()#
\begin{lstlisting}[frame=single]
template<typename E, typename V>
bool checkIfGraphConnected(E &ed, V &vertices,
int x, int startVertice) {
    
   bool *visited = new bool[vertices.size()];
   for (int i = 0; i < vertices.size(); i++) 
      visited[i] = false;

   stack<int>stack;
   int vc = 0;
    
   stack.push(startVertice);
   visited[startVertice] = true;
    
   while (!stack.empty()) {
       int v = stack.top();
       stack.pop();
       vc++;
    
       for(typename E::iterator it = ed.begin();
          it != ed.end(); it++){
          if(it->getA() == v && !visited[it->getB()]){
             visited[it->getB()] = true;
             stack.push(it->getB());
          } else if(it->getB() == v &&
          !visited[it->getA()]){
             visited[it->getA()] = true;
             stack.push(it->getA());
          }
        }
    }

    delete [] visited;

    return (vc == vertices.size()-x);

}
\end{lstlisting}

Algorytm przechodzi przez graf, po kolei wrzucając odwiedzane wierzchołki na stos, zaznaczając je w tablicy odwiedzonych (\verb#visited#) i zaraz zdejmuje z tego stosu, zwiększając licznik \verb#vc#. Robi to dopóki stos nie jest pusty. Zwraca warunek porównujący licznik \verb#vc# z rozmiarem kontenera wierzchołków (wszystkie wierzchołki zostały odwiedzone, czyli istnieją ścieżki między wierzchołkami, graf jest spójny).

\verb#checkIfAllEdgesEvenDegree#:
\begin{lstlisting}[frame=single]
template<typename E, typename V>
bool checkIfAllEdgesEvenDegree(E &edges, V &vertices){
    int counter = 0, i = 0;
    for(auto v : vertices){
        for(auto e : edges){
            if(e.getA() == v || e.getB() == v)
               counter++;
        }
        if(counter % 2 == 0) i++;
    }
    return (i == vertices.size()) ? true : false;
}
\end{lstlisting}

Zmienna \verb#i#  zwiększa się jeśli ilość wystąpień wierzchołka jest liczbą parzystą. Zwraca wartość \verb#true# jeśli zmienna \verb#i# jest równa liczbie elementów kontenera zawierającego wierzchołki(dla każdego wierzchołka zmienna \verb#i# zwiększała się o 1).

Jeśli warunek nie zostanie spełniony, użytkownik zostaje poinformowany o tym, że graf jest niepoprawny. W odwrotnej sytuacji, w pętli (dopóki kontener krawędzi nie jest pusty), wykonywana jest jedna dwóch operacji. Gdy wierzchołek ma jednego sąsiada, wywołuje się funkcja \verb#removeEdgeWithOneN-#\newline \verb#eighbour()#, a gdy więcej wierzchołków, funkcja \verb#removeEdgeWithMoreNei-# \newline \verb#ghbour()#. Pierwsza z nich ma dwa przeciążenia konceptowe:
\begin{itemize}

\item Dla kontenera sekwencyjnego:
\begin{lstlisting}[frame=single]
template<Sequence S>
void removeEdgeWithOneNeighbour(S &edges, int &v){
    
    typename S::iterator it = find(edges.begin(), 
    edges.end(), v);
    
    if(it->getA() == v) v = it->getB();
    else v = it->getA();

    edges.erase(it);
}
\end{lstlisting}

\item Dla kontenera asocjacyjnego:
\begin{lstlisting}[frame=single]
template<Associative_container A>
void removeEdgeWithOneNeighbour(A &edges, int &v){
    
    typename A::iterator it2;
    for(typename A::iterator it = edges.begin();
    it != edges.end(); it++){
        if(it->getA() == v || it->getB() == v){
            it2 = it;
            if(it->getA() == v) v = it->getB();
            else v = it->getA();
            it = prev(edges.end());
        }
    }

    edges.erase(it2);
}
\end{lstlisting}

Funkcja znajduje krawędź, dostając wierzchołek wychodzący. I wierzchołek znalezionej krawędzi przypisuje do tego przekazanego.

\end{itemize}

Druga \verb#removeEdgeWithMoreNeighbour()# wygląda:
\begin{lstlisting}[frame=single]
template<typename E, typename V>
void removeEdgeWithMoreNeighbour(E &edges, int &v,
V &vertices){
    for(typename E::iterator i = edges.begin(); 
    i != edges.end(); i++){
        if(i->getA() == v && 
        checkIfStillConnected(edges, *i, 
        getZeroDegreeCount(edges, vertices), v, 
        vertices)){
            v = i->getB();
            edges.erase(i);
            i = prev(edges.end());
        } else if(i->getB() == v && 
        checkIfStillConnected(edges, *i, 
        getZeroDegreeCount(edges, vertices), v, 
        vertices)){
            v = i->getA();
            edges.erase(i);
            i = prev(edges.end());
        }
    }
}
\end{lstlisting}
\newpage
Jeśli wierzchołek ma więcej sąsiadów, wybiera tego który nie rozspójni grafu. Żeby to sprawdzić używa funkcji \verb#checkIfStillConnected#:
\begin{lstlisting}[frame=single]
template<Associative_container E, typename V>
bool checkIfStillConnected(E &edges, Edge e, int x, 
int startVertice, V &vertices){
    
   E tmp;

   for(auto e : edges)
      tmp.insert(e);

   for(typename E::iterator it = tmp.begin(); 
   it != tmp.end(); it++)
      if (it->getA() == e.getA() && 
         it->getB() == e.getB()) {
         tmp.erase(it);
         it = prev(tmp.end());
      }
        
   return checkIfGraphConnected(tmp, vertices, 
   x, startVertice);
}
\end{lstlisting}

W celu sprawdzenia, czy graf po usunięciu jakiejś krawędzi dalej będzie spójny, potrzebny jest pomocniczy kontener. Zapisujemy do niego aktualne krawędzie, wyszukujemy w nim przekazaną i przekazujemy go do istniejącej już funkcji \verb#checkIfGraphConnected()#.

\end{document}
	
	\newpage
	
	%\newpage

	%\section{Rozdział 5}
	%Zawartość rozdziału 5
	
	%\newpage
	
	\input{5.Ostatni/Ostatni}
	
	\newpage
	
	\section{Bibliografia}
	\begin{thebibliography}{6}
	
	\bibitem{first} Gabriel Dos Reis, \emph{Generic Programming in C++: The Next Level.}, ACCU, 2002.
	\bibitem{second} Bjarne Stroustrup, \emph{The Design and Evolution of C++}, AddisonWesley, 1994
	\bibitem{third}  Bjarne Stroustrup, \emph{Expressing the standard library requirements asconcepts}	
	\bibitem{fourth} Gabriel Dos Reis, Bjarne Stroustrup, \href{http://www.stroustrup.com/popl06.pdf}{\emph{Specifying C++ Concepts}}
	\bibitem{fifth} J. C. Dehnert, A. Stepanov, \emph{Fundamentals of Generic Programming}, Dagstuhl Seminar on Generic Programming.1998. Springer LNCS.

	\bibitem{sixth} A. Stepanov, Daniel E.Rose, \emph{From Mathematics to Generic Programming}
	\bibitem{seventh} D. Gregor, J. Jarvi, J. Siek, B. Stroustrup, G. Dos Reis, A. Lumsdaine, \emph{Concepts: Linguistic Support for Generic Programming in C++}, OOPSLA’06.
	\bibitem {eighth} Scott Meyers, \emph{Effective Modern C++}, O'REILLY 2015
	\end{thebibliography}

\end{document}