\documentclass[11pt, a4paper]{article}
\usepackage{polski}
\usepackage[utf8]{inputenc}

\author{Maciej Zbierowski}
\title{Koncepty jako sposób ograniczania argumentów szablonu}

\linespread{1.3}
\usepackage{hyperref}
\usepackage{listings}
\usepackage{amsthm}
\usepackage[]{algorithm2e}
\usepackage{standalone}

\hypersetup{
	colorlinks=true,
	linkcolor=black,
	filecolor=magenta,
	urlcolor=cyan,
	pdftitle=Praca Magisterska,
	bookmarks=true,
	pdfpagemode=FullScreen
}

\renewcommand\refname{\vspace*{-4ex}}

\makeatletter
\renewcommand\@seccntformat[1]{\csname the#1\endcsname.\quad}
\renewcommand\numberline[1]{#1.\hskip0.7em}
\makeatother

\newenvironment{university}[1]{
	\begin{center}
		\Large{\textsc{#1}}\\[1ex] 
	\end{center}
}

\newenvironment{studentname}[2]{
	\begin{center}	
		\vspace{#1}\textbf{#2}
	\end{center}
}

\newenvironment{workname}[2]{
	\begin{center}	
		\vspace{#1}\textbf{\Large{#2}}
	\end{center}
}

\newenvironment{fac}[2]{
	\begin{center}	
		\emph{#1} #2 
	\end{center}
}

\newenvironment{pro}[3]{
	\begin{flushright}	
		\vspace{#1}\hspace{#2} \Large{#3} 
		\end{flushright}
}

\newenvironment{bottompar}{
	\par\vspace*{\fill}}{\clearpage
}

\begin{document}
	\pagenumbering{gobble}
	%\maketitle

	\begin{university}{Uniwersytet Gdański}\end{university}
\begin{university}{Wydział Matematyki, Fizyki i Informatyki}\end{university}
\begin{studentname}{80pt}{Maciej Zbierowski}\end{studentname}
\begin{fac}{Kierunek: }{Informatyka}\end{fac}
\begin{fac}{Numer albumu: }{206310}\end{fac}
\begin{workname}{40pt}{Koncepty jako sposób ograniczania argumentów szablonu}\end{workname}
\begin{pro}{40pt}{100pt}{Praca magisterska napisana pod kierunkiem profesora Christopha Schwarzwellera}\end{pro}
\begin{bottompar}
	\begin{center}\textbf{Gdańsk 2017} \end{center}
\end{bottompar}	
	
	\newpage
	\tableofcontents
	\pagenumbering{arabic}
	\newpage

	\documentclass[11pt, a4paper]{article}
\usepackage{polski}
\usepackage[utf8]{inputenc}
\usepackage{listings}

\begin{document}
\lstset{language=C++}

\section{Szablony - definicja, zastosowania}
Szablony są jedną z głównych funkcjonalności języka \emph{C++}. Dzięki nim możemy dostarczać generyczne typy i funkcje, bez kosztów czasu wykonania. Skupiają się na pisaniu kodu w sposób niezależny od konkretnego typu, dzięki czemu wspierają programowanie generyczne. \emph{C++} to bogaty język wspierający polimorficzne zachowania zarówno w czasie wykonania jak i kompilacji. W czasie wykonania używa on hierarchii klas i wywołań funkcji wirtualnych by wspierać praktyki zorientowane obiektowo, gdzie wywoływana funkcja zależy od typu obiektu docelowego podczas czasu wykonania. Natomiast w czasie kompilacji szablony wspierają programowanie generyczne, gdzie wywoływana funkcja zależy od statycznego typu czasu kompilacji argumentów szablonu.

Polimorfizm czasu kompilacji był w języku od bardzo dawna. Polega na dostarczeniu szablonu, który umożliwia kompilatorowi wygenerowanie kodu w czasie kompilacji.

Szablony grają kluczową rolę w projektowaniu obecnych, znanych i popularnych bibliotek i systemów. Stanowią podstawę technik programowania w różnych dziedzinach, począwszy od konwencjonalnego programowania ogólnego przeznaczenia do oprogramowywania wbudowanych systemów bezpieczeństwa.

Szablon to coś w rodzaju przepisu, z którego translator \emph{C++} generuje deklaracje.

\begin{lstlisting}[frame=single]
template<typename T>
T kwadrat (T x) {
   return x * x;
}
\end{lstlisting}

Kod ten deklaruje rodzinę funkcji indeksowanych po parametrze typu. Można odnieść się do konkretnego członka tej rodziny przez zastosowanie konstrukcji \verb#kwadrat<int>#. Mówimy wtedy, że żądana jest specjalizacja szablonu dla funkcji \verb#kwadrat# z listą argumentów szablonu \verb#<int>#. Proces tworzenia specjalizacji nosi nazwę \emph{inicjalizacji szablonu}, potocznie zwaną \emph{inicjalizacją}. Kompilator \emph{C++} stworzy stosowny odpowiednik definicji funkcji:

\begin{lstlisting}[frame=single]
int kwadrat(int x) {
   return x * x;
}
\end{lstlisting}

Argument typu \verb#int# jest podstawiony za parametr typu \verb#T#. Kod wynikowy jest sprawdzany pod względem typu, by zapewnić brak błędów wynikających z podmiany. Inicjalizacja szablonu jest wykonywana tylko raz dla danej specyfikacji nawet jeśli program zawiera jej wielokrotne żądania. 

W przeciwieństwie do języków takich jak \emph{Ada} czy \emph{System F}, lista argumentów szablonu może być pominięta z żądania inicjalizacji szablonu funkcji. Zazwyczaj, wartości parametrów szablonu są \emph{dedukowane}\footnote{(ang. deduction) Dedukcja - określenie lub wyliczenie (przez kompilator) argumentu szablonu pominiętego przy wywołaniu funkcji.}.
\newline

\verb#double d = kwadrat(2.0);# \newline

Argument typu jest dedukowany na \verb#double#. Warto zauważyć, że inaczej niż w językach takich jak \emph{Haskell} czy \emph{System F}, parametry szablonu w \emph{C++} nie są ograniczone względem typów.

Szablonów używa się do zmniejszania kar abstrakcji i zjawiska \emph{code bloat}\footnote{Code bloat - produkowanie kodu, który postrzegany jest jako niepotrzebnie długi, spowalniający lub w inny sposób marnujący zasoby} w systemach wbudowanych w stopniu, który jest niepraktyczny w standardowych systemach obiektowych. Robi się to z dwóch powodów:

\begin{itemize}

\item Po pierwsze, inicjalizacja szablonu łączy informacje zarówno z definicji, jak i z kontekstu użycia. To oznacza, że pełna informacja zarówno z definicji jak i z wywołanych kontekstów (włączając w to informacje o typach) jest udostępniana generatorowi kodu. Dzisiejsze generatory kodu dobrze sobie radzą z używaniem tych informacji w celu zminimalizowania czasu wykonania i przestrzeni kodu. Różni się to od zwykłego przypadku w języku obiektowym, gdzie wywołujący i wywoływany są kompletnie oddzieleni przez interfejs, który zakłada pośrednie wywołania funkcji.

\item Po drugie, szablon w \emph{C++} jest zazwyczaj domyślnie tworzony tylko jeśli jest używany w sposób niezbędny dla semantyki programu, automatycznie minimalizując miejsce w pamięci, które wykorzystuje aplikacja. W przeciwieństwie do języka \emph{Ada} czy \emph{System F}, gdzie programista musi wyraźnie zarządzać inicjalizacjami.

\end{itemize}

\end{document}
	
	\documentclass[11pt, a4paper]{article}
\usepackage{polski}
\usepackage[utf8]{inputenc}
\usepackage{listings}
\usepackage{standalone}

\begin{document}
\lstset{language=C++}

\documentclass[11pt, a4paper]{article}
\usepackage{polski}
\usepackage[utf8]{inputenc}
\usepackage{listings}

\begin{document}
\lstset{language=C++}

\section{Szablony - definicja, zastosowania}
Szablony są jedną z głównych funkcjonalności języka \emph{C++}. Dzięki nim możemy dostarczać generyczne typy i funkcje, bez kosztów czasu wykonania. Skupiają się na pisaniu kodu w sposób niezależny od konkretnego typu, dzięki czemu wspierają programowanie generyczne. \emph{C++} to bogaty język wspierający polimorficzne zachowania zarówno w czasie wykonania jak i kompilacji. W czasie wykonania używa on hierarchii klas i wywołań funkcji wirtualnych by wspierać praktyki zorientowane obiektowo, gdzie wywoływana funkcja zależy od typu obiektu docelowego podczas czasu wykonania. Natomiast w czasie kompilacji szablony wspierają programowanie generyczne, gdzie wywoływana funkcja zależy od statycznego typu czasu kompilacji argumentów szablonu.

Polimorfizm czasu kompilacji był w języku od bardzo dawna. Polega na dostarczeniu szablonu, który umożliwia kompilatorowi wygenerowanie kodu w czasie kompilacji.

Szablony grają kluczową rolę w projektowaniu obecnych, znanych i popularnych bibliotek i systemów. Stanowią podstawę technik programowania w różnych dziedzinach, począwszy od konwencjonalnego programowania ogólnego przeznaczenia do oprogramowywania wbudowanych systemów bezpieczeństwa.

Szablon to coś w rodzaju przepisu, z którego translator \emph{C++} generuje deklaracje.

\begin{lstlisting}[frame=single]
template<typename T>
T kwadrat (T x) {
   return x * x;
}
\end{lstlisting}

Kod ten deklaruje rodzinę funkcji indeksowanych po parametrze typu. Można odnieść się do konkretnego członka tej rodziny przez zastosowanie konstrukcji \verb#kwadrat<int>#. Mówimy wtedy, że żądana jest specjalizacja szablonu dla funkcji \verb#kwadrat# z listą argumentów szablonu \verb#<int>#. Proces tworzenia specjalizacji nosi nazwę \emph{inicjalizacji szablonu}, potocznie zwaną \emph{inicjalizacją}. Kompilator \emph{C++} stworzy stosowny odpowiednik definicji funkcji:

\begin{lstlisting}[frame=single]
int kwadrat(int x) {
   return x * x;
}
\end{lstlisting}

Argument typu \verb#int# jest podstawiony za parametr typu \verb#T#. Kod wynikowy jest sprawdzany pod względem typu, by zapewnić brak błędów wynikających z podmiany. Inicjalizacja szablonu jest wykonywana tylko raz dla danej specyfikacji nawet jeśli program zawiera jej wielokrotne żądania. 

W przeciwieństwie do języków takich jak \emph{Ada} czy \emph{System F}, lista argumentów szablonu może być pominięta z żądania inicjalizacji szablonu funkcji. Zazwyczaj, wartości parametrów szablonu są \emph{dedukowane}\footnote{(ang. deduction) Dedukcja - określenie lub wyliczenie (przez kompilator) argumentu szablonu pominiętego przy wywołaniu funkcji.}.
\newline

\verb#double d = kwadrat(2.0);# \newline

Argument typu jest dedukowany na \verb#double#. Warto zauważyć, że inaczej niż w językach takich jak \emph{Haskell} czy \emph{System F}, parametry szablonu w \emph{C++} nie są ograniczone względem typów.

Szablonów używa się do zmniejszania kar abstrakcji i zjawiska \emph{code bloat}\footnote{Code bloat - produkowanie kodu, który postrzegany jest jako niepotrzebnie długi, spowalniający lub w inny sposób marnujący zasoby} w systemach wbudowanych w stopniu, który jest niepraktyczny w standardowych systemach obiektowych. Robi się to z dwóch powodów:

\begin{itemize}

\item Po pierwsze, inicjalizacja szablonu łączy informacje zarówno z definicji, jak i z kontekstu użycia. To oznacza, że pełna informacja zarówno z definicji jak i z wywołanych kontekstów (włączając w to informacje o typach) jest udostępniana generatorowi kodu. Dzisiejsze generatory kodu dobrze sobie radzą z używaniem tych informacji w celu zminimalizowania czasu wykonania i przestrzeni kodu. Różni się to od zwykłego przypadku w języku obiektowym, gdzie wywołujący i wywoływany są kompletnie oddzieleni przez interfejs, który zakłada pośrednie wywołania funkcji.

\item Po drugie, szablon w \emph{C++} jest zazwyczaj domyślnie tworzony tylko jeśli jest używany w sposób niezbędny dla semantyki programu, automatycznie minimalizując miejsce w pamięci, które wykorzystuje aplikacja. W przeciwieństwie do języka \emph{Ada} czy \emph{System F}, gdzie programista musi wyraźnie zarządzać inicjalizacjami.

\end{itemize}

\end{document}

\documentclass[11pt, a4paper]{article}
\usepackage{polski}
\usepackage[utf8]{inputenc}
\usepackage{listings}

\begin{document}
\lstset{language=C++}

\subsection{Parametryzacja szablonów}

Parametry szablonu są określane na dwa sposoby:

\begin{enumerate}

\item \emph{parametry szablonu} – wyraźnie wspomniane jako parametry w deklaracji szablonu

\item \emph{nazwy zależne} - wywnioskowane z użycia parametrów w definicji szablonu

\end{enumerate}

W \emph{C++} nazwa nie może być użyta bez wcześniejszej deklaracji. To wymaga od użytkownika ostrożnego traktowania definicji szablonów, np. w definicji funkcji \verb#kwadrat# nie ma widocznej deklaracji symbolu \verb#*#. Jednak, podczas inicjalizacji szablonu \verb#kwadrat<int># kompilator może sprowadzić symbol \verb#*# do (wbudowanego) operatora mnożenia dla wartości \verb#int#. Dla wywołania \verb#kwadrat(zespolona(2.0))#, operator * zostałby rozwiązany do (zdefiniowanego przez użytkownika) operatora mnożenia dla wartości \verb#zespolona#. Symbol \verb#*# jest więc \emph{nazwą zależną} w definicji funkcji \verb#kwadrat#. Oznacza to, że jest to ukryty parametr definicji szablonu. Możemy uczynić z operacji mnożenia formalny parametr:

\begin{lstlisting}[frame=single]
template<typename Multiply, typename T>
T square(T x) {
   return Multiply() (x,x);
}
\end{lstlisting}

Pod-wyrażenie \verb#Multiply()# tworzy obiekt funkcji, który wprowadza operację mnożenia wartości typu \verb#T#. Pojęcie \emph{nazw zależnych} pomaga utrzymać liczbę jawnych argumentów.

\end{document}

\documentclass[11pt, a4paper]{article}
\usepackage{polski}
\usepackage[utf8]{inputenc}
\usepackage{listings}

\begin{document}
\lstset{language=C++}

\subsection{Inicjalizacje i sprawdzanie}

Minimalne przetwarzanie semantyczne odbywa się, gdy po raz pierwszy pojawia się definicję szablonu lub jego użycie. Pełne przetwarzanie semantyczne jest przesuwane na czas inicjalizacji (tuż przed czasem linkowania), na podstawie każdej instancji. Oznacza to, że założenia dotyczące argumentów szablonu nie są sprawdzane przed czasem inicjalizacji. Np.\newline

\noindent \verb#string x = "testowy tekst";# \newline
\verb#kwadrat(x);# \newline

Bezsensowne użycie zmiennej \verb#string# jako argumentu funkcji \verb#kwadrat# nie jest wyłapane w momencie użycia. Dopiero w czasie inicjalizacji kompilator odkryje, że nie ma odpowiedniej deklaracji dla operatora *. To ogromny praktyczny błąd, bo inicjalizacja może być przeprowadzona przez kod napisany przez użytkownika, który nie napisał definicji funkcji \verb#kwadrat# ani definicji \verb#string#. Programista, który nie znał definicji funkcji \verb#kwadrat# ani \verb#string# miałby ogromne trudności w zrozumieniu komunikatów błędów związanych z ich interakcją (np. ”illegal operand for *”).

Istnienie symbolu operatora * nie jest wystarczające by zapewnić pomyślną kompilację funkcji \verb#kwadrat#. Musi istnieć operator *, który przyjmuje argumenty odpowiednich typów i ten operator * musi być bezkonkurencyjnym dopasowaniem według zasad przeciążania \emph{C++}. Dodatkowo funkcja \verb#kwadrat# przyjmuje argumenty przez wartość i zwraca swój wynik przez wartość. Z tego wynika, że musi być możliwe skopiowanie obiektów dedukowanego typu. Potrzebny jest rygorystyczny framework do opisywania wymagań definicji szablonów na ich argumentach.

Doświadczenie podpowiada nam, że pomyślna kompilacja i linkowanie może nie gwarantować końca problemów. Udana budowa pokazuje tylko, że inicjalizacje szablonów były poprawne pod względem typów, dostając argumenty które przekazaliśmy. Co z typami argumentów szablonu i wartościami, z którymi nie próbowaliśmy użyć naszych szablonów? Definicja szablonu może zawierać przypuszczenia na temat argumentów, które przekazaliśmy ale nie zadziała dla innych, prawdopodobnie rozsądnych argumentów. Uproszczona wersja klasycznego przykładu:

\begin{lstlisting}[frame=single]
template<typename FwdIter>
bool czyJestPalindromem(FwdIter first, FwdIter last){
   if(last <= first) return true;
   if(*first != *last) return false;
   return czyJestPalindromem(++first, --last);
}

\end{lstlisting}

Testujemy czy sekwencja wyznaczona przez parę iteratorów do jego pierwszego i ostatniego elementu, jest palindromem. Przyjmuje się, że te iteratory są z kategorii \emph{forward iterator}. To znaczy, że powinny wspierać co najmniej operacje takie jak: *, != i ++. Definicja funkcji \verb#czyJestPalindromem# bada czy elementy sekwencji zmierzają z początku i końca do środka. Możemy przetestować tę funkcję używając \verb#vector#, tablicy w stylu \verb#C# i \verb#string#. W każdym przypadku nasz szablon funkcji zainicjalizuje się i wykona się poprawnie. Niestety, umieszczenie tej funkcji w bibliotece byłoby dużym błędem. Nie wszystkie sekwencje wspierają operatory \verb#--# i $\leq$. Np. listy pojedyncze nie wspierają. Eksperci używają wyszukanych, regularnych technik by uniknąć takich problemów.  Jednakże, fundamentalny problem jest taki, że definicja szablonu nie jest (według siebie) dobrą specyfikacją wymagań na swoje parametry.

\end{document}

\documentclass[11pt, a4paper]{article}
\usepackage{polski}
\usepackage[utf8]{inputenc}
\usepackage{listings}

\begin{document}
\lstset{language=C++}

\subsection{Wydajność}

Szablony grają kluczową rolę w programowaniu w \emph{C++} dla wydajnych aplikacji. Ta wydajność ma trzy źródła:

\begin{itemize}

\item eliminacja wywołań funkcji na korzyść \emph{inliningu}\footnote{Optymalizacja kompilatora, która zamienia wywołanie funkcji na jej ciało w czasie kompilacji.}
\item łączenie informacji z różnych kontekstów w celu lepszej optymalizacji
\item unikanie generowania kodu dla niewykorzystanych funkcji

\end{itemize}

Pierwszy punkt nie odnosi się tylko do szablonów ale ogólnie do cech funkcji \emph{inline} w \emph{C++}. 
Wydajność ta przekłada się zarówno na czas wykonania jak i pamięć. Szablony mogą równocześnie zmniejszyć obie wydajności. Zmniejszenie rozmiaru kodu jest szczególnie ważne, ponieważ w przypadku nowoczesnych procesorów zmniejszenie rozmiaru kodu pociąga za sobą zmniejszenie ruchu w pamięci i poprawienie wydajności pamięci podręcznej.

%Jakkolwiek, \emph{inlining} jest istotny dla drobno-granularnej parametryzacji, którą powszechnie stosuje się w bibliotece \emph{STL} i innych bibliotekach  bazujących na generycznych technikach programowania. 

\begin{lstlisting}[frame=single]
template<typename FwdIter, typename T>
T suma(FwdIter first, FwdIter last, T init){
   for(FwdIter cur = first, cur != last, T init)
      init = init + *cur;
   return init;
}

\end{lstlisting}

Funkcja \verb#suma# zwraca sumę elementów jej sekwencji wejściowej używając trzeciego argumentu ("akumulatora") jako wartości początkowej\newline

\noindent \verb#vector<zespolona<double>> v;#  \newline
\verb#zespolona<double> z = 0;# \newline
\verb#z = suma(v.begin(), v.end(), z);# \newline

By wykonać swoją pracę, \verb#suma# użyje operatorów dodawania i przypisania na elementach typu \verb#zespolona<double># i dereferencji iteratorów \verb#vector<zespolona<double>>#.  Dodanie wartości typu \verb#zespolona<double># pociąga za sobą dodanie wartości typu \verb#double#. By zrobić to wydajnie wszystkie te operacje muszą być \emph{inline}.  Zarówno \verb#vector# jak i \verb#zespolona# są typami zdefiniowanymi przez użytkownika. Oznacza to, że typy te jak i ich operacje są zdefiniowane gdzie indziej w kodzie źródłowym \emph{C++}. Obecne kompilatory \emph{C++} radzą sobie z tym przykładem, dzięki czemu jedyne wygenerowane wywołanie to wywołanie funkcji \verb#suma#. Dostęp do pól zmiennej \verb#vector# staje się prostą operacją maszyny ładującej, dodawanie wartości typu \verb#zespolona# staje się dwiema instrukcjami maszyny dodającej dwa elementy zmiennoprzecinkowe. Aby to osiągnąć, kompilator potrzebuje dostępu do pełnej definicji \verb#vector# i \verb#zespolona#. Jednak wynik jest ogromną poprawą (prawdopodobnie optymalną) w stosunku do naiwnego podejścia generowania wywołania funkcji dla każdego użycia operacji na parametrze szablonu. Oczywiście instrukcja dodawania wykonuje się znacznie szybciej niż wywołanie funkcji zawierającej dodawanie. Poza tym, nie ma żadnego wstępu wywołania funkcji, przekazywania argumentów itd., więc kod wynikowy jest również wiele mniejszy. Dalsze zmniejszanie rozmiaru generowanego kodu uzyskuje się nie wysyłając kodu niewykorzystywanych funkcji. Klasa szablonu \verb#vector# ma wiele funkcji, które nie są wykorzystywane w tym przykładzie. Podobnie szablon klasy \verb#zespolona# ma wiele funkcji i funkcji nieskładowych (nienależących do funkcji klasy). Standard \emph{C++} gwarantuje, że nie jest emitowany żaden kod dla tych niewykorzystanych funkcji. 

Inaczej sprawa wygląda, gdy argumenty są dostępne za pośrednictwem interfejsów zdefiniowanych jako wywołania funkcji pośrednich. Każda operacja staje się wtedy wywołaniem funkcji w pliku wykonywalnym generowanym dla kodu użytkownika, takiego jak \verb#suma#. Co więcej, byłoby wyraźnie nietypowe unikać odkładania kodu nieużywanych (wirtualnych) funkcji składowych. Jest to poza zdolnością obecnych kompilatorów \emph{C++} i prawdopodobnie pozostanie takie dla głównych programów \emph{C++}, gdzie oddzielna kompilacja i łączenie dynamiczne jest normą. Ten problem nie jest wyjątkowy dla \emph{C++}. Opiera się on na podstawowej trudności w ocenieniu, która część kodu źródłowego jest używana, a która nie, gdy jakakolwiek forma procesu \emph{run-time dispatch}\footnote{Zwany również \emph{dynamic dispatch} proces wybierania, implementacji polimorficznej operacji (metody lub funkcji) do wywołania w czasie uruchomienia.} ma miejsce. Szablony nie cierpią na ten problem bo ich specjalizacje są rozwiązywane w czasie kompilacji.\newline

\noindent \verb#vector<int> v;#  \newline
\verb#zespolona<double> s = 0;#  \newline
\verb#s = suma(v.begin(), v.end(), s);#  \newline

W powyższej funkcji dodawanie wykonywane jest przez konwertowanie wartości \verb#int# do wartości \verb#double# i potem dodawanie tego do akumulatora \verb#s#, używając operatora \verb#+# typu \verb#zespolona<double># i \verb#double#. To podstawowe dodawanie zmiennoprzecinkowe. Kwestia jest taka, że operator \verb#+# w funkcji \verb#suma# zależy od dwóch parametrów szablonu i leży to w kwestii kompilatora by wybrać bardziej odpowiedni operator \verb#+# bazując na informacji o tych dwóch argumentach. Byłoby możliwe utrzymanie lepszego rozdzielenia między różnymi kontekstami przez przekształcanie typu elementu w typ akumulatora. W takim przypadku spowodowałoby to powstanie dodatkowego \verb#zespolona<double># dla każdego elementu i dodania dwóch wartości typu \verb#zespolona#. Rozmiar kodu i czas wykonywania byłyby większe niż dwukrotnie.

Duże ilości prawdziwego oprogramowania zależą od optymalizacji. W konsekwencji udoskonalone sprawdzanie typu, co zostało obiecane przy użyciu konceptów, nie może kosztować tych optymalizacji.

\end{document}

\end{document}
	
	\newpage
	
	\documentclass[11pt, a4paper]{article}
\usepackage{polski}
\usepackage[utf8]{inputenc}
\usepackage{listings}
\usepackage{standalone}

\begin{document}
\lstset{language=C++}

\section{Koncepty}

\documentclass[11pt, a4paper]{article}
\usepackage{polski}
\usepackage[utf8]{inputenc}
\usepackage{listings}

\begin{document}
\lstset{language=C++}

%\subsection*{Wprowadzenie}
W 1987 próbowano projektować szablony z odpowiednimi interfejsami. Chciano by szablony:

\begin{itemize}

\item były w pełni ogólne i wyraziste
\item by nie wykorzystywały większych zasobów w porównaniu do kodowania ręcznego
\item by miały dobrze określone interfejsy

\end{itemize}

\noindent Długo nie dało się osiągnąć tych trzech rzeczy, ale za to osiągnięto:

\begin{itemize}

\item \emph{kompletność Turinga}\footnote{(ang. Turing Completness) umiejętność do rozwiązania każdego zadania, czyli udzielenie odpowiedzi na każde zadanie. Program, który jest kompletny według Turinga może być wykorzystany do symulacji jakiejkolwiek 1-taśmowej maszyny Turinga}
\item lepszą wydajność (w porównaniu do kodu pisanego ręcznie)
\item kiepskie interfejsy (praktycznie \emph{typowanie kaczkowe czasu kompilacji})\footnote{(ang. duck typing) rozpoznanie typu obiektu, nie na podstawie deklaracji, ale przez badanie metod udostępnionych przez obiekt}

\end{itemize}

Brak dobrze określonych interfejsów prowadzi do szczególnie złych wiadomości błędów. Dwie pozostałe właściwości uczyniły z szablonów sukces.

Rozwiązanie problemu specyfikacji interfejsu zostało, przez Alexa Stepanova nazwane \textbf{konceptami}. \textbf{Koncept} to zbiór wymagań argumentów szablonu. Można też go nazwać systemem typów dla szablonów, który obiecuje znacząco ulepszyć diagnostyki błędów i zwiększyć siłę ekspresji, taką jak przeciążanie konceptowe oraz częściową specjalizację szablonu funkcji.

Koncepty (\emph{The Concepts TS} zostały opublikowane i zaimplementowane w wersji 6.1 kompilatora GCC w kwietniu 2016 roku. Fundamentalnie to predykaty czasu kompilacji typów i wartości. Mogą być łączone zwykłymi operatorami logicznymi (\verb#&&#, \verb#||#, \verb#!#)

%\addcontentsline{toc}{subsection}{Wprowadzenie}

\end{document}

\documentclass[11pt, a4paper]{article}
\usepackage{polski}
\usepackage[utf8]{inputenc}
\usepackage{listings}

\begin{document}
\lstset{language=C++}

\subsection{Przeciążanie funkcji przy użyciu konceptów}

Głowna idea programowania generycznego polega na używaniu tej samej nazwy dla równoważnych operacji używających różnych typów. A zatem, w grę wchodzi przeciążanie. Jest bardzo często przeoczaną, źle rozumianą ale niezwykle potężną cechą konceptów. Koncepty pozwalają na wybieranie spośród funkcji opierając się na właściwościach danych argumentów. Są przydatne nie tylko do poprawiania komunikatów o błędach i dokładnej specyfikacji interfejsów. Zwiększają również ekspresywność. Mogą być użyte do skracania kodu, robienia go bardziej ogólnym i zwiększania wydajności.

C++ jest językiem nie tylko assemblerowym wykorzystywanym do metaprogramowania szablonów. Koncepty pozwalają na podnoszenie poziomu programowania i upraszczają kod, bez angażowania dodatkowych zasobów czasu wykonania.

Przykład algorytmu \emph{advance}\footnote{Algorytm \emph{advance(it, n);} inkrementuje otrzymany iterator \emph{it} o \emph{n} elementów.} ze standardowej biblioteki

\begin{lstlisting}[frame=single]
template<typename Iter> void advance(Iter p, int n);
\end{lstlisting}

Potrzeba różnych wersji tego algorytmu, m.in.
\begin{itemize}
\item prostej, dla iteratorów \emph{Forward}, przechodzących przez sekwencję element po elemencie
\item szybkiej, dla iteratorów \emph{RandomAccess}, by wykorzystać umiejętność do zwiększania iteratora do arbitralnej pozycji w sekwencji używając jednej operacji.
\end{itemize}

Taka selekcja czasu kompilacji jest istotna dla wykonania kodu generycznego. Tradycyjnie, da się to zaimplementować używając funkcji pomocniczych lub techniki \emph{Tag Dispatching}\footnote{Technika programowania generycznego polegająca na wykorzystaniu przeciążania funkcji w celu wybrania, którą implementację funkcji wywołać w czasie wykonania}, lecz z konceptami rozwiązanie jest proste i oczywiste:

\begin{lstlisting}[frame=single]
template<Forward_iterator F, int n> 
void advance(F f, int n){
   while(n--) ++f;
}
\end{lstlisting}

\begin{lstlisting}[frame=single]
template<Random_access_iterator R, int n> 
void advance(R r, int n){
   r += n;
}
\end{lstlisting}

\begin{lstlisting}[frame=single]
void test(vector<string> &v, list<string> &l){
   auto pv = find(v, "test"); //(1)
   advance(pv, 2);
   
   auto pl = find(l, "test"); //(2)
   advance(pl, 2);
}
\end{lstlisting}

1) użycie szybkiego \verb#advance#
2) użycie wolnego \verb#advance#\newline

Skąd kompilator wie kiedy wywołać odpowiednią wersję \verb#advance#? Rozwiązanie przeciążania bazującego na konceptach jest zasadniczo proste:

\begin{itemize}
\item jeśli funkcja spełnia wymagania tylko jednego konceptu - wywołaj ją
\item jeśli funkcja nie spełnia wymagań żadnego konceptu wywołanie - błąd
\item sprawdź czy funkcja spełnia wymagania dwóch konceptów - zobacz czy wymagania jednego
konceptu są podzbiorem wymagań drugiego
\begin{itemize}
\item jeśli tak - wywołaj funkcję z największą liczbą wymagań (najściślejszych wymagań)
\item jeśli nie - błąd (dwuznaczność)
\end{itemize}
\end{itemize}

W funkcji \verb#test#, \verb#Random_access_iterator# ma więcej wymagań niż \newline \verb#Forward_iterator#, więc wywołuje się szybka wersja \verb#advance# dla iteratora zmiennej \verb#vector#. Dla ietratora zmiennej \verb#list#, pasuje tylko iterator \emph{Forward}, więc używamy wolnej wersji \verb#advance#.

\verb#Random_access_iterator# jest bardziej określony niż \verb#Forward_iterator# bo wymaga wszystkiego co 
\verb#Forward_iterator# i dodatkowo operatorów takich jak \verb#[]# i \verb#+#.

Ważne jest to że nie musimy wyraźnie określać ”hierarchii dziedziczenia” pośród konceptami czy definiować \emph{klas traits}\footnote{klasy traits}. Kompilator przetwarza hierarchię dla użytkownika. To jest prostsze, bardziej elastyczne i mniej podatne na błędy.

Przeciążanie oparte na konceptach eliminuje znaczącą ilość \emph{boiler-plate}\footnote{BP} z programowania generycznego i kodu meta programowania (użycia \verb#enable_if# \footnote{EI}).

Funkcja \verb#czyZnaleziono# ocenia czy element znaduje się w sekwencji

\begin{lstlisting}[frame=single]
template<Sequence S, Equality_comparable T>
   requires Same_as<T, value_type_t<S>>
bool czyZnaleziono(const S& seq, const T& value){
   for(const S& seq, const T& value)
      if(x == value)
         return true;
   return false;
}
\end{lstlisting}

Funkcja przyjmuje jako parametr sekwencję i wartość typu \verb#Equality_comparable#. Algorytm ma 3 ograniczenia:

\begin{itemize}

\item typ parametru \verb#seq# musi być typu \verb#Sequence#
\item typ parametru \verb#value# musi być typu \verb#Equality_comparable#
\item typ wartości typu \verb#S# musi być taki sam jak typ elementu zmiennej \verb#seq#

\end{itemize}

Definicje konceptów \verb#Range# i \verb#Sequence# potrzebne do tego algorytmu

\begin{lstlisting}[frame=single]
template<typename R>
concept bool Range() {
   return requires (R range){
      typename value_type_t<R>;
      typename iterator_t<R>;
      { begin(range) } -> iterator_t<R>;
      { end(range) } -> iterator_t<R>;
      requires Input_iterator<iterator_t<R>>();
      requires Same_as<value_type_t<R>,
         value_type_t<iterator_t<R>>>();
   };
};

template<typename S>
concept bool Sequence() {
   return Range<R> && requires (S seq) {
      { seq.front() } -> const value_type<S>&;
      { seq.back() } -> const value_type<S>&;
   };
};

\end{lstlisting}

Specyfikacja wymaga by typ \verb#Range# miał:

\begin{itemize}

\item dwa powiązane typy nazwane \verb#value_type_t# i \verb#iterator_t#
\item dwa poprawne operacje \verb#begin()# i \verb#end()#, które zwracają iteratory
\item typ wartości typu \verb#R# jest taki sam jak typ wartości iteratora tego typu.

\end{itemize}

Wydaje się w porządku. Możemy użyć tego algorytmu, żeby sprawdzić czy element jest w sekwencji. Niestety to nie działa dla wszystkich kolekcji:\newline

\noindent \verb#std::set<int> x { ... };#\newline
\verb#if(czyZnaleziono(x, 42)){#\newline
\verb#// błąd: brak operatora front() lub back()#\newline
\verb#}#\newline

Rozwiązaniem jest dodanie przeciążenia, które przyjmuje kontenery asocjacyjne

\begin{lstlisting}[frame=single]
template<Associative_container A, Same_as<key_type_t<T>> T>
bool czyZnaleziono(const A& a, const T& value){
   return a.find(value) != s.end();
}
\end{lstlisting}

Ta wersja funkcji \verb#czyZnaleziono# ma tylko dwa ograniczenia: typ \verb#A# musi być \verb#Associative_container# i typ \verb#T# musi być taki sam jak typ klucza \verb#A# (\verb#key_type_t<A>#). Dla kontenerów asocjacyjnych, szukamy wartości używając funkcji \verb#find()# a potem sprawdzamy czy się udało przez porównanie z \verb#end()#. W przeciwieństwie do wersji \verb#Sequence#, typ \verb#T# nie musi być \verb#Equality_comparable#. To dlatego, że precyzyjne wymagania typu \verb#T# są ustalone przez kontener asocjacyjny (te wymagania są ustalane przez oddzielny komparator lub funkcję haszującą.

Zdefiniowany koncept \verb#Associative_container#
\begin{lstlisting}[frame=single]
template<typename S>
concept bool Associative_container() {
   return Regular<S> && Range<S>() && requires {
      typename key_type_t<S>;
      requires Object_type<key_type_t<S>>;
   } && requires (S s, key_type_t<S> k){
      { s.empty() } -> bool;
      { s.size() } -> int;
      { s.find(k) } -> iterator_t<S>;
      { s.count(k) } -> int;
   };
};
\end{lstlisting}

\end{document}

\documentclass[11pt, a4paper]{article}
\usepackage{polski}
\usepackage[utf8]{inputenc}
\usepackage{listings}

\begin{document}
\lstset{language=C++}

\subsection{Przeciążanie funkcji przy użyciu konceptów}

Głowna idea programowania generycznego polega na używaniu tej samej nazwy dla równoważnych operacji używających różnych typów. A zatem, w grę wchodzi przeciążanie. Jest bardzo często przeoczaną, źle rozumianą ale niezwykle potężną cechą konceptów. Koncepty pozwalają na wybieranie spośród funkcji opierając się na właściwościach danych argumentów. Są przydatne nie tylko do poprawiania komunikatów o błędach i dokładnej specyfikacji interfejsów. Zwiększają również ekspresywność. Mogą być użyte do skracania kodu, robienia go bardziej ogólnym i zwiększania wydajności.

C++ jest językiem nie tylko assemblerowym wykorzystywanym do metaprogramowania szablonów. Koncepty pozwalają na podnoszenie poziomu programowania i upraszczają kod, bez angażowania dodatkowych zasobów czasu wykonania.

Przykład algorytmu \emph{advance}\footnote{Algorytm \emph{advance(it, n);} inkrementuje otrzymany iterator \emph{it} o \emph{n} elementów.} ze standardowej biblioteki

\begin{lstlisting}[frame=single]
template<typename Iter> void advance(Iter p, int n);
\end{lstlisting}

Potrzeba różnych wersji tego algorytmu, m.in.
\begin{itemize}
\item prostej, dla iteratorów \emph{Forward}, przechodzących przez sekwencję element po elemencie
\item szybkiej, dla iteratorów \emph{RandomAccess}, by wykorzystać umiejętność do zwiększania iteratora do arbitralnej pozycji w sekwencji używając jednej operacji.
\end{itemize}

Taka selekcja czasu kompilacji jest istotna dla wykonania kodu generycznego. Tradycyjnie, da się to zaimplementować używając funkcji pomocniczych lub techniki \emph{Tag Dispatching}\footnote{Technika programowania generycznego polegająca na wykorzystaniu przeciążania funkcji w celu wybrania, którą implementację funkcji wywołać w czasie wykonania}, lecz z konceptami rozwiązanie jest proste i oczywiste:

\begin{lstlisting}[frame=single]
template<Forward_iterator F, int n> 
void advance(F f, int n){
   while(n--) ++f;
}
\end{lstlisting}

\begin{lstlisting}[frame=single]
template<Random_access_iterator R, int n> 
void advance(R r, int n){
   r += n;
}
\end{lstlisting}

\begin{lstlisting}[frame=single]
void test(vector<string> &v, list<string> &l){
   auto pv = find(v, "test"); //(1)
   advance(pv, 2);
   
   auto pl = find(l, "test"); //(2)
   advance(pl, 2);
}
\end{lstlisting}

1) użycie szybkiego \verb#advance#
2) użycie wolnego \verb#advance#\newline

Skąd kompilator wie kiedy wywołać odpowiednią wersję \verb#advance#? Rozwiązanie przeciążania bazującego na konceptach jest zasadniczo proste:

\begin{itemize}
\item jeśli funkcja spełnia wymagania tylko jednego konceptu - wywołaj ją
\item jeśli funkcja nie spełnia wymagań żadnego konceptu wywołanie - błąd
\item sprawdź czy funkcja spełnia wymagania dwóch konceptów - zobacz czy wymagania jednego
konceptu są podzbiorem wymagań drugiego
\begin{itemize}
\item jeśli tak - wywołaj funkcję z największą liczbą wymagań (najściślejszych wymagań)
\item jeśli nie - błąd (dwuznaczność)
\end{itemize}
\end{itemize}

W funkcji \verb#test#, \verb#Random_access_iterator# ma więcej wymagań niż \newline \verb#Forward_iterator#, więc wywołuje się szybka wersja \verb#advance# dla iteratora zmiennej \verb#vector#. Dla ietratora zmiennej \verb#list#, pasuje tylko iterator \emph{Forward}, więc używamy wolnej wersji \verb#advance#.

\verb#Random_access_iterator# jest bardziej określony niż \verb#Forward_iterator# bo wymaga wszystkiego co 
\verb#Forward_iterator# i dodatkowo operatorów takich jak \verb#[]# i \verb#+#.

Ważne jest to że nie musimy wyraźnie określać ”hierarchii dziedziczenia” pośród konceptami czy definiować \emph{klas traits}\footnote{klasy traits}. Kompilator przetwarza hierarchię dla użytkownika. To jest prostsze, bardziej elastyczne i mniej podatne na błędy.

Przeciążanie oparte na konceptach eliminuje znaczącą ilość \emph{boiler-plate}\footnote{BP} z programowania generycznego i kodu meta programowania (użycia \verb#enable_if# \footnote{EI}).

Funkcja \verb#czyZnaleziono# ocenia czy element znaduje się w sekwencji

\begin{lstlisting}[frame=single]
template<Sequence S, Equality_comparable T>
   requires Same_as<T, value_type_t<S>>
bool czyZnaleziono(const S& seq, const T& value){
   for(const S& seq, const T& value)
      if(x == value)
         return true;
   return false;
}
\end{lstlisting}

Funkcja przyjmuje jako parametr sekwencję i wartość typu \verb#Equality_comparable#. Algorytm ma 3 ograniczenia:

\begin{itemize}

\item typ parametru \verb#seq# musi być typu \verb#Sequence#
\item typ parametru \verb#value# musi być typu \verb#Equality_comparable#
\item typ wartości typu \verb#S# musi być taki sam jak typ elementu zmiennej \verb#seq#

\end{itemize}

Definicje konceptów \verb#Range# i \verb#Sequence# potrzebne do tego algorytmu

\begin{lstlisting}[frame=single]
template<typename R>
concept bool Range() {
   return requires (R range){
      typename value_type_t<R>;
      typename iterator_t<R>;
      { begin(range) } -> iterator_t<R>;
      { end(range) } -> iterator_t<R>;
      requires Input_iterator<iterator_t<R>>();
      requires Same_as<value_type_t<R>,
         value_type_t<iterator_t<R>>>();
   };
};

template<typename S>
concept bool Sequence() {
   return Range<R> && requires (S seq) {
      { seq.front() } -> const value_type<S>&;
      { seq.back() } -> const value_type<S>&;
   };
};

\end{lstlisting}

Specyfikacja wymaga by typ \verb#Range# miał:

\begin{itemize}

\item dwa powiązane typy nazwane \verb#value_type_t# i \verb#iterator_t#
\item dwa poprawne operacje \verb#begin()# i \verb#end()#, które zwracają iteratory
\item typ wartości typu \verb#R# jest taki sam jak typ wartości iteratora tego typu.

\end{itemize}

Wydaje się w porządku. Możemy użyć tego algorytmu, żeby sprawdzić czy element jest w sekwencji. Niestety to nie działa dla wszystkich kolekcji:\newline

\noindent \verb#std::set<int> x { ... };#\newline
\verb#if(czyZnaleziono(x, 42)){#\newline
\verb#// błąd: brak operatora front() lub back()#\newline
\verb#}#\newline

Rozwiązaniem jest dodanie przeciążenia, które przyjmuje kontenery asocjacyjne

\begin{lstlisting}[frame=single]
template<Associative_container A, Same_as<key_type_t<T>> T>
bool czyZnaleziono(const A& a, const T& value){
   return a.find(value) != s.end();
}
\end{lstlisting}

Ta wersja funkcji \verb#czyZnaleziono# ma tylko dwa ograniczenia: typ \verb#A# musi być \verb#Associative_container# i typ \verb#T# musi być taki sam jak typ klucza \verb#A# (\verb#key_type_t<A>#). Dla kontenerów asocjacyjnych, szukamy wartości używając funkcji \verb#find()# a potem sprawdzamy czy się udało przez porównanie z \verb#end()#. W przeciwieństwie do wersji \verb#Sequence#, typ \verb#T# nie musi być \verb#Equality_comparable#. To dlatego, że precyzyjne wymagania typu \verb#T# są ustalone przez kontener asocjacyjny (te wymagania są ustalane przez oddzielny komparator lub funkcję haszującą.

Zdefiniowany koncept \verb#Associative_container#
\begin{lstlisting}[frame=single]
template<typename S>
concept bool Associative_container() {
   return Regular<S> && Range<S>() && requires {
      typename key_type_t<S>;
      requires Object_type<key_type_t<S>>;
   } && requires (S s, key_type_t<S> k){
      { s.empty() } -> bool;
      { s.size() } -> int;
      { s.find(k) } -> iterator_t<S>;
      { s.count(k) } -> int;
   };
};
\end{lstlisting}

\end{document}

\documentclass[11pt, a4paper]{article}
\usepackage{polski}
\usepackage[utf8]{inputenc}
\usepackage{listings}

\begin{document}
\lstset{language=C++}

\subsection{Przeciążanie funkcji przy użyciu konceptów}

Głowna idea programowania generycznego polega na używaniu tej samej nazwy dla równoważnych operacji używających różnych typów. A zatem, w grę wchodzi przeciążanie. Jest bardzo często przeoczaną, źle rozumianą ale niezwykle potężną cechą konceptów. Koncepty pozwalają na wybieranie spośród funkcji opierając się na właściwościach danych argumentów. Są przydatne nie tylko do poprawiania komunikatów o błędach i dokładnej specyfikacji interfejsów. Zwiększają również ekspresywność. Mogą być użyte do skracania kodu, robienia go bardziej ogólnym i zwiększania wydajności.

C++ jest językiem nie tylko assemblerowym wykorzystywanym do metaprogramowania szablonów. Koncepty pozwalają na podnoszenie poziomu programowania i upraszczają kod, bez angażowania dodatkowych zasobów czasu wykonania.

Przykład algorytmu \emph{advance}\footnote{Algorytm \emph{advance(it, n);} inkrementuje otrzymany iterator \emph{it} o \emph{n} elementów.} ze standardowej biblioteki

\begin{lstlisting}[frame=single]
template<typename Iter> void advance(Iter p, int n);
\end{lstlisting}

Potrzeba różnych wersji tego algorytmu, m.in.
\begin{itemize}
\item prostej, dla iteratorów \emph{Forward}, przechodzących przez sekwencję element po elemencie
\item szybkiej, dla iteratorów \emph{RandomAccess}, by wykorzystać umiejętność do zwiększania iteratora do arbitralnej pozycji w sekwencji używając jednej operacji.
\end{itemize}

Taka selekcja czasu kompilacji jest istotna dla wykonania kodu generycznego. Tradycyjnie, da się to zaimplementować używając funkcji pomocniczych lub techniki \emph{Tag Dispatching}\footnote{Technika programowania generycznego polegająca na wykorzystaniu przeciążania funkcji w celu wybrania, którą implementację funkcji wywołać w czasie wykonania}, lecz z konceptami rozwiązanie jest proste i oczywiste:

\begin{lstlisting}[frame=single]
template<Forward_iterator F, int n> 
void advance(F f, int n){
   while(n--) ++f;
}
\end{lstlisting}

\begin{lstlisting}[frame=single]
template<Random_access_iterator R, int n> 
void advance(R r, int n){
   r += n;
}
\end{lstlisting}

\begin{lstlisting}[frame=single]
void test(vector<string> &v, list<string> &l){
   auto pv = find(v, "test"); //(1)
   advance(pv, 2);
   
   auto pl = find(l, "test"); //(2)
   advance(pl, 2);
}
\end{lstlisting}

1) użycie szybkiego \verb#advance#
2) użycie wolnego \verb#advance#\newline

Skąd kompilator wie kiedy wywołać odpowiednią wersję \verb#advance#? Rozwiązanie przeciążania bazującego na konceptach jest zasadniczo proste:

\begin{itemize}
\item jeśli funkcja spełnia wymagania tylko jednego konceptu - wywołaj ją
\item jeśli funkcja nie spełnia wymagań żadnego konceptu wywołanie - błąd
\item sprawdź czy funkcja spełnia wymagania dwóch konceptów - zobacz czy wymagania jednego
konceptu są podzbiorem wymagań drugiego
\begin{itemize}
\item jeśli tak - wywołaj funkcję z największą liczbą wymagań (najściślejszych wymagań)
\item jeśli nie - błąd (dwuznaczność)
\end{itemize}
\end{itemize}

W funkcji \verb#test#, \verb#Random_access_iterator# ma więcej wymagań niż \newline \verb#Forward_iterator#, więc wywołuje się szybka wersja \verb#advance# dla iteratora zmiennej \verb#vector#. Dla ietratora zmiennej \verb#list#, pasuje tylko iterator \emph{Forward}, więc używamy wolnej wersji \verb#advance#.

\verb#Random_access_iterator# jest bardziej określony niż \verb#Forward_iterator# bo wymaga wszystkiego co 
\verb#Forward_iterator# i dodatkowo operatorów takich jak \verb#[]# i \verb#+#.

Ważne jest to że nie musimy wyraźnie określać ”hierarchii dziedziczenia” pośród konceptami czy definiować \emph{klas traits}\footnote{klasy traits}. Kompilator przetwarza hierarchię dla użytkownika. To jest prostsze, bardziej elastyczne i mniej podatne na błędy.

Przeciążanie oparte na konceptach eliminuje znaczącą ilość \emph{boiler-plate}\footnote{BP} z programowania generycznego i kodu meta programowania (użycia \verb#enable_if# \footnote{EI}).

Funkcja \verb#czyZnaleziono# ocenia czy element znaduje się w sekwencji

\begin{lstlisting}[frame=single]
template<Sequence S, Equality_comparable T>
   requires Same_as<T, value_type_t<S>>
bool czyZnaleziono(const S& seq, const T& value){
   for(const S& seq, const T& value)
      if(x == value)
         return true;
   return false;
}
\end{lstlisting}

Funkcja przyjmuje jako parametr sekwencję i wartość typu \verb#Equality_comparable#. Algorytm ma 3 ograniczenia:

\begin{itemize}

\item typ parametru \verb#seq# musi być typu \verb#Sequence#
\item typ parametru \verb#value# musi być typu \verb#Equality_comparable#
\item typ wartości typu \verb#S# musi być taki sam jak typ elementu zmiennej \verb#seq#

\end{itemize}

Definicje konceptów \verb#Range# i \verb#Sequence# potrzebne do tego algorytmu

\begin{lstlisting}[frame=single]
template<typename R>
concept bool Range() {
   return requires (R range){
      typename value_type_t<R>;
      typename iterator_t<R>;
      { begin(range) } -> iterator_t<R>;
      { end(range) } -> iterator_t<R>;
      requires Input_iterator<iterator_t<R>>();
      requires Same_as<value_type_t<R>,
         value_type_t<iterator_t<R>>>();
   };
};

template<typename S>
concept bool Sequence() {
   return Range<R> && requires (S seq) {
      { seq.front() } -> const value_type<S>&;
      { seq.back() } -> const value_type<S>&;
   };
};

\end{lstlisting}

Specyfikacja wymaga by typ \verb#Range# miał:

\begin{itemize}

\item dwa powiązane typy nazwane \verb#value_type_t# i \verb#iterator_t#
\item dwa poprawne operacje \verb#begin()# i \verb#end()#, które zwracają iteratory
\item typ wartości typu \verb#R# jest taki sam jak typ wartości iteratora tego typu.

\end{itemize}

Wydaje się w porządku. Możemy użyć tego algorytmu, żeby sprawdzić czy element jest w sekwencji. Niestety to nie działa dla wszystkich kolekcji:\newline

\noindent \verb#std::set<int> x { ... };#\newline
\verb#if(czyZnaleziono(x, 42)){#\newline
\verb#// błąd: brak operatora front() lub back()#\newline
\verb#}#\newline

Rozwiązaniem jest dodanie przeciążenia, które przyjmuje kontenery asocjacyjne

\begin{lstlisting}[frame=single]
template<Associative_container A, Same_as<key_type_t<T>> T>
bool czyZnaleziono(const A& a, const T& value){
   return a.find(value) != s.end();
}
\end{lstlisting}

Ta wersja funkcji \verb#czyZnaleziono# ma tylko dwa ograniczenia: typ \verb#A# musi być \verb#Associative_container# i typ \verb#T# musi być taki sam jak typ klucza \verb#A# (\verb#key_type_t<A>#). Dla kontenerów asocjacyjnych, szukamy wartości używając funkcji \verb#find()# a potem sprawdzamy czy się udało przez porównanie z \verb#end()#. W przeciwieństwie do wersji \verb#Sequence#, typ \verb#T# nie musi być \verb#Equality_comparable#. To dlatego, że precyzyjne wymagania typu \verb#T# są ustalone przez kontener asocjacyjny (te wymagania są ustalane przez oddzielny komparator lub funkcję haszującą.

Zdefiniowany koncept \verb#Associative_container#
\begin{lstlisting}[frame=single]
template<typename S>
concept bool Associative_container() {
   return Regular<S> && Range<S>() && requires {
      typename key_type_t<S>;
      requires Object_type<key_type_t<S>>;
   } && requires (S s, key_type_t<S> k){
      { s.empty() } -> bool;
      { s.size() } -> int;
      { s.find(k) } -> iterator_t<S>;
      { s.count(k) } -> int;
   };
};
\end{lstlisting}

\end{document}

%\documentclass[11pt, a4paper]{article}
\usepackage{polski}
\usepackage[utf8]{inputenc}
\usepackage{listings}

\begin{document}
\lstset{language=C++}

\subsection{Przeciążanie funkcji przy użyciu konceptów}

Głowna idea programowania generycznego polega na używaniu tej samej nazwy dla równoważnych operacji używających różnych typów. A zatem, w grę wchodzi przeciążanie. Jest bardzo często przeoczaną, źle rozumianą ale niezwykle potężną cechą konceptów. Koncepty pozwalają na wybieranie spośród funkcji opierając się na właściwościach danych argumentów. Są przydatne nie tylko do poprawiania komunikatów o błędach i dokładnej specyfikacji interfejsów. Zwiększają również ekspresywność. Mogą być użyte do skracania kodu, robienia go bardziej ogólnym i zwiększania wydajności.

C++ jest językiem nie tylko assemblerowym wykorzystywanym do metaprogramowania szablonów. Koncepty pozwalają na podnoszenie poziomu programowania i upraszczają kod, bez angażowania dodatkowych zasobów czasu wykonania.

Przykład algorytmu \emph{advance}\footnote{Algorytm \emph{advance(it, n);} inkrementuje otrzymany iterator \emph{it} o \emph{n} elementów.} ze standardowej biblioteki

\begin{lstlisting}[frame=single]
template<typename Iter> void advance(Iter p, int n);
\end{lstlisting}

Potrzeba różnych wersji tego algorytmu, m.in.
\begin{itemize}
\item prostej, dla iteratorów \emph{Forward}, przechodzących przez sekwencję element po elemencie
\item szybkiej, dla iteratorów \emph{RandomAccess}, by wykorzystać umiejętność do zwiększania iteratora do arbitralnej pozycji w sekwencji używając jednej operacji.
\end{itemize}

Taka selekcja czasu kompilacji jest istotna dla wykonania kodu generycznego. Tradycyjnie, da się to zaimplementować używając funkcji pomocniczych lub techniki \emph{Tag Dispatching}\footnote{Technika programowania generycznego polegająca na wykorzystaniu przeciążania funkcji w celu wybrania, którą implementację funkcji wywołać w czasie wykonania}, lecz z konceptami rozwiązanie jest proste i oczywiste:

\begin{lstlisting}[frame=single]
template<Forward_iterator F, int n> 
void advance(F f, int n){
   while(n--) ++f;
}
\end{lstlisting}

\begin{lstlisting}[frame=single]
template<Random_access_iterator R, int n> 
void advance(R r, int n){
   r += n;
}
\end{lstlisting}

\begin{lstlisting}[frame=single]
void test(vector<string> &v, list<string> &l){
   auto pv = find(v, "test"); //(1)
   advance(pv, 2);
   
   auto pl = find(l, "test"); //(2)
   advance(pl, 2);
}
\end{lstlisting}

1) użycie szybkiego \verb#advance#
2) użycie wolnego \verb#advance#\newline

Skąd kompilator wie kiedy wywołać odpowiednią wersję \verb#advance#? Rozwiązanie przeciążania bazującego na konceptach jest zasadniczo proste:

\begin{itemize}
\item jeśli funkcja spełnia wymagania tylko jednego konceptu - wywołaj ją
\item jeśli funkcja nie spełnia wymagań żadnego konceptu wywołanie - błąd
\item sprawdź czy funkcja spełnia wymagania dwóch konceptów - zobacz czy wymagania jednego
konceptu są podzbiorem wymagań drugiego
\begin{itemize}
\item jeśli tak - wywołaj funkcję z największą liczbą wymagań (najściślejszych wymagań)
\item jeśli nie - błąd (dwuznaczność)
\end{itemize}
\end{itemize}

W funkcji \verb#test#, \verb#Random_access_iterator# ma więcej wymagań niż \newline \verb#Forward_iterator#, więc wywołuje się szybka wersja \verb#advance# dla iteratora zmiennej \verb#vector#. Dla ietratora zmiennej \verb#list#, pasuje tylko iterator \emph{Forward}, więc używamy wolnej wersji \verb#advance#.

\verb#Random_access_iterator# jest bardziej określony niż \verb#Forward_iterator# bo wymaga wszystkiego co 
\verb#Forward_iterator# i dodatkowo operatorów takich jak \verb#[]# i \verb#+#.

Ważne jest to że nie musimy wyraźnie określać ”hierarchii dziedziczenia” pośród konceptami czy definiować \emph{klas traits}\footnote{klasy traits}. Kompilator przetwarza hierarchię dla użytkownika. To jest prostsze, bardziej elastyczne i mniej podatne na błędy.

Przeciążanie oparte na konceptach eliminuje znaczącą ilość \emph{boiler-plate}\footnote{BP} z programowania generycznego i kodu meta programowania (użycia \verb#enable_if# \footnote{EI}).

Funkcja \verb#czyZnaleziono# ocenia czy element znaduje się w sekwencji

\begin{lstlisting}[frame=single]
template<Sequence S, Equality_comparable T>
   requires Same_as<T, value_type_t<S>>
bool czyZnaleziono(const S& seq, const T& value){
   for(const S& seq, const T& value)
      if(x == value)
         return true;
   return false;
}
\end{lstlisting}

Funkcja przyjmuje jako parametr sekwencję i wartość typu \verb#Equality_comparable#. Algorytm ma 3 ograniczenia:

\begin{itemize}

\item typ parametru \verb#seq# musi być typu \verb#Sequence#
\item typ parametru \verb#value# musi być typu \verb#Equality_comparable#
\item typ wartości typu \verb#S# musi być taki sam jak typ elementu zmiennej \verb#seq#

\end{itemize}

Definicje konceptów \verb#Range# i \verb#Sequence# potrzebne do tego algorytmu

\begin{lstlisting}[frame=single]
template<typename R>
concept bool Range() {
   return requires (R range){
      typename value_type_t<R>;
      typename iterator_t<R>;
      { begin(range) } -> iterator_t<R>;
      { end(range) } -> iterator_t<R>;
      requires Input_iterator<iterator_t<R>>();
      requires Same_as<value_type_t<R>,
         value_type_t<iterator_t<R>>>();
   };
};

template<typename S>
concept bool Sequence() {
   return Range<R> && requires (S seq) {
      { seq.front() } -> const value_type<S>&;
      { seq.back() } -> const value_type<S>&;
   };
};

\end{lstlisting}

Specyfikacja wymaga by typ \verb#Range# miał:

\begin{itemize}

\item dwa powiązane typy nazwane \verb#value_type_t# i \verb#iterator_t#
\item dwa poprawne operacje \verb#begin()# i \verb#end()#, które zwracają iteratory
\item typ wartości typu \verb#R# jest taki sam jak typ wartości iteratora tego typu.

\end{itemize}

Wydaje się w porządku. Możemy użyć tego algorytmu, żeby sprawdzić czy element jest w sekwencji. Niestety to nie działa dla wszystkich kolekcji:\newline

\noindent \verb#std::set<int> x { ... };#\newline
\verb#if(czyZnaleziono(x, 42)){#\newline
\verb#// błąd: brak operatora front() lub back()#\newline
\verb#}#\newline

Rozwiązaniem jest dodanie przeciążenia, które przyjmuje kontenery asocjacyjne

\begin{lstlisting}[frame=single]
template<Associative_container A, Same_as<key_type_t<T>> T>
bool czyZnaleziono(const A& a, const T& value){
   return a.find(value) != s.end();
}
\end{lstlisting}

Ta wersja funkcji \verb#czyZnaleziono# ma tylko dwa ograniczenia: typ \verb#A# musi być \verb#Associative_container# i typ \verb#T# musi być taki sam jak typ klucza \verb#A# (\verb#key_type_t<A>#). Dla kontenerów asocjacyjnych, szukamy wartości używając funkcji \verb#find()# a potem sprawdzamy czy się udało przez porównanie z \verb#end()#. W przeciwieństwie do wersji \verb#Sequence#, typ \verb#T# nie musi być \verb#Equality_comparable#. To dlatego, że precyzyjne wymagania typu \verb#T# są ustalone przez kontener asocjacyjny (te wymagania są ustalane przez oddzielny komparator lub funkcję haszującą.

Zdefiniowany koncept \verb#Associative_container#
\begin{lstlisting}[frame=single]
template<typename S>
concept bool Associative_container() {
   return Regular<S> && Range<S>() && requires {
      typename key_type_t<S>;
      requires Object_type<key_type_t<S>>;
   } && requires (S s, key_type_t<S> k){
      { s.empty() } -> bool;
      { s.size() } -> int;
      { s.find(k) } -> iterator_t<S>;
      { s.count(k) } -> int;
   };
};
\end{lstlisting}

\end{document}

\documentclass[11pt, a4paper]{article}
\usepackage{polski}
\usepackage[utf8]{inputenc}
\usepackage{listings}

\begin{document}
\lstset{language=C++}

\subsection{Przeciążanie funkcji przy użyciu konceptów}

Głowna idea programowania generycznego polega na używaniu tej samej nazwy dla równoważnych operacji używających różnych typów. A zatem, w grę wchodzi przeciążanie. Jest bardzo często przeoczaną, źle rozumianą ale niezwykle potężną cechą konceptów. Koncepty pozwalają na wybieranie spośród funkcji opierając się na właściwościach danych argumentów. Są przydatne nie tylko do poprawiania komunikatów o błędach i dokładnej specyfikacji interfejsów. Zwiększają również ekspresywność. Mogą być użyte do skracania kodu, robienia go bardziej ogólnym i zwiększania wydajności.

C++ jest językiem nie tylko assemblerowym wykorzystywanym do metaprogramowania szablonów. Koncepty pozwalają na podnoszenie poziomu programowania i upraszczają kod, bez angażowania dodatkowych zasobów czasu wykonania.

Przykład algorytmu \emph{advance}\footnote{Algorytm \emph{advance(it, n);} inkrementuje otrzymany iterator \emph{it} o \emph{n} elementów.} ze standardowej biblioteki

\begin{lstlisting}[frame=single]
template<typename Iter> void advance(Iter p, int n);
\end{lstlisting}

Potrzeba różnych wersji tego algorytmu, m.in.
\begin{itemize}
\item prostej, dla iteratorów \emph{Forward}, przechodzących przez sekwencję element po elemencie
\item szybkiej, dla iteratorów \emph{RandomAccess}, by wykorzystać umiejętność do zwiększania iteratora do arbitralnej pozycji w sekwencji używając jednej operacji.
\end{itemize}

Taka selekcja czasu kompilacji jest istotna dla wykonania kodu generycznego. Tradycyjnie, da się to zaimplementować używając funkcji pomocniczych lub techniki \emph{Tag Dispatching}\footnote{Technika programowania generycznego polegająca na wykorzystaniu przeciążania funkcji w celu wybrania, którą implementację funkcji wywołać w czasie wykonania}, lecz z konceptami rozwiązanie jest proste i oczywiste:

\begin{lstlisting}[frame=single]
template<Forward_iterator F, int n> 
void advance(F f, int n){
   while(n--) ++f;
}
\end{lstlisting}

\begin{lstlisting}[frame=single]
template<Random_access_iterator R, int n> 
void advance(R r, int n){
   r += n;
}
\end{lstlisting}

\begin{lstlisting}[frame=single]
void test(vector<string> &v, list<string> &l){
   auto pv = find(v, "test"); //(1)
   advance(pv, 2);
   
   auto pl = find(l, "test"); //(2)
   advance(pl, 2);
}
\end{lstlisting}

1) użycie szybkiego \verb#advance#
2) użycie wolnego \verb#advance#\newline

Skąd kompilator wie kiedy wywołać odpowiednią wersję \verb#advance#? Rozwiązanie przeciążania bazującego na konceptach jest zasadniczo proste:

\begin{itemize}
\item jeśli funkcja spełnia wymagania tylko jednego konceptu - wywołaj ją
\item jeśli funkcja nie spełnia wymagań żadnego konceptu wywołanie - błąd
\item sprawdź czy funkcja spełnia wymagania dwóch konceptów - zobacz czy wymagania jednego
konceptu są podzbiorem wymagań drugiego
\begin{itemize}
\item jeśli tak - wywołaj funkcję z największą liczbą wymagań (najściślejszych wymagań)
\item jeśli nie - błąd (dwuznaczność)
\end{itemize}
\end{itemize}

W funkcji \verb#test#, \verb#Random_access_iterator# ma więcej wymagań niż \newline \verb#Forward_iterator#, więc wywołuje się szybka wersja \verb#advance# dla iteratora zmiennej \verb#vector#. Dla ietratora zmiennej \verb#list#, pasuje tylko iterator \emph{Forward}, więc używamy wolnej wersji \verb#advance#.

\verb#Random_access_iterator# jest bardziej określony niż \verb#Forward_iterator# bo wymaga wszystkiego co 
\verb#Forward_iterator# i dodatkowo operatorów takich jak \verb#[]# i \verb#+#.

Ważne jest to że nie musimy wyraźnie określać ”hierarchii dziedziczenia” pośród konceptami czy definiować \emph{klas traits}\footnote{klasy traits}. Kompilator przetwarza hierarchię dla użytkownika. To jest prostsze, bardziej elastyczne i mniej podatne na błędy.

Przeciążanie oparte na konceptach eliminuje znaczącą ilość \emph{boiler-plate}\footnote{BP} z programowania generycznego i kodu meta programowania (użycia \verb#enable_if# \footnote{EI}).

Funkcja \verb#czyZnaleziono# ocenia czy element znaduje się w sekwencji

\begin{lstlisting}[frame=single]
template<Sequence S, Equality_comparable T>
   requires Same_as<T, value_type_t<S>>
bool czyZnaleziono(const S& seq, const T& value){
   for(const S& seq, const T& value)
      if(x == value)
         return true;
   return false;
}
\end{lstlisting}

Funkcja przyjmuje jako parametr sekwencję i wartość typu \verb#Equality_comparable#. Algorytm ma 3 ograniczenia:

\begin{itemize}

\item typ parametru \verb#seq# musi być typu \verb#Sequence#
\item typ parametru \verb#value# musi być typu \verb#Equality_comparable#
\item typ wartości typu \verb#S# musi być taki sam jak typ elementu zmiennej \verb#seq#

\end{itemize}

Definicje konceptów \verb#Range# i \verb#Sequence# potrzebne do tego algorytmu

\begin{lstlisting}[frame=single]
template<typename R>
concept bool Range() {
   return requires (R range){
      typename value_type_t<R>;
      typename iterator_t<R>;
      { begin(range) } -> iterator_t<R>;
      { end(range) } -> iterator_t<R>;
      requires Input_iterator<iterator_t<R>>();
      requires Same_as<value_type_t<R>,
         value_type_t<iterator_t<R>>>();
   };
};

template<typename S>
concept bool Sequence() {
   return Range<R> && requires (S seq) {
      { seq.front() } -> const value_type<S>&;
      { seq.back() } -> const value_type<S>&;
   };
};

\end{lstlisting}

Specyfikacja wymaga by typ \verb#Range# miał:

\begin{itemize}

\item dwa powiązane typy nazwane \verb#value_type_t# i \verb#iterator_t#
\item dwa poprawne operacje \verb#begin()# i \verb#end()#, które zwracają iteratory
\item typ wartości typu \verb#R# jest taki sam jak typ wartości iteratora tego typu.

\end{itemize}

Wydaje się w porządku. Możemy użyć tego algorytmu, żeby sprawdzić czy element jest w sekwencji. Niestety to nie działa dla wszystkich kolekcji:\newline

\noindent \verb#std::set<int> x { ... };#\newline
\verb#if(czyZnaleziono(x, 42)){#\newline
\verb#// błąd: brak operatora front() lub back()#\newline
\verb#}#\newline

Rozwiązaniem jest dodanie przeciążenia, które przyjmuje kontenery asocjacyjne

\begin{lstlisting}[frame=single]
template<Associative_container A, Same_as<key_type_t<T>> T>
bool czyZnaleziono(const A& a, const T& value){
   return a.find(value) != s.end();
}
\end{lstlisting}

Ta wersja funkcji \verb#czyZnaleziono# ma tylko dwa ograniczenia: typ \verb#A# musi być \verb#Associative_container# i typ \verb#T# musi być taki sam jak typ klucza \verb#A# (\verb#key_type_t<A>#). Dla kontenerów asocjacyjnych, szukamy wartości używając funkcji \verb#find()# a potem sprawdzamy czy się udało przez porównanie z \verb#end()#. W przeciwieństwie do wersji \verb#Sequence#, typ \verb#T# nie musi być \verb#Equality_comparable#. To dlatego, że precyzyjne wymagania typu \verb#T# są ustalone przez kontener asocjacyjny (te wymagania są ustalane przez oddzielny komparator lub funkcję haszującą.

Zdefiniowany koncept \verb#Associative_container#
\begin{lstlisting}[frame=single]
template<typename S>
concept bool Associative_container() {
   return Regular<S> && Range<S>() && requires {
      typename key_type_t<S>;
      requires Object_type<key_type_t<S>>;
   } && requires (S s, key_type_t<S> k){
      { s.empty() } -> bool;
      { s.size() } -> int;
      { s.find(k) } -> iterator_t<S>;
      { s.count(k) } -> int;
   };
};
\end{lstlisting}

\end{document}

\documentclass[11pt, a4paper]{article}
\usepackage{polski}
\usepackage[utf8]{inputenc}
\usepackage{listings}

\begin{document}
\lstset{language=C++}

\subsection{Przeciążanie funkcji przy użyciu konceptów}

Głowna idea programowania generycznego polega na używaniu tej samej nazwy dla równoważnych operacji używających różnych typów. A zatem, w grę wchodzi przeciążanie. Jest bardzo często przeoczaną, źle rozumianą ale niezwykle potężną cechą konceptów. Koncepty pozwalają na wybieranie spośród funkcji opierając się na właściwościach danych argumentów. Są przydatne nie tylko do poprawiania komunikatów o błędach i dokładnej specyfikacji interfejsów. Zwiększają również ekspresywność. Mogą być użyte do skracania kodu, robienia go bardziej ogólnym i zwiększania wydajności.

C++ jest językiem nie tylko assemblerowym wykorzystywanym do metaprogramowania szablonów. Koncepty pozwalają na podnoszenie poziomu programowania i upraszczają kod, bez angażowania dodatkowych zasobów czasu wykonania.

Przykład algorytmu \emph{advance}\footnote{Algorytm \emph{advance(it, n);} inkrementuje otrzymany iterator \emph{it} o \emph{n} elementów.} ze standardowej biblioteki

\begin{lstlisting}[frame=single]
template<typename Iter> void advance(Iter p, int n);
\end{lstlisting}

Potrzeba różnych wersji tego algorytmu, m.in.
\begin{itemize}
\item prostej, dla iteratorów \emph{Forward}, przechodzących przez sekwencję element po elemencie
\item szybkiej, dla iteratorów \emph{RandomAccess}, by wykorzystać umiejętność do zwiększania iteratora do arbitralnej pozycji w sekwencji używając jednej operacji.
\end{itemize}

Taka selekcja czasu kompilacji jest istotna dla wykonania kodu generycznego. Tradycyjnie, da się to zaimplementować używając funkcji pomocniczych lub techniki \emph{Tag Dispatching}\footnote{Technika programowania generycznego polegająca na wykorzystaniu przeciążania funkcji w celu wybrania, którą implementację funkcji wywołać w czasie wykonania}, lecz z konceptami rozwiązanie jest proste i oczywiste:

\begin{lstlisting}[frame=single]
template<Forward_iterator F, int n> 
void advance(F f, int n){
   while(n--) ++f;
}
\end{lstlisting}

\begin{lstlisting}[frame=single]
template<Random_access_iterator R, int n> 
void advance(R r, int n){
   r += n;
}
\end{lstlisting}

\begin{lstlisting}[frame=single]
void test(vector<string> &v, list<string> &l){
   auto pv = find(v, "test"); //(1)
   advance(pv, 2);
   
   auto pl = find(l, "test"); //(2)
   advance(pl, 2);
}
\end{lstlisting}

1) użycie szybkiego \verb#advance#
2) użycie wolnego \verb#advance#\newline

Skąd kompilator wie kiedy wywołać odpowiednią wersję \verb#advance#? Rozwiązanie przeciążania bazującego na konceptach jest zasadniczo proste:

\begin{itemize}
\item jeśli funkcja spełnia wymagania tylko jednego konceptu - wywołaj ją
\item jeśli funkcja nie spełnia wymagań żadnego konceptu wywołanie - błąd
\item sprawdź czy funkcja spełnia wymagania dwóch konceptów - zobacz czy wymagania jednego
konceptu są podzbiorem wymagań drugiego
\begin{itemize}
\item jeśli tak - wywołaj funkcję z największą liczbą wymagań (najściślejszych wymagań)
\item jeśli nie - błąd (dwuznaczność)
\end{itemize}
\end{itemize}

W funkcji \verb#test#, \verb#Random_access_iterator# ma więcej wymagań niż \newline \verb#Forward_iterator#, więc wywołuje się szybka wersja \verb#advance# dla iteratora zmiennej \verb#vector#. Dla ietratora zmiennej \verb#list#, pasuje tylko iterator \emph{Forward}, więc używamy wolnej wersji \verb#advance#.

\verb#Random_access_iterator# jest bardziej określony niż \verb#Forward_iterator# bo wymaga wszystkiego co 
\verb#Forward_iterator# i dodatkowo operatorów takich jak \verb#[]# i \verb#+#.

Ważne jest to że nie musimy wyraźnie określać ”hierarchii dziedziczenia” pośród konceptami czy definiować \emph{klas traits}\footnote{klasy traits}. Kompilator przetwarza hierarchię dla użytkownika. To jest prostsze, bardziej elastyczne i mniej podatne na błędy.

Przeciążanie oparte na konceptach eliminuje znaczącą ilość \emph{boiler-plate}\footnote{BP} z programowania generycznego i kodu meta programowania (użycia \verb#enable_if# \footnote{EI}).

Funkcja \verb#czyZnaleziono# ocenia czy element znaduje się w sekwencji

\begin{lstlisting}[frame=single]
template<Sequence S, Equality_comparable T>
   requires Same_as<T, value_type_t<S>>
bool czyZnaleziono(const S& seq, const T& value){
   for(const S& seq, const T& value)
      if(x == value)
         return true;
   return false;
}
\end{lstlisting}

Funkcja przyjmuje jako parametr sekwencję i wartość typu \verb#Equality_comparable#. Algorytm ma 3 ograniczenia:

\begin{itemize}

\item typ parametru \verb#seq# musi być typu \verb#Sequence#
\item typ parametru \verb#value# musi być typu \verb#Equality_comparable#
\item typ wartości typu \verb#S# musi być taki sam jak typ elementu zmiennej \verb#seq#

\end{itemize}

Definicje konceptów \verb#Range# i \verb#Sequence# potrzebne do tego algorytmu

\begin{lstlisting}[frame=single]
template<typename R>
concept bool Range() {
   return requires (R range){
      typename value_type_t<R>;
      typename iterator_t<R>;
      { begin(range) } -> iterator_t<R>;
      { end(range) } -> iterator_t<R>;
      requires Input_iterator<iterator_t<R>>();
      requires Same_as<value_type_t<R>,
         value_type_t<iterator_t<R>>>();
   };
};

template<typename S>
concept bool Sequence() {
   return Range<R> && requires (S seq) {
      { seq.front() } -> const value_type<S>&;
      { seq.back() } -> const value_type<S>&;
   };
};

\end{lstlisting}

Specyfikacja wymaga by typ \verb#Range# miał:

\begin{itemize}

\item dwa powiązane typy nazwane \verb#value_type_t# i \verb#iterator_t#
\item dwa poprawne operacje \verb#begin()# i \verb#end()#, które zwracają iteratory
\item typ wartości typu \verb#R# jest taki sam jak typ wartości iteratora tego typu.

\end{itemize}

Wydaje się w porządku. Możemy użyć tego algorytmu, żeby sprawdzić czy element jest w sekwencji. Niestety to nie działa dla wszystkich kolekcji:\newline

\noindent \verb#std::set<int> x { ... };#\newline
\verb#if(czyZnaleziono(x, 42)){#\newline
\verb#// błąd: brak operatora front() lub back()#\newline
\verb#}#\newline

Rozwiązaniem jest dodanie przeciążenia, które przyjmuje kontenery asocjacyjne

\begin{lstlisting}[frame=single]
template<Associative_container A, Same_as<key_type_t<T>> T>
bool czyZnaleziono(const A& a, const T& value){
   return a.find(value) != s.end();
}
\end{lstlisting}

Ta wersja funkcji \verb#czyZnaleziono# ma tylko dwa ograniczenia: typ \verb#A# musi być \verb#Associative_container# i typ \verb#T# musi być taki sam jak typ klucza \verb#A# (\verb#key_type_t<A>#). Dla kontenerów asocjacyjnych, szukamy wartości używając funkcji \verb#find()# a potem sprawdzamy czy się udało przez porównanie z \verb#end()#. W przeciwieństwie do wersji \verb#Sequence#, typ \verb#T# nie musi być \verb#Equality_comparable#. To dlatego, że precyzyjne wymagania typu \verb#T# są ustalone przez kontener asocjacyjny (te wymagania są ustalane przez oddzielny komparator lub funkcję haszującą.

Zdefiniowany koncept \verb#Associative_container#
\begin{lstlisting}[frame=single]
template<typename S>
concept bool Associative_container() {
   return Regular<S> && Range<S>() && requires {
      typename key_type_t<S>;
      requires Object_type<key_type_t<S>>;
   } && requires (S s, key_type_t<S> k){
      { s.empty() } -> bool;
      { s.size() } -> int;
      { s.find(k) } -> iterator_t<S>;
      { s.count(k) } -> int;
   };
};
\end{lstlisting}

\end{document}

\documentclass[11pt, a4paper]{article}
\usepackage{polski}
\usepackage[utf8]{inputenc}
\usepackage{listings}

\begin{document}
\lstset{language=C++}

\subsection{Przeciążanie funkcji przy użyciu konceptów}

Głowna idea programowania generycznego polega na używaniu tej samej nazwy dla równoważnych operacji używających różnych typów. A zatem, w grę wchodzi przeciążanie. Jest bardzo często przeoczaną, źle rozumianą ale niezwykle potężną cechą konceptów. Koncepty pozwalają na wybieranie spośród funkcji opierając się na właściwościach danych argumentów. Są przydatne nie tylko do poprawiania komunikatów o błędach i dokładnej specyfikacji interfejsów. Zwiększają również ekspresywność. Mogą być użyte do skracania kodu, robienia go bardziej ogólnym i zwiększania wydajności.

C++ jest językiem nie tylko assemblerowym wykorzystywanym do metaprogramowania szablonów. Koncepty pozwalają na podnoszenie poziomu programowania i upraszczają kod, bez angażowania dodatkowych zasobów czasu wykonania.

Przykład algorytmu \emph{advance}\footnote{Algorytm \emph{advance(it, n);} inkrementuje otrzymany iterator \emph{it} o \emph{n} elementów.} ze standardowej biblioteki

\begin{lstlisting}[frame=single]
template<typename Iter> void advance(Iter p, int n);
\end{lstlisting}

Potrzeba różnych wersji tego algorytmu, m.in.
\begin{itemize}
\item prostej, dla iteratorów \emph{Forward}, przechodzących przez sekwencję element po elemencie
\item szybkiej, dla iteratorów \emph{RandomAccess}, by wykorzystać umiejętność do zwiększania iteratora do arbitralnej pozycji w sekwencji używając jednej operacji.
\end{itemize}

Taka selekcja czasu kompilacji jest istotna dla wykonania kodu generycznego. Tradycyjnie, da się to zaimplementować używając funkcji pomocniczych lub techniki \emph{Tag Dispatching}\footnote{Technika programowania generycznego polegająca na wykorzystaniu przeciążania funkcji w celu wybrania, którą implementację funkcji wywołać w czasie wykonania}, lecz z konceptami rozwiązanie jest proste i oczywiste:

\begin{lstlisting}[frame=single]
template<Forward_iterator F, int n> 
void advance(F f, int n){
   while(n--) ++f;
}
\end{lstlisting}

\begin{lstlisting}[frame=single]
template<Random_access_iterator R, int n> 
void advance(R r, int n){
   r += n;
}
\end{lstlisting}

\begin{lstlisting}[frame=single]
void test(vector<string> &v, list<string> &l){
   auto pv = find(v, "test"); //(1)
   advance(pv, 2);
   
   auto pl = find(l, "test"); //(2)
   advance(pl, 2);
}
\end{lstlisting}

1) użycie szybkiego \verb#advance#
2) użycie wolnego \verb#advance#\newline

Skąd kompilator wie kiedy wywołać odpowiednią wersję \verb#advance#? Rozwiązanie przeciążania bazującego na konceptach jest zasadniczo proste:

\begin{itemize}
\item jeśli funkcja spełnia wymagania tylko jednego konceptu - wywołaj ją
\item jeśli funkcja nie spełnia wymagań żadnego konceptu wywołanie - błąd
\item sprawdź czy funkcja spełnia wymagania dwóch konceptów - zobacz czy wymagania jednego
konceptu są podzbiorem wymagań drugiego
\begin{itemize}
\item jeśli tak - wywołaj funkcję z największą liczbą wymagań (najściślejszych wymagań)
\item jeśli nie - błąd (dwuznaczność)
\end{itemize}
\end{itemize}

W funkcji \verb#test#, \verb#Random_access_iterator# ma więcej wymagań niż \newline \verb#Forward_iterator#, więc wywołuje się szybka wersja \verb#advance# dla iteratora zmiennej \verb#vector#. Dla ietratora zmiennej \verb#list#, pasuje tylko iterator \emph{Forward}, więc używamy wolnej wersji \verb#advance#.

\verb#Random_access_iterator# jest bardziej określony niż \verb#Forward_iterator# bo wymaga wszystkiego co 
\verb#Forward_iterator# i dodatkowo operatorów takich jak \verb#[]# i \verb#+#.

Ważne jest to że nie musimy wyraźnie określać ”hierarchii dziedziczenia” pośród konceptami czy definiować \emph{klas traits}\footnote{klasy traits}. Kompilator przetwarza hierarchię dla użytkownika. To jest prostsze, bardziej elastyczne i mniej podatne na błędy.

Przeciążanie oparte na konceptach eliminuje znaczącą ilość \emph{boiler-plate}\footnote{BP} z programowania generycznego i kodu meta programowania (użycia \verb#enable_if# \footnote{EI}).

Funkcja \verb#czyZnaleziono# ocenia czy element znaduje się w sekwencji

\begin{lstlisting}[frame=single]
template<Sequence S, Equality_comparable T>
   requires Same_as<T, value_type_t<S>>
bool czyZnaleziono(const S& seq, const T& value){
   for(const S& seq, const T& value)
      if(x == value)
         return true;
   return false;
}
\end{lstlisting}

Funkcja przyjmuje jako parametr sekwencję i wartość typu \verb#Equality_comparable#. Algorytm ma 3 ograniczenia:

\begin{itemize}

\item typ parametru \verb#seq# musi być typu \verb#Sequence#
\item typ parametru \verb#value# musi być typu \verb#Equality_comparable#
\item typ wartości typu \verb#S# musi być taki sam jak typ elementu zmiennej \verb#seq#

\end{itemize}

Definicje konceptów \verb#Range# i \verb#Sequence# potrzebne do tego algorytmu

\begin{lstlisting}[frame=single]
template<typename R>
concept bool Range() {
   return requires (R range){
      typename value_type_t<R>;
      typename iterator_t<R>;
      { begin(range) } -> iterator_t<R>;
      { end(range) } -> iterator_t<R>;
      requires Input_iterator<iterator_t<R>>();
      requires Same_as<value_type_t<R>,
         value_type_t<iterator_t<R>>>();
   };
};

template<typename S>
concept bool Sequence() {
   return Range<R> && requires (S seq) {
      { seq.front() } -> const value_type<S>&;
      { seq.back() } -> const value_type<S>&;
   };
};

\end{lstlisting}

Specyfikacja wymaga by typ \verb#Range# miał:

\begin{itemize}

\item dwa powiązane typy nazwane \verb#value_type_t# i \verb#iterator_t#
\item dwa poprawne operacje \verb#begin()# i \verb#end()#, które zwracają iteratory
\item typ wartości typu \verb#R# jest taki sam jak typ wartości iteratora tego typu.

\end{itemize}

Wydaje się w porządku. Możemy użyć tego algorytmu, żeby sprawdzić czy element jest w sekwencji. Niestety to nie działa dla wszystkich kolekcji:\newline

\noindent \verb#std::set<int> x { ... };#\newline
\verb#if(czyZnaleziono(x, 42)){#\newline
\verb#// błąd: brak operatora front() lub back()#\newline
\verb#}#\newline

Rozwiązaniem jest dodanie przeciążenia, które przyjmuje kontenery asocjacyjne

\begin{lstlisting}[frame=single]
template<Associative_container A, Same_as<key_type_t<T>> T>
bool czyZnaleziono(const A& a, const T& value){
   return a.find(value) != s.end();
}
\end{lstlisting}

Ta wersja funkcji \verb#czyZnaleziono# ma tylko dwa ograniczenia: typ \verb#A# musi być \verb#Associative_container# i typ \verb#T# musi być taki sam jak typ klucza \verb#A# (\verb#key_type_t<A>#). Dla kontenerów asocjacyjnych, szukamy wartości używając funkcji \verb#find()# a potem sprawdzamy czy się udało przez porównanie z \verb#end()#. W przeciwieństwie do wersji \verb#Sequence#, typ \verb#T# nie musi być \verb#Equality_comparable#. To dlatego, że precyzyjne wymagania typu \verb#T# są ustalone przez kontener asocjacyjny (te wymagania są ustalane przez oddzielny komparator lub funkcję haszującą.

Zdefiniowany koncept \verb#Associative_container#
\begin{lstlisting}[frame=single]
template<typename S>
concept bool Associative_container() {
   return Regular<S> && Range<S>() && requires {
      typename key_type_t<S>;
      requires Object_type<key_type_t<S>>;
   } && requires (S s, key_type_t<S> k){
      { s.empty() } -> bool;
      { s.size() } -> int;
      { s.find(k) } -> iterator_t<S>;
      { s.count(k) } -> int;
   };
};
\end{lstlisting}

\end{document}

%\documentclass[11pt, a4paper]{article}
\usepackage{polski}
\usepackage[utf8]{inputenc}
\usepackage{listings}

\begin{document}
\lstset{language=C++}

\subsection{Przeciążanie funkcji przy użyciu konceptów}

Głowna idea programowania generycznego polega na używaniu tej samej nazwy dla równoważnych operacji używających różnych typów. A zatem, w grę wchodzi przeciążanie. Jest bardzo często przeoczaną, źle rozumianą ale niezwykle potężną cechą konceptów. Koncepty pozwalają na wybieranie spośród funkcji opierając się na właściwościach danych argumentów. Są przydatne nie tylko do poprawiania komunikatów o błędach i dokładnej specyfikacji interfejsów. Zwiększają również ekspresywność. Mogą być użyte do skracania kodu, robienia go bardziej ogólnym i zwiększania wydajności.

C++ jest językiem nie tylko assemblerowym wykorzystywanym do metaprogramowania szablonów. Koncepty pozwalają na podnoszenie poziomu programowania i upraszczają kod, bez angażowania dodatkowych zasobów czasu wykonania.

Przykład algorytmu \emph{advance}\footnote{Algorytm \emph{advance(it, n);} inkrementuje otrzymany iterator \emph{it} o \emph{n} elementów.} ze standardowej biblioteki

\begin{lstlisting}[frame=single]
template<typename Iter> void advance(Iter p, int n);
\end{lstlisting}

Potrzeba różnych wersji tego algorytmu, m.in.
\begin{itemize}
\item prostej, dla iteratorów \emph{Forward}, przechodzących przez sekwencję element po elemencie
\item szybkiej, dla iteratorów \emph{RandomAccess}, by wykorzystać umiejętność do zwiększania iteratora do arbitralnej pozycji w sekwencji używając jednej operacji.
\end{itemize}

Taka selekcja czasu kompilacji jest istotna dla wykonania kodu generycznego. Tradycyjnie, da się to zaimplementować używając funkcji pomocniczych lub techniki \emph{Tag Dispatching}\footnote{Technika programowania generycznego polegająca na wykorzystaniu przeciążania funkcji w celu wybrania, którą implementację funkcji wywołać w czasie wykonania}, lecz z konceptami rozwiązanie jest proste i oczywiste:

\begin{lstlisting}[frame=single]
template<Forward_iterator F, int n> 
void advance(F f, int n){
   while(n--) ++f;
}
\end{lstlisting}

\begin{lstlisting}[frame=single]
template<Random_access_iterator R, int n> 
void advance(R r, int n){
   r += n;
}
\end{lstlisting}

\begin{lstlisting}[frame=single]
void test(vector<string> &v, list<string> &l){
   auto pv = find(v, "test"); //(1)
   advance(pv, 2);
   
   auto pl = find(l, "test"); //(2)
   advance(pl, 2);
}
\end{lstlisting}

1) użycie szybkiego \verb#advance#
2) użycie wolnego \verb#advance#\newline

Skąd kompilator wie kiedy wywołać odpowiednią wersję \verb#advance#? Rozwiązanie przeciążania bazującego na konceptach jest zasadniczo proste:

\begin{itemize}
\item jeśli funkcja spełnia wymagania tylko jednego konceptu - wywołaj ją
\item jeśli funkcja nie spełnia wymagań żadnego konceptu wywołanie - błąd
\item sprawdź czy funkcja spełnia wymagania dwóch konceptów - zobacz czy wymagania jednego
konceptu są podzbiorem wymagań drugiego
\begin{itemize}
\item jeśli tak - wywołaj funkcję z największą liczbą wymagań (najściślejszych wymagań)
\item jeśli nie - błąd (dwuznaczność)
\end{itemize}
\end{itemize}

W funkcji \verb#test#, \verb#Random_access_iterator# ma więcej wymagań niż \newline \verb#Forward_iterator#, więc wywołuje się szybka wersja \verb#advance# dla iteratora zmiennej \verb#vector#. Dla ietratora zmiennej \verb#list#, pasuje tylko iterator \emph{Forward}, więc używamy wolnej wersji \verb#advance#.

\verb#Random_access_iterator# jest bardziej określony niż \verb#Forward_iterator# bo wymaga wszystkiego co 
\verb#Forward_iterator# i dodatkowo operatorów takich jak \verb#[]# i \verb#+#.

Ważne jest to że nie musimy wyraźnie określać ”hierarchii dziedziczenia” pośród konceptami czy definiować \emph{klas traits}\footnote{klasy traits}. Kompilator przetwarza hierarchię dla użytkownika. To jest prostsze, bardziej elastyczne i mniej podatne na błędy.

Przeciążanie oparte na konceptach eliminuje znaczącą ilość \emph{boiler-plate}\footnote{BP} z programowania generycznego i kodu meta programowania (użycia \verb#enable_if# \footnote{EI}).

Funkcja \verb#czyZnaleziono# ocenia czy element znaduje się w sekwencji

\begin{lstlisting}[frame=single]
template<Sequence S, Equality_comparable T>
   requires Same_as<T, value_type_t<S>>
bool czyZnaleziono(const S& seq, const T& value){
   for(const S& seq, const T& value)
      if(x == value)
         return true;
   return false;
}
\end{lstlisting}

Funkcja przyjmuje jako parametr sekwencję i wartość typu \verb#Equality_comparable#. Algorytm ma 3 ograniczenia:

\begin{itemize}

\item typ parametru \verb#seq# musi być typu \verb#Sequence#
\item typ parametru \verb#value# musi być typu \verb#Equality_comparable#
\item typ wartości typu \verb#S# musi być taki sam jak typ elementu zmiennej \verb#seq#

\end{itemize}

Definicje konceptów \verb#Range# i \verb#Sequence# potrzebne do tego algorytmu

\begin{lstlisting}[frame=single]
template<typename R>
concept bool Range() {
   return requires (R range){
      typename value_type_t<R>;
      typename iterator_t<R>;
      { begin(range) } -> iterator_t<R>;
      { end(range) } -> iterator_t<R>;
      requires Input_iterator<iterator_t<R>>();
      requires Same_as<value_type_t<R>,
         value_type_t<iterator_t<R>>>();
   };
};

template<typename S>
concept bool Sequence() {
   return Range<R> && requires (S seq) {
      { seq.front() } -> const value_type<S>&;
      { seq.back() } -> const value_type<S>&;
   };
};

\end{lstlisting}

Specyfikacja wymaga by typ \verb#Range# miał:

\begin{itemize}

\item dwa powiązane typy nazwane \verb#value_type_t# i \verb#iterator_t#
\item dwa poprawne operacje \verb#begin()# i \verb#end()#, które zwracają iteratory
\item typ wartości typu \verb#R# jest taki sam jak typ wartości iteratora tego typu.

\end{itemize}

Wydaje się w porządku. Możemy użyć tego algorytmu, żeby sprawdzić czy element jest w sekwencji. Niestety to nie działa dla wszystkich kolekcji:\newline

\noindent \verb#std::set<int> x { ... };#\newline
\verb#if(czyZnaleziono(x, 42)){#\newline
\verb#// błąd: brak operatora front() lub back()#\newline
\verb#}#\newline

Rozwiązaniem jest dodanie przeciążenia, które przyjmuje kontenery asocjacyjne

\begin{lstlisting}[frame=single]
template<Associative_container A, Same_as<key_type_t<T>> T>
bool czyZnaleziono(const A& a, const T& value){
   return a.find(value) != s.end();
}
\end{lstlisting}

Ta wersja funkcji \verb#czyZnaleziono# ma tylko dwa ograniczenia: typ \verb#A# musi być \verb#Associative_container# i typ \verb#T# musi być taki sam jak typ klucza \verb#A# (\verb#key_type_t<A>#). Dla kontenerów asocjacyjnych, szukamy wartości używając funkcji \verb#find()# a potem sprawdzamy czy się udało przez porównanie z \verb#end()#. W przeciwieństwie do wersji \verb#Sequence#, typ \verb#T# nie musi być \verb#Equality_comparable#. To dlatego, że precyzyjne wymagania typu \verb#T# są ustalone przez kontener asocjacyjny (te wymagania są ustalane przez oddzielny komparator lub funkcję haszującą.

Zdefiniowany koncept \verb#Associative_container#
\begin{lstlisting}[frame=single]
template<typename S>
concept bool Associative_container() {
   return Regular<S> && Range<S>() && requires {
      typename key_type_t<S>;
      requires Object_type<key_type_t<S>>;
   } && requires (S s, key_type_t<S> k){
      { s.empty() } -> bool;
      { s.size() } -> int;
      { s.find(k) } -> iterator_t<S>;
      { s.count(k) } -> int;
   };
};
\end{lstlisting}

\end{document}

\end{document}
	
	\newpage
	
	\documentclass[11pt, a4paper]{article}
\usepackage{polski}
\usepackage[utf8]{inputenc}
\usepackage{listings}
\usepackage{standalone}

\begin{document}
\lstset{language=C++}

\section{Przeciążanie}

\documentclass[11pt, a4paper]{article}
\usepackage{polski}
\usepackage[utf8]{inputenc}
\usepackage{listings}

\linespread{1.3}

\begin{document}
\lstset{language=C++}

%\subsection*{Wprowadzenie}

Koncepty są użyteczne nie tylko w poprawianiu wiadomości błędów i precyzyjnej specyfikacji interfejsów. Zwiększają również ekspresyjność. Używane są do skracania kodu, robieniu go generycznym i zwiększania wydajności. Wyjątkowo potężną cechą jest ich rola w przeciążaniu funkcji. 

W kwietniu 2016 został wydany kompilator \emph{GCC 6.2}. Ta wersja zawierała główne unowocześnienie dwóch komponentów implementacji konceptów. Jeden z nich to generator diagnostyki, który został znacznie odnowiony, aby zapewnić dokładną diagnostykę niepowodzeń konceptu przy sprawdzaniu czy jest spełniony. Drugi to wsparcie dla przeciążania ograniczeń, które zostało całkowicie przepisane, aby zapewnić znaczne zwiększenie wydajności. W \emph{GCC} można teraz używać konceptów do projektów o znacznej wielkości i złożoności.

Niektórzy twierdzą, że wyrażenia takie jak \verb#SFINAE#\footnote{(ang. Substitution failure is not an error) sytuacja w \emph{C++} gdzie nieprawidłowe zastąpienie parametrów szablonu nie jest samo w sobie błędem}, \verb#constexpr if#\footnote{Wyrażenie, którego wartość warunku musi być kontekstowo konwertowanym stałym wyrażeniem typu bool.}, \newline \verb#static_assert#\footnote{Wykonuje sprawdzanie porównania w czasie kompilacji} i mądre techniki metaprogramowania w zupełności wystarczą do przeciążania. To oczywiście poprawne myślenie, lecz jest to obniżanie poziomu abstrakcji, co skutkuje tym, że programuje się w sposób żeby było zrobione a nie jak powinno być. Wynikiem jest więcej pracy dla programisty, zwiększona ilość błędów i mniej szans optymalizacyjnych. \emph{C++} nie jest przeznaczony do metaprogramowania szablonów. Koncepty pomagają nam podnieść poziom programowania i ułatwić kod, bez dodawania kosztów czasu wykonania. \newline

\begin{lstlisting}[frame=single]
template<Sequence S, Equality_comparable T>
  requires Same_as<T, value_type_t<S>>
bool czyIstnieje(const S &seq, const T &value) {
  for (const auto &x : range)
    if (x == value)
      return true;
  return false;
}
\end{lstlisting}

\noindent Funkcja \verb#czyIstnieje# przyjmuje sekwencję typu \verb#Sequence# jako pierwszy argument i wartość \verb#Equality comparable# jako drugi. Algorytm ma trzy ograniczenia:

\begin{itemize}

\item \verb#seq# musi być typu \verb#Sequence#
\item \verb#value# musi być typu \verb#Equality_comparable#
\item typ \verb#value# musi być taki sam jak element typu \verb#seq#

\end{itemize}

\noindent Wyrażenie \verb#value_type_t# to alias typu, który odnosi się do zdeklarowanego lub wydedukowanego typu wartości \verb#R#. Definicje konceptów \verb#Sequence# i \verb#Range# potrzebne do tego algorytmu wyglądają tak: \newline

\begin{lstlisting}[frame=single]
template<typename R>
concept bool Range() {
  return requires (R range) {
    typename value_type_t<R>;
    typename iterator_t<R>;
    { begin(range) } -> iterator_t<R>;
    { end(range) } -> iterator_t<R>;
    requires Input_iterator<iterator_t<R>>();
    requires Same_as<value_type_t<R>,
      value_type_t<iterator_t<R>>>();
  };
}

template<typename S>
concept bool Sequence() {
  return Range<R>() && requires (S seq) {
    { seq.front() } -> const value_type<S>&;
    { seq.back() } -> const value_type<S>&;
  };
}
\end{lstlisting}

Większość sekwencji posiada operacje \verb#front()# i \verb#back()#, które zwracają pierwszy i ostatni element przedziału. To nie jest w pełni rozwinięta specyfikacja sekwencji. Możemy użyć algorytmu do określenia, czy element znajduje się w dowolnej sekwencji. Niestety, algorytm nie działa w przypadku niektórych kolekcji:

\begin{lstlisting}[frame=single]
std::set<int> testSet { ... };
if (czyIstnieje(testSet, 42)) // (1)
  ...
\end{lstlisting}

(1) - błąd: brak operacji \verb#front()# lub \verb#back()#\newline

Potrzebny jest sposób, żeby jasno określić, czy klucz znajduję się w zbiorze.

%\addcontentsline{toc}{subsection}{Wprowadzenie}

\end{document}

\documentclass[11pt, a4paper]{article}
\usepackage{polski}
\usepackage[utf8]{inputenc}
\usepackage{listings}

\begin{document}
\lstset{language=C++}

\subsection{Parametryzacja szablonów}

Parametry szablonu są określane na dwa sposoby:

\begin{enumerate}

\item \emph{parametry szablonu} – wyraźnie wspomniane jako parametry w deklaracji szablonu

\item \emph{nazwy zależne} - wywnioskowane z użycia parametrów w definicji szablonu

\end{enumerate}

W \emph{C++} nazwa nie może być użyta bez wcześniejszej deklaracji. To wymaga od użytkownika ostrożnego traktowania definicji szablonów, np. w definicji funkcji \verb#kwadrat# nie ma widocznej deklaracji symbolu \verb#*#. Jednak, podczas inicjalizacji szablonu \verb#kwadrat<int># kompilator może sprowadzić symbol \verb#*# do (wbudowanego) operatora mnożenia dla wartości \verb#int#. Dla wywołania \verb#kwadrat(zespolona(2.0))#, operator * zostałby rozwiązany do (zdefiniowanego przez użytkownika) operatora mnożenia dla wartości \verb#zespolona#. Symbol \verb#*# jest więc \emph{nazwą zależną} w definicji funkcji \verb#kwadrat#. Oznacza to, że jest to ukryty parametr definicji szablonu. Możemy uczynić z operacji mnożenia formalny parametr:

\begin{lstlisting}[frame=single]
template<typename Multiply, typename T>
T square(T x) {
   return Multiply() (x,x);
}
\end{lstlisting}

Pod-wyrażenie \verb#Multiply()# tworzy obiekt funkcji, który wprowadza operację mnożenia wartości typu \verb#T#. Pojęcie \emph{nazw zależnych} pomaga utrzymać liczbę jawnych argumentów.

\end{document}

\documentclass[11pt, a4paper]{article}
\usepackage{polski}
\usepackage[utf8]{inputenc}
\usepackage{listings}

\begin{document}
\lstset{language=C++}

\subsection{Inicjalizacje i sprawdzanie}

Minimalne przetwarzanie semantyczne odbywa się, gdy po raz pierwszy pojawia się definicję szablonu lub jego użycie. Pełne przetwarzanie semantyczne jest przesuwane na czas inicjalizacji (tuż przed czasem linkowania), na podstawie każdej instancji. Oznacza to, że założenia dotyczące argumentów szablonu nie są sprawdzane przed czasem inicjalizacji. Np.\newline

\noindent \verb#string x = "testowy tekst";# \newline
\verb#kwadrat(x);# \newline

Bezsensowne użycie zmiennej \verb#string# jako argumentu funkcji \verb#kwadrat# nie jest wyłapane w momencie użycia. Dopiero w czasie inicjalizacji kompilator odkryje, że nie ma odpowiedniej deklaracji dla operatora *. To ogromny praktyczny błąd, bo inicjalizacja może być przeprowadzona przez kod napisany przez użytkownika, który nie napisał definicji funkcji \verb#kwadrat# ani definicji \verb#string#. Programista, który nie znał definicji funkcji \verb#kwadrat# ani \verb#string# miałby ogromne trudności w zrozumieniu komunikatów błędów związanych z ich interakcją (np. ”illegal operand for *”).

Istnienie symbolu operatora * nie jest wystarczające by zapewnić pomyślną kompilację funkcji \verb#kwadrat#. Musi istnieć operator *, który przyjmuje argumenty odpowiednich typów i ten operator * musi być bezkonkurencyjnym dopasowaniem według zasad przeciążania \emph{C++}. Dodatkowo funkcja \verb#kwadrat# przyjmuje argumenty przez wartość i zwraca swój wynik przez wartość. Z tego wynika, że musi być możliwe skopiowanie obiektów dedukowanego typu. Potrzebny jest rygorystyczny framework do opisywania wymagań definicji szablonów na ich argumentach.

Doświadczenie podpowiada nam, że pomyślna kompilacja i linkowanie może nie gwarantować końca problemów. Udana budowa pokazuje tylko, że inicjalizacje szablonów były poprawne pod względem typów, dostając argumenty które przekazaliśmy. Co z typami argumentów szablonu i wartościami, z którymi nie próbowaliśmy użyć naszych szablonów? Definicja szablonu może zawierać przypuszczenia na temat argumentów, które przekazaliśmy ale nie zadziała dla innych, prawdopodobnie rozsądnych argumentów. Uproszczona wersja klasycznego przykładu:

\begin{lstlisting}[frame=single]
template<typename FwdIter>
bool czyJestPalindromem(FwdIter first, FwdIter last){
   if(last <= first) return true;
   if(*first != *last) return false;
   return czyJestPalindromem(++first, --last);
}

\end{lstlisting}

Testujemy czy sekwencja wyznaczona przez parę iteratorów do jego pierwszego i ostatniego elementu, jest palindromem. Przyjmuje się, że te iteratory są z kategorii \emph{forward iterator}. To znaczy, że powinny wspierać co najmniej operacje takie jak: *, != i ++. Definicja funkcji \verb#czyJestPalindromem# bada czy elementy sekwencji zmierzają z początku i końca do środka. Możemy przetestować tę funkcję używając \verb#vector#, tablicy w stylu \verb#C# i \verb#string#. W każdym przypadku nasz szablon funkcji zainicjalizuje się i wykona się poprawnie. Niestety, umieszczenie tej funkcji w bibliotece byłoby dużym błędem. Nie wszystkie sekwencje wspierają operatory \verb#--# i $\leq$. Np. listy pojedyncze nie wspierają. Eksperci używają wyszukanych, regularnych technik by uniknąć takich problemów.  Jednakże, fundamentalny problem jest taki, że definicja szablonu nie jest (według siebie) dobrą specyfikacją wymagań na swoje parametry.

\end{document}

\documentclass[11pt, a4paper]{article}
\usepackage{polski}
\usepackage[utf8]{inputenc}
\usepackage{listings}

\begin{document}
\lstset{language=C++}

\subsection{Wydajność}

Szablony grają kluczową rolę w programowaniu w \emph{C++} dla wydajnych aplikacji. Ta wydajność ma trzy źródła:

\begin{itemize}

\item eliminacja wywołań funkcji na korzyść \emph{inliningu}\footnote{Optymalizacja kompilatora, która zamienia wywołanie funkcji na jej ciało w czasie kompilacji.}
\item łączenie informacji z różnych kontekstów w celu lepszej optymalizacji
\item unikanie generowania kodu dla niewykorzystanych funkcji

\end{itemize}

Pierwszy punkt nie odnosi się tylko do szablonów ale ogólnie do cech funkcji \emph{inline} w \emph{C++}. 
Wydajność ta przekłada się zarówno na czas wykonania jak i pamięć. Szablony mogą równocześnie zmniejszyć obie wydajności. Zmniejszenie rozmiaru kodu jest szczególnie ważne, ponieważ w przypadku nowoczesnych procesorów zmniejszenie rozmiaru kodu pociąga za sobą zmniejszenie ruchu w pamięci i poprawienie wydajności pamięci podręcznej.

%Jakkolwiek, \emph{inlining} jest istotny dla drobno-granularnej parametryzacji, którą powszechnie stosuje się w bibliotece \emph{STL} i innych bibliotekach  bazujących na generycznych technikach programowania. 

\begin{lstlisting}[frame=single]
template<typename FwdIter, typename T>
T suma(FwdIter first, FwdIter last, T init){
   for(FwdIter cur = first, cur != last, T init)
      init = init + *cur;
   return init;
}

\end{lstlisting}

Funkcja \verb#suma# zwraca sumę elementów jej sekwencji wejściowej używając trzeciego argumentu ("akumulatora") jako wartości początkowej\newline

\noindent \verb#vector<zespolona<double>> v;#  \newline
\verb#zespolona<double> z = 0;# \newline
\verb#z = suma(v.begin(), v.end(), z);# \newline

By wykonać swoją pracę, \verb#suma# użyje operatorów dodawania i przypisania na elementach typu \verb#zespolona<double># i dereferencji iteratorów \verb#vector<zespolona<double>>#.  Dodanie wartości typu \verb#zespolona<double># pociąga za sobą dodanie wartości typu \verb#double#. By zrobić to wydajnie wszystkie te operacje muszą być \emph{inline}.  Zarówno \verb#vector# jak i \verb#zespolona# są typami zdefiniowanymi przez użytkownika. Oznacza to, że typy te jak i ich operacje są zdefiniowane gdzie indziej w kodzie źródłowym \emph{C++}. Obecne kompilatory \emph{C++} radzą sobie z tym przykładem, dzięki czemu jedyne wygenerowane wywołanie to wywołanie funkcji \verb#suma#. Dostęp do pól zmiennej \verb#vector# staje się prostą operacją maszyny ładującej, dodawanie wartości typu \verb#zespolona# staje się dwiema instrukcjami maszyny dodającej dwa elementy zmiennoprzecinkowe. Aby to osiągnąć, kompilator potrzebuje dostępu do pełnej definicji \verb#vector# i \verb#zespolona#. Jednak wynik jest ogromną poprawą (prawdopodobnie optymalną) w stosunku do naiwnego podejścia generowania wywołania funkcji dla każdego użycia operacji na parametrze szablonu. Oczywiście instrukcja dodawania wykonuje się znacznie szybciej niż wywołanie funkcji zawierającej dodawanie. Poza tym, nie ma żadnego wstępu wywołania funkcji, przekazywania argumentów itd., więc kod wynikowy jest również wiele mniejszy. Dalsze zmniejszanie rozmiaru generowanego kodu uzyskuje się nie wysyłając kodu niewykorzystywanych funkcji. Klasa szablonu \verb#vector# ma wiele funkcji, które nie są wykorzystywane w tym przykładzie. Podobnie szablon klasy \verb#zespolona# ma wiele funkcji i funkcji nieskładowych (nienależących do funkcji klasy). Standard \emph{C++} gwarantuje, że nie jest emitowany żaden kod dla tych niewykorzystanych funkcji. 

Inaczej sprawa wygląda, gdy argumenty są dostępne za pośrednictwem interfejsów zdefiniowanych jako wywołania funkcji pośrednich. Każda operacja staje się wtedy wywołaniem funkcji w pliku wykonywalnym generowanym dla kodu użytkownika, takiego jak \verb#suma#. Co więcej, byłoby wyraźnie nietypowe unikać odkładania kodu nieużywanych (wirtualnych) funkcji składowych. Jest to poza zdolnością obecnych kompilatorów \emph{C++} i prawdopodobnie pozostanie takie dla głównych programów \emph{C++}, gdzie oddzielna kompilacja i łączenie dynamiczne jest normą. Ten problem nie jest wyjątkowy dla \emph{C++}. Opiera się on na podstawowej trudności w ocenieniu, która część kodu źródłowego jest używana, a która nie, gdy jakakolwiek forma procesu \emph{run-time dispatch}\footnote{Zwany również \emph{dynamic dispatch} proces wybierania, implementacji polimorficznej operacji (metody lub funkcji) do wywołania w czasie uruchomienia.} ma miejsce. Szablony nie cierpią na ten problem bo ich specjalizacje są rozwiązywane w czasie kompilacji.\newline

\noindent \verb#vector<int> v;#  \newline
\verb#zespolona<double> s = 0;#  \newline
\verb#s = suma(v.begin(), v.end(), s);#  \newline

W powyższej funkcji dodawanie wykonywane jest przez konwertowanie wartości \verb#int# do wartości \verb#double# i potem dodawanie tego do akumulatora \verb#s#, używając operatora \verb#+# typu \verb#zespolona<double># i \verb#double#. To podstawowe dodawanie zmiennoprzecinkowe. Kwestia jest taka, że operator \verb#+# w funkcji \verb#suma# zależy od dwóch parametrów szablonu i leży to w kwestii kompilatora by wybrać bardziej odpowiedni operator \verb#+# bazując na informacji o tych dwóch argumentach. Byłoby możliwe utrzymanie lepszego rozdzielenia między różnymi kontekstami przez przekształcanie typu elementu w typ akumulatora. W takim przypadku spowodowałoby to powstanie dodatkowego \verb#zespolona<double># dla każdego elementu i dodania dwóch wartości typu \verb#zespolona#. Rozmiar kodu i czas wykonywania byłyby większe niż dwukrotnie.

Duże ilości prawdziwego oprogramowania zależą od optymalizacji. W konsekwencji udoskonalone sprawdzanie typu, co zostało obiecane przy użyciu konceptów, nie może kosztować tych optymalizacji.

\end{document}

\end{document}
	
	\newpage
	
	\documentclass[11pt, a4paper]{article}
\usepackage{polski}
\usepackage[utf8]{inputenc}
\usepackage{listings}
\usepackage{amsthm}
\usepackage[]{algorithm2e}
\renewcommand{\algorithmcfname}{Algorytm}

\begin{document}
\lstset{language=C++}

\section{Implementacja algorytmu Fleury'ego jako przykład wykorzystujący koncepty}

\subsection{Omówienie problemu}
\emph{Algorytm Fleury'ego} to algorytm pozwalający na znalezienie \emph{cyklu Eulera} w \emph{grafie eulerowskim.}

\newtheorem{mydef}{Definicja}

\begin{mydef}
\textbf{Graf} - struktura służąca do przedstawiania i badania relacji między obiektami. Jest to zbiór wierzchołków, które mogą być połączone krawędziami, gdzie krawędź zaczyna się i kończy w którymś z wierzchołków.
\end{mydef}

\begin{mydef}
\textbf{Multigraf} - graf, w którym mogą występować krawędzie wielokrotne(powtarzające się) oraz pętle (krawędzie, których końcami jest ten sam wierzchołek).
\end{mydef}

\begin{mydef}
\textbf{Graf spójny} - graf spełniający warunek, że dla każdej pary wierzchołków istnieje ścieżka, która je łączy.
\end{mydef}

\begin{mydef}
\textbf{Ścieżka} - ciąg wierzchołków, połączonych krawędziami.
\end{mydef}

\begin{mydef}
\textbf{Droga} - ścieżka, w której wierzchołki są różne.
\end{mydef}

\begin{mydef}
\textbf{Cykl} - droga zamknięta czyli ścieżka, w której pierwszy i ostatni wierzchołek są równe.
\end{mydef}

\begin{mydef}
\textbf{Graf eulerowski} - spójny multigraf posiadający cykl, który zawiera wszystkie krawędzie. 
\end{mydef}

\begin{mydef}
\textbf{Warunek istnienia cyklu Eulera w spójnym multigrafie} - stopień każdego wierzchołka musi być liczbą parzystą.
\end{mydef}

\begin{algorithm}[H]
 \KwData{G = (V,E), G - spójny multigraf, V - zbiór wierzchołków, E - zbiór krawędzi}\label{Wejście}
 \KwResult{zbiór wierzchołków reprezentujących cykl Eulera}
 Zaczynamy od dowolnego wierzchołka ze zbioru V\;
 \While{Dopóki zbiór krawędzi nie jest pusty}{
  \eIf{Jeżeli z bieżącego wierzchołka x odchodzi tylko jedna krawędź}{
   to przechodzimy wzdłuż tej krawędzi do następnego wierzchołka i usuwamy tę krawędź wraz z wierzchołkiem x\;
   }{
  wybieramy tę krawędź, której usunięcie nie rozspójnia grafu i przechodzimy wzdłuż tej krawędzi do następnego wierzchołka, a następnie usuwamy tę krawędź z grafu\;
  }
 }
 \caption{Algorytm Fleury'ego}
\end{algorithm}

\subsection{Działanie programu}
Założeniem programu jest symulacja algorytmu Fleury'ego dla jak największej ilości kontenerów biblioteki STL. Dzięki przeciążaniu funkcji jakie oferują koncepty, w prosty i czytelny sposób, udało się napisać generyczny algorytm.

W programie są dwa kontenery do przechowywania krawędzi i wierzchołków. Krawędź reprezentowana jest przez klasę \verb#Edge#, która przyjmuje dwie wartości typu \verb#int# do konstruktora. Wierzchołek, z kolei reprezentowany jest przez zmienną \verb#int#. 

Dane (pary wierzchołków) są wczytywane z pliku, a potem w zależności od rodzaju kontenera i iteratora, sortowane. Dla kontenerów: 
\begin{itemize}
\item sekwencyjnych z iteratorem \verb#Random access# (\verb#vector#) wywoływana jest funkcja szablonu ograniczonego przez koncepty: \verb#Sequence# i \newline \verb#Random_access_iterator#.

\begin{lstlisting}[frame=single]
template<Sequence S, Random_access_iterator R>
void sortVertices(S &seq){
    sort(seq.begin(), seq.end());
}
\end{lstlisting}

\item sekwencyjnych z iteratorem \verb#Bidirectional# (\verb#list#) wywoływana jest funkcja szablonu ograniczonego przez koncepty: \verb#Sequence# i \newline \verb#Bidirectional_iterator#.

\begin{lstlisting}[frame=single]
template<Sequence S, Bidirectional_iterator R>
void sortVertices(S &seq){
    seq.sort();
}
\end{lstlisting}

\item asocjacyjnych (\verb#set#) wywoływana jest funkcja szablonu ograniczonego przez koncept: \verb#Associative_container #.

\begin{lstlisting}[frame=single]
template<Associative_container A>
void sortVertices(A &seq){}
\end{lstlisting}

\end{itemize}
\newpage
Omawiany algorytm wykonuje funkcja \verb#determineEulerCycle#:
\begin{lstlisting}[frame=single]
template<typename E, typename V>
void determineEulerCycle(E &edges, V &vertices){
   int v = 0;
   bool condition = (checkIfGraphConnected(edges, 
   vertices, 0, v) && checkIfAllEdgesEvenDegree
   (edges, vertices));
    
   if(condition){
      cout << "Euler cycle:" << endl << endl << v;
      while(!edges.empty()){
         switch(getNeighboursCount(edges, v)){
            case 1 : {
               removeEdgeWithOneNeighbour(edges, v);
               break;
            }
            default: {
               removeEdgeWithMoreNeighbour(edges,
               v, vertices);
               break;
            }
         }
         cout <<" -> "<<v;
      }
      cout << endl << endl;
    } else {
       cout<<"Invalid graph."<<endl;
       if(!checkIfGraphConnected(edges, vertices,
       0, v))
          cout <<"Graph is not connected"<<endl;
       else if(!checkIfAllEdgesEvenDegree(edges,
       vertices))
          cout <<"Not all the edges are even"<<endl;
    }
}
\end{lstlisting}

Żeby algorytm się wykonał, muszą zostać spełnione dwa warunki: graf musi być spójny(za to odpowiedzialna jest funkcja \verb#checkIfGraphConnected#) i wszystkie krawędzie muszą być parzystego stopnia (\verb#checkIfAllEdgesEvenDegree#).

\verb#checkIfGraphConnected()#
\begin{lstlisting}[frame=single]
template<typename E, typename V>
bool checkIfGraphConnected(E &ed, V &vertices,
int x, int startVertice) {
    
   bool *visited = new bool[vertices.size()];
   for (int i = 0; i < vertices.size(); i++) 
      visited[i] = false;

   stack<int>stack;
   int vc = 0;
    
   stack.push(startVertice);
   visited[startVertice] = true;
    
   while (!stack.empty()) {
       int v = stack.top();
       stack.pop();
       vc++;
    
       for(typename E::iterator it = ed.begin();
          it != ed.end(); it++){
          if(it->getA() == v && !visited[it->getB()]){
             visited[it->getB()] = true;
             stack.push(it->getB());
          } else if(it->getB() == v &&
          !visited[it->getA()]){
             visited[it->getA()] = true;
             stack.push(it->getA());
          }
        }
    }

    delete [] visited;

    return (vc == vertices.size()-x);

}
\end{lstlisting}

Algorytm przechodzi przez graf, po kolei wrzucając odwiedzane wierzchołki na stos, zaznaczając je w tablicy odwiedzonych (\verb#visited#) i zaraz zdejmuje z tego stosu, zwiększając licznik \verb#vc#. Robi to dopóki stos nie jest pusty. Zwraca warunek porównujący licznik \verb#vc# z rozmiarem kontenera wierzchołków (wszystkie wierzchołki zostały odwiedzone, czyli istnieją ścieżki między wierzchołkami, graf jest spójny).

\verb#checkIfAllEdgesEvenDegree#:
\begin{lstlisting}[frame=single]
template<typename E, typename V>
bool checkIfAllEdgesEvenDegree(E &edges, V &vertices){
    int counter = 0, i = 0;
    for(auto v : vertices){
        for(auto e : edges){
            if(e.getA() == v || e.getB() == v)
               counter++;
        }
        if(counter % 2 == 0) i++;
    }
    return (i == vertices.size()) ? true : false;
}
\end{lstlisting}

Zmienna \verb#i#  zwiększa się jeśli ilość wystąpień wierzchołka jest liczbą parzystą. Zwraca wartość \verb#true# jeśli zmienna \verb#i# jest równa liczbie elementów kontenera zawierającego wierzchołki(dla każdego wierzchołka zmienna \verb#i# zwiększała się o 1).

Jeśli warunek nie zostanie spełniony, użytkownik zostaje poinformowany o tym, że graf jest niepoprawny. W odwrotnej sytuacji, w pętli (dopóki kontener krawędzi nie jest pusty), wykonywana jest jedna dwóch operacji. Gdy wierzchołek ma jednego sąsiada, wywołuje się funkcja \verb#removeEdgeWithOneN-#\newline \verb#eighbour()#, a gdy więcej wierzchołków, funkcja \verb#removeEdgeWithMoreNei-# \newline \verb#ghbour()#. Pierwsza z nich ma dwa przeciążenia konceptowe:
\begin{itemize}

\item Dla kontenera sekwencyjnego:
\begin{lstlisting}[frame=single]
template<Sequence S>
void removeEdgeWithOneNeighbour(S &edges, int &v){
    
    typename S::iterator it = find(edges.begin(), 
    edges.end(), v);
    
    if(it->getA() == v) v = it->getB();
    else v = it->getA();

    edges.erase(it);
}
\end{lstlisting}

\item Dla kontenera asocjacyjnego:
\begin{lstlisting}[frame=single]
template<Associative_container A>
void removeEdgeWithOneNeighbour(A &edges, int &v){
    
    typename A::iterator it2;
    for(typename A::iterator it = edges.begin();
    it != edges.end(); it++){
        if(it->getA() == v || it->getB() == v){
            it2 = it;
            if(it->getA() == v) v = it->getB();
            else v = it->getA();
            it = prev(edges.end());
        }
    }

    edges.erase(it2);
}
\end{lstlisting}

Funkcja znajduje krawędź, dostając wierzchołek wychodzący. I wierzchołek znalezionej krawędzi przypisuje do tego przekazanego.

\end{itemize}

Druga \verb#removeEdgeWithMoreNeighbour()# wygląda:
\begin{lstlisting}[frame=single]
template<typename E, typename V>
void removeEdgeWithMoreNeighbour(E &edges, int &v,
V &vertices){
    for(typename E::iterator i = edges.begin(); 
    i != edges.end(); i++){
        if(i->getA() == v && 
        checkIfStillConnected(edges, *i, 
        getZeroDegreeCount(edges, vertices), v, 
        vertices)){
            v = i->getB();
            edges.erase(i);
            i = prev(edges.end());
        } else if(i->getB() == v && 
        checkIfStillConnected(edges, *i, 
        getZeroDegreeCount(edges, vertices), v, 
        vertices)){
            v = i->getA();
            edges.erase(i);
            i = prev(edges.end());
        }
    }
}
\end{lstlisting}
\newpage
Jeśli wierzchołek ma więcej sąsiadów, wybiera tego który nie rozspójni grafu. Żeby to sprawdzić używa funkcji \verb#checkIfStillConnected#:
\begin{lstlisting}[frame=single]
template<Associative_container E, typename V>
bool checkIfStillConnected(E &edges, Edge e, int x, 
int startVertice, V &vertices){
    
   E tmp;

   for(auto e : edges)
      tmp.insert(e);

   for(typename E::iterator it = tmp.begin(); 
   it != tmp.end(); it++)
      if (it->getA() == e.getA() && 
         it->getB() == e.getB()) {
         tmp.erase(it);
         it = prev(tmp.end());
      }
        
   return checkIfGraphConnected(tmp, vertices, 
   x, startVertice);
}
\end{lstlisting}

W celu sprawdzenia, czy graf po usunięciu jakiejś krawędzi dalej będzie spójny, potrzebny jest pomocniczy kontener. Zapisujemy do niego aktualne krawędzie, wyszukujemy w nim przekazaną i przekazujemy go do istniejącej już funkcji \verb#checkIfGraphConnected()#.

\end{document}
	
	\newpage
	
	%\newpage

	%\section{Rozdział 5}
	%Zawartość rozdziału 5
	
	%\newpage
	
	\documentclass[11pt, a4paper]{article}
\usepackage{polski}
\usepackage[utf8]{inputenc}
\usepackage{listings}
\usepackage{standalone}

\begin{document}
\lstset{language=C++}

\section{Włączenie konceptów do standardu C++}

Koncepty nie zostały włączone do standardu \emph{C++17}. Krótkie uzasadnienie jest takie, że komisja nie osiągnęła porozumienia, że koncepty (określone w specyfikacji technicznej) osiągnęły wystarczające doświadczenie w zakresie wdrożenia i użytkowania, aby być wystarczające do dopuszczenia w obecnym projekcie. Zasadniczo komisja nie powiedziała "nie", ale "jeszcze nie".

Największe zastrzeżenia nie wynikały z problemów technicznych. Powstały następujące obawy:

\begin{itemize}

\item specyfikacja konceptów istniała w opublikowanej formie przez mniej niż cztery miesiące
\item jedyna znana i dostępna publicznie implementacja konceptów znajduje się w nieopublikowanej wersji \emph{kompilatora GCC}
\item implementacja \emph{kompilatora GCC} została opracowana przez tę samą osobę, która napisała specyfikację. W związku z tym implementacja jest dostępna do testowania, ale nie podjęto żadnej próby wprowadzenia w życia specyfikacji. A zatem specyfikacja nie jest przetestowana. Kilku członków komisji wskazało, że posiadanie implementacji wyprodukowanej ze specyfikacji ma decydujące znaczenie dla określenia kwestii specyfikacji.
\item najbardziej znaczące i znane użycie konceptów jest dostępne w specyfikacji \emph{Ranges TS}. Jest kilka innych projektów eksperymentujących z konceptami, ale żaden z nich nie zbliża się do skali, której można oczekiwać gdy programiści zaczną korzystać z tej funkcjonalności. Wydajność i problemy związane z obsługą błędów przy użyciu bieżącej implementacji GCC dowodzą, że nie wykonano większej próby używania konceptów.
\item specyfikacja konceptów nie dostarcza żadnych definicji. Niektórzy członkowie komisji kwestionują użyteczność konceptów bez dostępności biblioteki definicji konceptów, takiej jak \emph{Ranges TS}. Przyjęcie specyfikacji konceptów do \emph{C++17} bez odpowiedniej biblioteki definicji niesie ryzyko zablokowania języka bez udowodnienia, że zawiera funkcje potrzebne do wdrożenia biblioteki, które mogłyby być zaprojektowane do konceptualizacji biblioteki standardowej.

\end{itemize}

Obawy techniczne:
\begin{itemize}

\item koncepty zawierają nową składnię do definiowania szablonów funkcji. Skrócona deklaracja szablonu funkcji wygląda podobnie to nieszablonowej deklaracji funkcji z wyjątkiem tego, że co najmniej jeden z jej  parametrów zostanie zadeklarowany ze specyfikatorem typu zastępczego \verb#auto# albo nazwą konceptu. Obawa wynika z tego, że taka deklaracja:\newline
\noindent \verb#void f(X x){}# \newline
definiuje nieszablonową funkcję jeśli \verb#X# jest typem, ale definiuje szablon funkcji jeśli \verb#X# jest konceptem. To ma subtelne konsekwencje dla tego czy funkcja może być zdefiniowana w pliku nagłówkowym, czy słowo kluczowe \verb#typename# jest potrzebne by odnieść się do składowych typów typu \verb#X#, czy istnieje dokładnie jedna zmienna lub brak lub kilka dla każdej deklarowanej zmiennej lokalnej, statycznej. itd.


\item specyfikacja konceptów zawiera również składnię szablonów wstępnych, która pozwala ominąć rozwlekłą składnię deklaracji szablonu, do której wszyscy są przyzwyczajeni jednocześnie określając ograniczenia typu. Następujący przykład deklaruje szablon funkcji \verb#f#, przyjmujący dwa parametry \verb#A# i \verb#B#, które spełniają wymagania konceptu \verb#C#: \newline
\verb#C{ A, B } void f(A a, B b);#\newline
Ta składnia nie jest lubiana. Wspomniano, że biblioteka \emph{Ranges TS} jej używała w pewnych miejscach a grupa pracująca nad ewolucją biblioteki zażądała żeby ją zmienić i już nigdy nie używać.
\item Są dwie formy definiowania konceptów: funkcja i zmienna. Forma funkcji istnieje po to by wspierać przeciążanie definicji konceptów oparte na parametrach szablonu. Forma zmiennej istnieje by wspierać nieco krótsze definicje:

\begin{lstlisting}[frame=single]
//forma funkcji
template<typename T>
concept bool C(){
   return ...;
}

//forma zmiennej
template<typename T>
concept bool C = ...;
\end{lstlisting}

Wszystkie koncepty, które można zdefiniować przy użyciu formy zmiennej można zdefiniować za pomocą formy funkcji. Stosowana forma wpływa na składnię wymaganą do oszacowania konceptu, a zatem użycie konceptu wymaga znajomości formy użytej do jego zdefiniowania. Wczesna wersja \emph{Ranges TS} używała zarówno formy zmiennej, jak i funkcji do definiowania konceptów i niespójność spowodowała wiele błędów w specyfikacji. Aktualna \emph{Ranges TS} wykorzystuje tylko formę funkcji do zdefiniowania określonych konceptów. Niektórzy członkowie komitetu uważają, że jedna forma definicji uprości język i uniknie trudności w używaniu i nauczaniu. Zapewnienie odrębnej składni definicji konceptów, a nie określenie ich w kategoriach funkcji lub zmiennych uniknęłoby również dziwnej składni \verb#concept bool#.

\item została dodana możliwość używania \verb#auto# jako specyfikatora dla parametrów szablonu bez typu: \newline\newline
\verb#template<auto V>#\newline
\verb#constexpr auto v = V*2;#\newline

Z konceptami można by ograniczyć powyższy szablon tak, że typ \verb#V# spełniałby wymagania konceptu \verb#Integral#:\newline
\verb#template<Integral V>#\newline
\verb#constexpr auto v = V*2;#\newline

Jednak to jest ta sama składnia aktualnie używana przez \emph{Concepts TS}, do deklarowania parametrów typu szablonu ograniczonego. Jeśli \emph{Concepts TS} miały być wprowadzone, potrzebna by była inna składnia aby deklarować ograniczony parametr szablonu bez typu. Prawdopodobnie składnia stosowana przez \emph{Concepts TS} bardziej nadaje się do deklarowania parametrów szablonów bez typu, jak pokazano powyżej, ponieważ pasuje do składni stosowanej dla innych deklaracji zmiennych. To oznacza, że nowa składnia do deklarowania ograniczonych parametrów typu byłaby pożądana ze względu na spójność języka.

\item Koncepty były powszechnie oczekiwane w celu uzyskania lepszych komunikatów o błędach niż obecnie są generowane, gdy pojawiają się niepowodzenie podczas tworzenia szablonów. Teoria idzie, ponieważ koncepty pozwalają odrzucić kod oparty na ograniczeniu w punkcie użycia szablonu, kompilator może po prostu zgłosić błąd ograniczenia, a nie błąd w niektórych wyrażeniach w potencjalnie głęboko zagnieżdżonym stosie instancji szablonu. Niestety okazuje się, że nie jest tak proste, a używanie konceptów skutkuje gorszymi komunikatami o błędach. Niepowodzenia ograniczeń często pojawiają się jako błędy w przeciążeniu, co powoduje potencjalnie długą listę kandydatów, z których każda ma własną listę przyczyn odrzucenia. Identyfikacja kandydata, który był przeznaczony do danego użycia, a następnie określenie, dlaczego wystąpiło niepowodzenia ograniczeń, może być gorszym doświadczeniem niż nawigowanie w stosie tworzenia instancji szablonów.

\item Wielu członków komisji wyraża zaniepokojenie faktem, czy obecny projekt konceptów wystarcza jako podstawa, na której można w przyszłości wdrożyć sprawdzenie pełnej definicji szablonu. Mimo zapewnień obrońców konceptów, że takie kontrole będą możliwe, wiele pytań pozostaje bez odpowiedzi, a członkowie komitetu pozostają bez przekonania. Wydaje się mało prawdopodobne, że obawy te zostaną rozwiązane w inny sposób niż poprzez wdrożenie sprawdzania definicji.

\end{itemize}

Wielu wierzy, że koncepty w jakiejś formie zostaną dodane do \emph{C++19/20}.

\end{document}
	
	\newpage
	
	\section{Bibliografia}
	\begin{thebibliography}{6}
	
	\bibitem{first} Gabriel Dos Reis, \emph{Generic Programming in C++: The Next Level.}, ACCU, 2002.
	\bibitem{second} Bjarne Stroustrup, \emph{The Design and Evolution of C++}, AddisonWesley, 1994
	\bibitem{third}  Bjarne Stroustrup, \emph{Expressing the standard library requirements asconcepts}	
	\bibitem{fourth} Gabriel Dos Reis, Bjarne Stroustrup, \href{http://www.stroustrup.com/popl06.pdf}{\emph{Specifying C++ Concepts}}
	\bibitem{fifth} J. C. Dehnert, A. Stepanov, \emph{Fundamentals of Generic Programming}, Dagstuhl Seminar on Generic Programming.1998. Springer LNCS.

	\bibitem{sixth} A. Stepanov, Daniel E.Rose, \emph{From Mathematics to Generic Programming}
	\bibitem{seventh} D. Gregor, J. Jarvi, J. Siek, B. Stroustrup, G. Dos Reis, A. Lumsdaine, \emph{Concepts: Linguistic Support for Generic Programming in C++}, OOPSLA’06.
	\bibitem {eighth} Scott Meyers, \emph{Effective Modern C++}, O'REILLY 2015
	\bibitem {ninth} Andrew Sutton, \href{https://accu.org/index.php/journals/2198}{Defining concepts}
	\bibitem {tenth} Kevin Chen, \href{https://kevinchen.co/blog/how-to-get-started-cpp-concepts/}{\emph{How to get started with C++ Concepts}}
	\bibitem {eleventh} Bjarne Stroustrup, \href{http://www.stroustrup.com/good_concepts.pdf}{\emph{Concepts: The Future of Generic Programming}}
	\bibitem {twelfth} Tom Honermann, \href{http://honermann.net/blog/category/c-concepts/}{\emph{Refinig concepts}}
	\end{thebibliography}
	\newpage
	\begin{center}
	OŚWIADCZENIE	
	\end{center}
	
	\emph{Ja, niżej podpisany oświadczam, że przedłożona praca dyplomowa została wykonana przeze mnie samodzielnie, nie narusza praw autorskich, interesów prawnych i materialnych innych osób}\newline\newline
	
	\begin{minipage}[t]{7cm}
	\flushleft
	\noindent \textsc{...........................}

	data
	\end{minipage}
	\hfill
	\begin{minipage}[t]{7cm}
	\flushright
	\textsc{.............................}

	podpis
	\end{minipage}


\end{document}