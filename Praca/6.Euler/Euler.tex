\documentclass[11pt, a4paper]{article}
\usepackage{polski}
\usepackage[utf8]{inputenc}
\usepackage{listings}
\usepackage{amsthm}
\usepackage[]{algorithm2e}
\renewcommand{\algorithmcfname}{Algorytm}

\begin{document}
\lstset{language=C++}

\section{Implementacja algorytmu Fleury'ego jako przykład wykorzystujący koncepty}

\subsection{Omówienie problemu}
\emph{Algorytm Fleury'ego} to algorytm pozwalający na znalezienie \emph{cyklu Eulera} w \emph{grafie eulerowskim.}

\newtheorem{mydef}{Definicja}

\begin{mydef}
\textbf{Graf} - struktura służąca do przedstawiania i badania relacji między obiektami. Jest to zbiór wierzchołków, które mogą być połączone krawędziami, gdzie krawędź zaczyna się i kończy w którymś z wierzchołków.
\end{mydef}

\begin{mydef}
\textbf{Multigraf} - graf, w którym mogą występować krawędzie wielokrotne(powtarzające się) oraz pętle (krawędzie, których końcami jest ten sam wierzchołek).
\end{mydef}

\begin{mydef}
\textbf{Graf spójny} - graf spełniający warunek, że dla każdej pary wierzchołków istnieje ścieżka, która je łączy.
\end{mydef}

\begin{mydef}
\textbf{Ścieżka} - ciąg wierzchołków, połączonych krawędziami.
\end{mydef}

\begin{mydef}
\textbf{Droga} - ścieżka, w której wierzchołki są różne.
\end{mydef}

\begin{mydef}
\textbf{Cykl} - droga zamknięta czyli ścieżka, w której pierwszy i ostatni wierzchołek są równe.
\end{mydef}

\begin{mydef}
\textbf{Graf eulerowski} - spójny multigraf posiadający cykl, który zawiera wszystkie krawędzie. 
\end{mydef}

\begin{mydef}
\textbf{Warunek istnienia cyklu Eulera w spójnym multigrafie} - stopień każdego wierzchołka musi być liczbą parzystą.
\end{mydef}

\begin{algorithm}[H]
 \KwData{G = (V,E), G - spójny multigraf, V - zbiór wierzchołków, E - zbiór krawędzi}\label{Wejście}
 \KwResult{zbiór wierzchołków reprezentujących cykl Eulera}
 Zaczynamy od dowolnego wierzchołka ze zbioru V\;
 \While{Dopóki zbiór krawędzi nie jest pusty}{
  \eIf{Jeżeli z bieżącego wierzchołka x odchodzi tylko jedna krawędź}{
   to przechodzimy wzdłuż tej krawędzi do następnego wierzchołka i usuwamy tę krawędź wraz z wierzchołkiem x\;
   }{
  wybieramy tę krawędź, której usunięcie nie rozspójnia grafu i przechodzimy wzdłuż tej krawędzi do następnego wierzchołka, a następnie usuwamy tę krawędź z grafu\;
  }
 }
 \caption{Algorytm Fleury'ego}
\end{algorithm}

\subsection{Działanie programu}
Założeniem programu jest symulacja algorytmu Fleury'ego dla jak największej ilości kontenerów biblioteki STL. Dzięki przeciążaniu funkcji jakie oferują koncepty, w prosty i czytelny sposób, udało się napisać generyczny algorytm.

W programie są dwa kontenery do przechowywania krawędzi i wierzchołków. Krawędź reprezentowana jest przez klasę \verb#Edge#, która przyjmuje dwie wartości typu \verb#int# do konstruktora. Wierzchołek, z kolei reprezentowany jest przez zmienną \verb#int#. 

Dane (pary wierzchołków) są wczytywane z pliku, a potem w zależności od rodzaju kontenera i iteratora, sortowane. Dla kontenerów: 
\begin{itemize}
\item sekwencyjnych z iteratorem \verb#Random access# (\verb#vector#) wywoływana jest funkcja szablonu ograniczonego przez koncepty: \verb#Sequence# i \newline \verb#Random_access_iterator#.

\begin{lstlisting}[frame=single]
template<Sequence S, Random_access_iterator R>
void sortVertices(S &seq){
    sort(seq.begin(), seq.end());
}
\end{lstlisting}

\item sekwencyjnych z iteratorem \verb#Bidirectional# (\verb#list#) lub sekwencyjnych \emph{forward}(\verb#forward_list#) z iteratorem \verb#Forward# wywoływana jest funkcja szablonu ograniczonego przez koncepty: \verb#Sequence# i \verb#Bidirectiona-# \newline \verb#l_iterator# lub \verb#SequenceForward# i \verb#Forward_iterator#

\begin{lstlisting}[frame=single]
template<typename S, typename R>
requires (Sequence<S>() && 
Bidirectional_iterator<R>()) || 
(SequenceForward<S>() && Forward_iterator<R>())
void sortVertices(S &seq){
    seq.sort();
}
\end{lstlisting}

\item asocjacyjnych (\verb#set#) wywoływana jest funkcja szablonu ograniczonego przez koncept: \verb#Associative_container#.

\begin{lstlisting}[frame=single]
template<Associative_container A>
void sortVertices(A &seq){}
\end{lstlisting}

\end{itemize}
\newpage
\noindent Omawiany algorytm wykonuje funkcja \verb#determineEulerCycle#:
\begin{lstlisting}[frame=single]
template<typename E, typename V>
void determineEulerCycle(E &edges, V &vertices){
   int v = 0;
   bool condition = (checkIfGraphConnected(edges, 
   vertices, 0, v) && checkIfAllEdgesEvenDegree
   (edges, vertices));
    
   if(condition){
      cout << "Euler cycle:" << endl << endl << v;
      while(!edges.empty()){
         switch(getNeighboursCount(edges, v)){
            case 1 : {
               removeEdgeWithOneNeighbour(edges, v);
               break;
            }
            default: {
               removeEdgeWithMoreNeighbour(edges,
               v, vertices);
               break;
            }
         }
         cout <<" -> "<<v;
      }
      cout << endl << endl;
    } else {
       cout<<"Invalid graph."<<endl;
       if(!checkIfGraphConnected(edges, vertices,
       0, v))
          cout <<"Graph is not connected"<<endl;
       else if(!checkIfAllEdgesEvenDegree(edges,
       vertices))
          cout <<"Not all the edges are even"<<endl;
    }
}
\end{lstlisting}

Żeby algorytm się wykonał, muszą zostać spełnione dwa warunki: graf musi być spójny (za to odpowiedzialna jest funkcja \verb#checkIfGraphConnected#) i wszystkie krawędzie muszą być parzystego stopnia (\verb#checkIfAllEdgesEvenDegree#).

\verb#checkIfGraphConnected()#
\begin{lstlisting}[frame=single]
template<typename E, typename V>
bool checkIfGraphConnected(E &ed, V &vertices,
int x, int startVertice) {
    
   bool *visited = new bool[vertices.size()];
   for (int i = 0; i < vertices.size(); i++) 
      visited[i] = false;

   stack<int>stack;
   int vc = 0;
    
   stack.push(startVertice);
   visited[startVertice] = true;
    
   while (!stack.empty()) {
       int v = stack.top();
       stack.pop();
       vc++;
    
       for(typename E::iterator it = ed.begin();
          it != ed.end(); it++){
          if(it->getA() == v && !visited[it->getB()]){
             visited[it->getB()] = true;
             stack.push(it->getB());
          } else if(it->getB() == v &&
          !visited[it->getA()]){
             visited[it->getA()] = true;
             stack.push(it->getA());
          }
        }
    }

    delete [] visited;

    return (vc == vertices.size()-x);

}
\end{lstlisting}

Algorytm przechodzi przez graf, po kolei wrzuca odwiedzane wierzchołki na stos, zaznacza je w tablicy odwiedzonych (\verb#visited#) i zaraz zdejmuje z tego stosu, zwiększając licznik \verb#vc#. Robi to dopóki stos nie jest pusty. Zwraca warunek porównujący licznik \verb#vc# z rozmiarem kontenera wierzchołków (wszystkie wierzchołki zostały odwiedzone, czyli istnieją ścieżki między wierzchołkami, graf jest spójny).

\verb#checkIfAllEdgesEvenDegree#:
\begin{lstlisting}[frame=single]
template<typename E, typename V>
bool checkIfAllEdgesEvenDegree(E &edges, V &vertices){
    int counter = 0, i = 0;
    for(auto v : vertices){
        for(auto e : edges){
            if(e.getA() == v || e.getB() == v)
               counter++;
        }
        if(counter % 2 == 0) i++;
    }
    return (i == vertices.size()) ? true : false;
}
\end{lstlisting}

Zmienna \verb#i#  zwiększa się jeśli ilość wystąpień wierzchołka jest liczbą parzystą. Zwraca wartość \verb#true# jeśli zmienna \verb#i# jest równa liczbie elementów kontenera zawierającego wierzchołki (dla każdego wierzchołka zmienna \verb#i# zwiększała się o 1).

Jeśli warunek nie zostanie spełniony, użytkownik zostaje poinformowany o tym, że graf jest niepoprawny. W odwrotnej sytuacji, w pętli (dopóki kontener krawędzi nie jest pusty), wykonywana jest jedna dwóch operacji. Gdy wierzchołek ma jednego sąsiada, wywołuje się funkcja \verb#removeEdgeWithOneN-#\newline \verb#eighbour()#, a gdy więcej wierzchołków, funkcja \verb#removeEdgeWithMoreNei-# \newline \verb#ghbour()#. Pierwsza z nich ma dwa przeciążenia konceptowe:
\begin{itemize}

\item Dla kontenera sekwencyjnego lub kontenera sekwencyjnego \emph{forward}
\begin{lstlisting}[frame=single]
template<typename S> requires Sequence<S>()
|| SequenceForward<S>()
void removeEdgeWithOneNeighbour(S &edges, int &v){
    
    typename S::iterator it = 
    findElement<S>(edges, v);
    
    if(it->getA() == v) v = it->getB();
    else v = it->getA();

    deleteElementFromContainer(edges, it);
}
\end{lstlisting}

\item Dla kontenera asocjacyjnego:
\begin{lstlisting}[frame=single]
template<Associative_container A>
void removeEdgeWithOneNeighbour(A &edges, int &v){
    
    typename A::iterator it2;
    for(typename A::iterator it = edges.begin();
    it != edges.end(); it++){
        if(it->getA() == v || it->getB() == v){
            it2 = it;
            if(it->getA() == v) v = it->getB();
            else v = it->getA();
            it = prev(edges.end());
        }
    }

    edges.erase(it2);
}
\end{lstlisting}

Funkcja znajduje krawędź, dostając wierzchołek wychodzący. I wierzchołek znalezionej krawędzi przypisuje do tego przekazanego.

\end{itemize}

Druga \verb#removeEdgeWithMoreNeighbour()# wygląda:
\begin{lstlisting}[frame=single]
template<typename E, typename V>
void removeEdgeWithMoreNeighbour(E &edges, int &v, 
V &vertices){
    
    bool w = true;
    
    for(typename E::iterator i = edges.begin(); 
    i != edges.end() && w; i++){
        if(i->getA() == v && checkIfStillConnected
        (edges, *i, getZeroDegreeCount(edges, 
        vertices), v, vertices)){
            v = i->getB();
            deleteElementFromContainer(edges, i);
            w = false;
        } else if(i->getB() == v && 
        checkIfStillConnected(edges, 
        *i, getZeroDegreeCount(edges, vertices), v, 
        vertices)){
            v = i->getA();
            deleteElementFromContainer(edges, i);
            w = false;
        }
    }
}
\end{lstlisting}
\newpage
Jeśli wierzchołek ma więcej sąsiadów, wybiera tego który nie rozspójni grafu. Żeby to sprawdzić używa funkcji \verb#checkIfStillConnected#:
\begin{lstlisting}[frame=single]
template<typename E, typename V>
bool checkIfStillConnected(E &edges, Edge e, int x, 
int startVertice, V &vertices){
    
    E tmp;

    for(auto e : edges)
        addElementToContainer(tmp, e);

    bool w = true;
    for(typename E::iterator it = tmp.begin(); 
    it != tmp.end() && w; it++)
		if (it->getA() == e.getA() && 
		it->getB() == e.getB()) {
            deleteElementFromContainer(tmp, it);
            w = false;
        }
        
    return checkIfGraphConnected(tmp, vertices, 
    x, startVertice);
}
\end{lstlisting}

W celu sprawdzenia, czy graf po usunięciu jakiejś krawędzi dalej będzie spójny, potrzebny jest pomocniczy kontener. Zapisujemy do niego aktualne krawędzie, wyszukujemy w nim przekazaną krawędź i przekazujemy go do istniejącej już funkcji \verb#checkIfGraphConnected()#.\newline

\noindent Dodatkowe funkcje:

\noindent \verb#addElementToContainer#:\newline

\noindent Dla kontenera sekwencyjnego, ograniczona przez koncept \verb#Sequence#:

\begin{lstlisting}[frame=single]
template<Sequence S, typename Element>
void addElementToContainer(S &seq, Element e){
    seq.push_back(e);
}
\end{lstlisting}

\noindent Dla kontenera sekwencyjnego \emph{forward}, ograniczona przez koncept \verb#SequenceForward#:

\begin{lstlisting}[frame=single]
template<SequenceForward S, typename Element>
void addElementToContainer(S &seq, Element e){
    seq.push_front(e);
}
\end{lstlisting}

\noindent Dla kontenera asocjacyjnego, ograniczona przez koncept \verb#Associative_container#:

\begin{lstlisting}[frame=single]
template<Associative_container S, typename Element>
void addElementToContainer(S &seq, Element e){
    seq.insert(e);
}
\end{lstlisting}

\noindent Funkcja dodaje element do kontenera.\newline

\noindent \verb#findElement#:\newline \newline
\noindent Dla kontenera sekwencyjnego lub sekwencyjnego \emph{forward}, ograniczona przez koncept \verb#Sequence# lub \verb#SequenceForward#
\begin{lstlisting}[frame=single]
template<typename S>
requires Sequence<S>() || SequenceForward<S>()
typename S::iterator findElement(S &seq, int e){
    return find(seq.begin(), seq.end(), e);
}
\end{lstlisting}

\noindent Funkcja zwraca znaleziony element w kontenerze \newline

\noindent \verb#getVerticesSize#:
\begin{lstlisting}[frame=single]
template<typename S>
int getVerticesSize(typename S::iterator begin, 
typename S::iterator end){
    return distance(begin, end);
}
\end{lstlisting}

\noindent Funkcja zwraca rozmiar kontenera.\newline

\noindent \verb#deleteElementFromContainer#:\newline

\noindent Dla kontenera sekwencyjnego, ograniczona przez koncept \verb#Sequence#:
\begin{lstlisting}[frame=single]
template<Sequence S>
void deleteElementFromContainer(S &seq, 
typename S::iterator it){
    seq.erase(it);
}
\end{lstlisting}

\noindent Dla kontenera sekwencyjnego \emph{forward}, ograniczona przez koncept \verb#SequenceForward#:
\begin{lstlisting}[frame=single]
template<SequenceForward S>
void deleteElementFromContainer(S &seq, 
typename S::iterator it){
    seq.remove(*it);
}
\end{lstlisting}

\noindent Dla kontenera asocjacyjnego, ograniczona przez koncept \verb#Associative_container#:
\begin{lstlisting}[frame=single]
template<Associative_container S>
void deleteElementFromContainer(S &seq, 
typename S::iterator it){
    seq.erase(it);
}
\end{lstlisting}

\noindent Funkcja usuwa element z kontenera.

\end{document}